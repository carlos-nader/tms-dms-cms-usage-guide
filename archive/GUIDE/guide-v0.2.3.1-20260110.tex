\documentclass[11pt,a4paper]{article}

% --------------------------------------------------------------------------
% BASIC ENCODING AND LANGUAGE
% --------------------------------------------------------------------------
\usepackage[utf8]{inputenc}
\usepackage[T1]{fontenc}
\usepackage[english]{babel}

% --------------------------------------------------------------------------
% FONTS AND MICROTYPOGRAPHY
% --------------------------------------------------------------------------
\usepackage{lmodern}
\usepackage{microtype}

% --------------------------------------------------------------------------
% PAGE GEOMETRY AND LAYOUT
% --------------------------------------------------------------------------
\usepackage{geometry}
\geometry{a4paper, left=2.0cm, right=2.0cm, top=2.5cm, bottom=2.5cm}
\usepackage{setspace}
\onehalfspacing

% --------------------------------------------------------------------------
% COLORS AND LINKS
% --------------------------------------------------------------------------
\usepackage[table]{xcolor}
\definecolor{linkblue}{HTML}{004488}
\definecolor{linkred}{HTML}{882222}
\definecolor{headerblue}{HTML}{003366}
\definecolor{rowgray}{HTML}{F5F5F5}
\definecolor{subheadgray}{HTML}{E0E0E0}
\usepackage[pdfencoding=auto, psdextra, colorlinks=true, linkcolor=linkblue, citecolor=linkred, urlcolor=linkblue, breaklinks=true]{hyperref}
\usepackage{bookmark}

% --------------------------------------------------------------------------
% HEADERS AND FOOTERS
% --------------------------------------------------------------------------
\usepackage{fancyhdr}
\setlength{\headheight}{15pt}
\pagestyle{fancy}
\fancyhf{}
\fancyhead[L]{\leftmark}
\fancyhead[R]{\rightmark}
\fancyfoot[C]{\thepage}
\renewcommand{\headrulewidth}{0.4pt}
\renewcommand{\footrulewidth}{0pt}

% --------------------------------------------------------------------------
% TABLES AND MACROS
% --------------------------------------------------------------------------
\usepackage{booktabs}
\usepackage{array}
\usepackage{longtable}
\usepackage{tabularx}

% Custom Columns
\newcolumntype{L}[1]{>{\raggedright\arraybackslash}p{#1}}
\newcolumntype{C}[1]{>{\centering\arraybackslash}p{#1}}
\newcolumntype{R}[1]{>{\raggedleft\arraybackslash}p{#1}}

% Macro for Visual Reference Links
\newcommand{\imglink}[1]{\hspace{2pt}\hyperref[#1]{\scriptsize\textbf{[Fig]}}}

% HOTAS table environment (per Briefing v0.2.0.1)
\newenvironment{hotastable}[1]{%
  \small
  \renewcommand{\arraystretch}{1.25}
  \begin{longtable}{L{1.6cm} L{1.0cm} L{1.0cm} L{3.4cm} L{5.8cm} L{1.4cm} L{1.4cm}}
  \caption{#1}\\
  \rowcolor{headerblue}
  \textbf{\color{white}State} &
  \textbf{\color{white}Dir} &
  \textbf{\color{white}Act} &
  \textbf{\color{white}Function} &
  \textbf{\color{white}Effect / Nuance} &
  \textbf{\color{white}Dash34} &
  \textbf{\color{white}Train} \\
  \endfirsthead
  \rowcolor{headerblue}
  \textbf{\color{white}State} &
  \textbf{\color{white}Dir} &
  \textbf{\color{white}Act} &
  \textbf{\color{white}Function} &
  \textbf{\color{white}Effect / Nuance} &
  \textbf{\color{white}Dash34} &
  \textbf{\color{white}Train} \\
  \endhead
  \multicolumn{7}{r}{\small\emph{Continued on next page}}\\
  \endfoot
  \endlastfoot
}{%
  \end{longtable}
}

% --------------------------------------------------------------------------
% SIMPLE REFERENCE MACROS FOR BMS DOCS
% --------------------------------------------------------------------------
\providecommand{\dashref}[1]{Dash-34~\S~#1}
\providecommand{\dashone}[1]{Dash-1~\S~#1}
\providecommand{\trnref}[1]{TRN~#1}
\providecommand{\trnman}{BMS Training Manual 4.38.1}
\providecommand{\bmsver}{Falcon BMS~4.38.1}
\providecommand{\dashrefs}[1]{\textit{TO 1F-16CMAM-34-1-1}, Dash-34, sections \texttt{#1}}

% --------------------------------------------------------------------------
% VERSION CONTROL MACROS
% --------------------------------------------------------------------------
\newcommand{\docversion}{0.2.3.1}
\newcommand{\docbuild}{20260110}
\newcommand{\docstartdate}{05 January 2026}
\newcommand{\docenddate}{DD MMM 2026}
\newcommand{\chapterscompletedof}{2/7}
\newcommand{\tablesfilledpct}{0\%}
\newcommand{\fulldocversion}{\docversion+\docbuild}

% --------------------------------------------------------------------------
% GRAPHICS
% --------------------------------------------------------------------------
\usepackage{graphicx}
\graphicspath{{fig/}}

% --------------------------------------------------------------------------
% TITLE
% --------------------------------------------------------------------------
\title{TMS, DMS and CMS Usage Guide for \bmsver}
\author{Carlos ``Metal'' Nader}
\date{Version \fulldocversion{} | Progress: Chapters \chapterscompletedof{} | Tables \tablesfilledpct{} | January 2026}

% --------------------------------------------------------------------------
% DOCUMENT
% --------------------------------------------------------------------------
\begin{document}

\maketitle

\pagenumbering{roman}
\newpage
\tableofcontents
\newpage
\pagenumbering{arabic}

% --------------------------------------------------------------------------
% CHAPTER 1: INTRODUCTION
% --------------------------------------------------------------------------
\section{Introduction}

This document is a community-made reference guide for \bmsver, focused on practical use of three specific HOTAS controls: the Target Management Switch (TMS), the Display Management Switch (DMS), and the Countermeasures Management Switch (CMS). Although this guide was developed under Falcon BMS 4.38.1, the fundamental behavior of these switches has remained constant since at least Falcon BMS 4.36, making this guide applicable to almost any player of Falcon BMS.

Although other controls exist on the F-16 throttle and stick---such as the Communication Switch, the Dogfight/MRM Override, and the RDR Cursor Enable control---they are mentioned only when essential to understanding the behavior and context of TMS, DMS, and CMS.\footnote{Falcon BMS core avionics and weapons behaviour are documented in \dashone{2} and \dashref{2}. \trnman{} describes how these systems are trained in practice.}

Its goal is to reorganize information that is spread across the Dash-1, Dash-34 and the BMS Training Manual into mode-based tables and short explanations, so that virtual pilots can quickly understand what each switch press does in a given context.\footnote{See \dashref{2.1.5} (Hands-On Controls) and the foreword of \trnman{} for the role of TMS, DMS and CMS in BMS training.}

This is the \emph{TMS, DMS and CMS Usage Guide---Version \fulldocversion}, prepared between \docstartdate{} and \docenddate{}, and created with extensive assistance from an AI language model (Perplexity AI) to help structure, cross-reference and format the material. The human author remains fully responsible for every choice of content, interpretation and final wording, and any mistakes or omissions are attributable to the author alone, not to the AI system.

This work is entirely unofficial. The author is not affiliated with Benchmark Sims, MicroProse, any real-world air force, or any aircraft or weapons manufacturer. All interpretations, simplifications, errors and omissions in this document are solely the responsibility of the author and must not be attributed to the Falcon BMS development team or to any real-world organization.\footnote{Compare the official disclaimer and copyright statements in the foreword of \trnman{}.} Nothing in this document should ever be used for real-world operations, training, or procedures.

% --------------------------------------------------------------------------
% SECTION 1.1: DEVELOPMENT TIMELINE
% --------------------------------------------------------------------------
\subsection{Development timeline and status}

This guide was developed in structured phases, beginning \docstartdate{}. Current development status and targets are shown in Table 1 below.

\begin{table}[h]
\centering
\caption{Development Status Snapshot}
\begin{tabular}{L{3.2cm} L{2.5cm} L{2.5cm}}
\toprule
\textbf{Metric} & \textbf{Current} & \textbf{Target} \\
\midrule
Start Date & \docstartdate & --- \\
Current Version & \fulldocversion & --- \\
Chapters Complete & \chapterscompletedof & 7/7 \\
Tables Filled & \tablesfilledpct & 100\% \\
\bottomrule
\end{tabular}
\end{table}

The development roadmap is structured in three phases: (1) \emph{Chapter scaffolding} (Versions 0.1.0--0.7.0), during which all chapters receive narrative content and table structures; (2) \emph{Table population} (Versions 1.0.0--1.0.5), during which all tables are filled with complete HOTAS behavior descriptions and diagrams are generated; and (3) \emph{Review and release} (Versions 2.0.0-RC1 through 2.0.0-Stable), during which content is reviewed for accuracy, consistency, and clarity. Each phase produces a versioned PDF artifact, and all versions are archived for traceability.

% --------------------------------------------------------------------------
% SECTION 1.2: SCOPE AND PURPOSE
% --------------------------------------------------------------------------
\subsection{Scope and purpose}

This guide focuses exclusively on three HOTAS switches: the Target Management Switch (TMS), the Display Management Switch (DMS), and the Countermeasures Management Switch (CMS). Although other controls exist on the F-16 throttle and stick---such as the Communication Switch, the Dogfight/MRM Override, and the RDR Cursor Enable control---they are mentioned only when essential to understanding the behavior and context of TMS, DMS, and CMS.

This is not a comprehensive HOTAS or avionics manual. Instead, it is a usage guide organized by context, with emphasis on practical tables that show what each switch input does in specific flight modes, sensor configurations, and weapon employment scenarios. The guide bridges information scattered across official documentation and training missions, making it immediately accessible to pilots who ask: \emph{``In this radar mode, what does TMS Up do?''} or \emph{``How do I cycle through MFD formats with the DMS?''}

The guide assumes knowledge of basic F-16 operation and familiarity with master modes (NAV, A-A, A-G, DGFT). It does not replace the Dash-34 or Training Manual; rather, it complements them by organizing TMS/DMS/CMS behavior into searchable tables with cross-references back to official sources and practical training missions where each behavior can be practiced.

% --------------------------------------------------------------------------
% SECTION 1.3: VERSION, AUTHORSHIP AND AI ASSISTANCE
% --------------------------------------------------------------------------
\subsection{Version, authorship and AI assistance}

\textbf{Document Version:} \fulldocversion{} (Progress: Chapters \chapterscompletedof{} | Tables \tablesfilledpct)

\textbf{Falcon BMS Version:} 4.38.1 (Update 1)

\textbf{Authorship:} This guide was created by a member of the Falcon BMS community with structured assistance from AI language models (Perplexity AI). The human author identified scope, validated content against official Falcon BMS documentation, made all organizational and editorial decisions, and bears full responsibility for the guide's accuracy and presentation. AI tools were used for research organization, cross-referencing, and text generation---not for defining technical correctness.

\textbf{Disclaimer:} This is an unofficial, community-made document not affiliated with, endorsed by, or affiliated with Benchmark Sims, MicroProse, any military organization, or any aircraft or weapons manufacturer. All technical content is paraphrased in original words from official BMS documentation. No copyrighted material is reproduced directly. The guide is provided ``as-is'' for educational and simulation training purposes only.

\textbf{Copyright \& Sharing:} This guide may be freely copied, printed, translated, and shared within the Falcon BMS community for non-commercial use. Derivative works and contributions are encouraged, provided that proper credit is given to the original author and no derivative version claims official status.

% --------------------------------------------------------------------------
% SECTION 1.4: SOURCES AND REFERENCES
% --------------------------------------------------------------------------
\subsection{Sources and references}

This guide is based on the following primary Falcon BMS documents consulted during research and development:

\begin{enumerate}
  \item \textbf{TO BMS 1F-16CMAM-34-1-1} (Dash-34, Change 4.38) -- Avionics and Nonnuclear Weapons Delivery Flight Manual
  \item \textbf{BMS Training Manual 4.38.1} (October 2025) -- Training missions and learning objectives
  \item \textbf{TO BMS 1F-16CMAM-1} (Dash-1) -- F-16 Aircraft Systems, Normal and Abnormal Procedures
  \item \textbf{BMS User Manual 4.38} -- BMS user interface and setup
  \item \textbf{MCH 11-F16 Vol 5} (May 1996) -- F-16 Flight Manual Vol 5 (Surface Attack and Weapons Delivery)
  \item \textbf{Falcon BMS Cockpit Arrangement Diagrams} (Multiple blocks)
\end{enumerate}

% --------------------------------------------------------------------------
% SECTION 1.5: DOCUMENT STRUCTURE AND HOW TO READ IT
% --------------------------------------------------------------------------
\subsection{Document structure and how to read it}

\subsubsection{Part A: Foundational Sections}

Foundational sections establish core HOTAS concepts: Sensor of Interest (SOI), short vs. long press timing, master modes, and an overview of TMS/DMS/CMS roles. Read this first if you are new to the F-16 or HOTAS in general.

\subsubsection{Part B: Switch-Specific Sections}

Sections focused on TMS, DMS, and CMS each contain detailed tables using the \texttt{hotastable} environment.
\\\\
\textbf{Table structure:}

Each table follows a seven-column format:

\begin{center}
\begin{tabular}{L{1.6cm} L{1.0cm} L{1.0cm} L{3.4cm} L{5.8cm} L{1.4cm} L{1.4cm}}
\toprule
\textbf{State} & \textbf{Dir} & \textbf{Act} & \textbf{Function} & \textbf{Effect / Nuance} & \textbf{Dash34} & \textbf{Train} \\
\midrule
Condition & Up & Shrt & Name & Explanation & Ref & TRN \\
\bottomrule
\end{tabular}
\end{center}

\textbf{How to find information:}

\begin{enumerate}
\item Identify the \textbf{master mode and sensor/weapon context} from the section title (e.g., ``TMS in Air-to-Air -- FCR CRM'').
\item Find the \textbf{State} within the table (e.g., ``Search'' vs. ``STT'').
\item Determine the \textbf{Direction} and \textbf{Action} (Short/Long).
\item Read the \textbf{Effect} and check the \textbf{Dash34} or \textbf{Training} reference.
\end{enumerate}

\subsubsection{Part C: Training and Visual Reference}

Training reference section links this guide to the 33 BMS training missions, offering a recommended progression and example tactical flows. Use this section to plan your training sequence.

Visual reference section provides schematic diagrams of the TMS, DMS, and CMS hats with arrows and short labels for each direction in common contexts. These are quick-reference visuals; always consult the tables for complete behavior descriptions.

\subsubsection{Part D: Appendices}

Appendices note any differences in TMS/DMS/CMS behavior across F-16 blocks and variants (e.g., Block 50/52 vs. Block 40/42), and provide a comprehensive index of all major tables and their locations.

% --------------------------------------------------------------------------
% CHAPTER 2: HOTAS FUNDAMENTALS (PLACEHOLDER)
% --------------------------------------------------------------------------
\section{HOTAS fundamentals}

\subsection{Sensor of Interest (SOI) and display logic}
[Content to be developed in next phase]

\subsection{Short vs long presses and timing}
[Content to be developed in next phase]

\subsection{Master modes and context-sensitive behaviour}
[Content to be developed in next phase]

\subsection{Overview of TMS, DMS and CMS}
[Content to be developed in next phase]

% --------------------------------------------------------------------------
% CHAPTER 3: TMS (PLACEHOLDER)
% --------------------------------------------------------------------------
\section{TMS -- Target Management Switch}

\subsection{Concept and general behaviour}

[Content to be developed in next phase]

\subsection{TMS and Situational Awareness displays}

[Content to be developed in next phase]

\subsection{TMS in Air-to-Air}

[Content to be developed in next phase]

\subsection{TMS in Air-to-Ground}

[Content to be developed in next phase]

\subsection{TMS in weapon employment}

[Content to be developed in next phase]

\subsection{TMS -- Block / variant notes}

[Content to be developed in next phase]

% --------------------------------------------------------------------------
% CHAPTER 4: DMS (PLACEHOLDER)
% --------------------------------------------------------------------------
\section{DMS -- Display Management Switch}

\subsection{Concept and Sensor of Interest (SOI)}

[Content to be developed in next phase]

\subsection{DMS in MFDS format selection and SWAP}

[Content to be developed in next phase]

\subsection{DMS in sensor and weapon context}

[Content to be developed in next phase]

\subsection{DMS -- Block and variant notes}

[Content to be developed in next phase]

% --------------------------------------------------------------------------
% CHAPTER 5: CMS
% --------------------------------------------------------------------------
\section{CMS -- Countermeasures Management Switch}

\subsection{Concept and Interaction with CMDS / ECM / RWR}

\subsubsection{Concept}

The Countermeasures Management Switch (CMS) is a four-direction hat switch mounted on the control stick that serves as the pilot's primary control interface to the F-16's integrated electronic warfare (EW) defensive systems: the ALE-47 CMDS (automatic chaff/flare dispenser), ECM systems (external pods or internal avionics), and avionics-based threat defeat systems. The CMS supervises the aircraft's defensive response by controlling defensive program selection, managing ECM operational modes, and granting or withholding consent authority to all defensive subsystems.

Its role is to grant the pilot rapid tactical control over the aircraft's defensive posture. This control is operationally critical because defensive decisions frequently occur during high-G maneuvering when hand position cannot be redirected to distant cockpit panels. A pilot executing a 6-G defensive turn cannot simultaneously reach the CMDS MODE knob on the left console or the ECM control panel without abandoning aircraft control. By placing the CMS within thumb reach during full-stick maneuver, the design ensures that no tactical scenario regardless of G-load or workload forces the pilot to choose between aircraft control and defensive system authority.

RWR, although not directly linked to CMS, is more than a display device: it is the decision engine for both CMDS and ECM. The RWR continuously evaluates detected threat radars, classifies them (SEARCH, TRACK, LAUNCH), assigns threat priority, and communicates this information to both the ALE-47 CMDS (in AUTO or SEMI mode) and the ECM system (for band selection or jamming initiation).

For in-depth explanations about CMDS, ECM and RWR operation, see \dashrefs{2.7.1, 2.7.2, and 2.7.3}, respectively. This section focuses exclusively on CMS usage and control interface. As presented in the preceding chapters, condensed diagrams of ECM and other throttle and flight stick functionalities can be found on section \dashrefs{2.1.5}. Below is an image of the F-16 Flight Stick, with the CMD switch location.

\begin{figure}[h]
\centering
\includegraphics[width=0.65\textwidth]{F-16_Side_Stick_Controller-1.jpg}
\caption{F-16 Throttle and Flight Stick. Image by Falconpedia (\url{falcon4.wikidot.com}), via Wikimedia Commons (\url{https://commons.wikimedia.org/wiki/File:F-16_Side_Stick_Controller.jpg}), licensed under the Creative Commons Attribution-Share Alike 3.0 Unported (CC BY-SA 3.0) license.}
\label{fig:f16_hotas_cms_location}
\end{figure}

\subsubsection{Interaction with CMDS / ECM}

Operationally, the CMS manages two distinct defensive layers: CMDS and ECM (in both configurations: internal avionics or external pod). All CMS button pressings will be detailed in the next section.

Differences in F-16 Blocks or variants, especially regarding ECM, will be discussed in Section~5.3.

\begin{enumerate}
  \item \textbf{ECM (External Pod):} Controls the external ECM pod's operational state through pilot-directed transmission modes and consent authority.
  \item \textbf{ECM (Integrated IDIAS):} Controls the integrated ECM system through automatic threat-reactive modes.
  \item \textbf{CMDS in Manual Mode:} Allows the pilot to execute pre-selected dispenser programs on demand, independent of automatic systems.
  \item \textbf{CMDS in Automatic/Semi-Automatic Modes:} Authorizes the ALE-47 CMDS to respond autonomously to RWR-detected threats when operating in AUTO or SEMI mode.
\end{enumerate}

% --------------------------------------------------------------------------
% SECTION 5.2: CMS SWITCH ACTUATION (FULL WIP, LEVEL-ADJUSTED)
% --------------------------------------------------------------------------
\subsection{CMS Switch Actuation}
\label{sec:C5-S2-cms-actuation}

The Countermeasures Management Switch (CMS) is a four-direction hat switch located at the flight stick that provides pilots with rapid, direct control over the F-16's defensive systems during demanding tactical situations. This section tabulates all CMS button combinations, organized by operational layer: Counter Measures (CMDS manual/automatic/semi modes) and Defensive Avionics integration (jamming) with both external ECM pods (ALQ-131/ALQ-184) and internal avionics (IDIAS). CMS actuation is independent of the Master Mode currently selected.

Each table entry specifies:

\begin{itemize}
  \item \textbf{State}: The operational context (master mode, CMDS mode, sensor state).
  \item \textbf{Direction}: The physical direction for pressing the CMS hat (Up, Down, Left, Right).
  \item \textbf{Action}: The press type (Short, Long, Long Hold).
  \item \textbf{Function}: What the CMS command activates or controls.
  \item \textbf{Effect / Nuance}: The resulting system behavior, including tactics and constraints.
  \item \textbf{Dash34}: Reference section in the Dash-34 manual.
  \item \textbf{Training}: Recommended BMS training missions for hands-on practice.
\end{itemize}

% 5.2.1 CMS Actuation with CMDS
\subsubsection{CMS Actuation with CMDS}
\label{sec:C5-S2-S1-cms-cmds}

The ALE-47 CMDS (Automatic Chaff and Flare Dispensing System) provides three operational modes: Manual (MAN), Automatic (AUTO), and Semi-Automatic (SEMI). Each mode grants the pilot different levels of control and autonomy over chaff and flare dispensing. The CMS is the primary interface for program execution and consent authority across all three modes.

% 5.2.1.1 Manual Mode
\paragraph{Manual Mode:}
\label{sec:C5-S2-S1-S1-cms-cmds-man}

The CMDS Manual (MAN) mode grants the pilot direct, program-by-program control over countermeasure expenditure. Each CMS direction selects or executes a specific program. This mode is recommended when threat types are well-known or when chaff/flare inventory must be conserved, or by pilot choice.

\begin{hotastable}{CMS Actuation with CMDS Manual Mode}
CMDS MAN & Up & Short & Execute Program 1--4 &
Runs the program selected via the CMDS panel PRGM knob once per press. No threat sensing; purely pilot-commanded. Overrides any AUTO dispensing, if running.
& \dashref{2.7.2.2} & \trnref{18 (BARCAP)}, \trnref{28 (SEAD-EW)} \\
CMDS MAN & Left & Short & Execute Program 6 &
Flare-only program. Often pre-configured for close-range air-to-air engagements or MANPAD defense. No dependency on PRGM knob.
& \dashref{2.7.2.2} & \trnref{19 (GUN AIM)} \\
\end{hotastable}

% 5.2.1.2 Automatic Mode
\paragraph{Automatic Mode:}
\label{sec:C5-S2-S1-S2-cms-cmds-auto}

The CMDS Automatic (AUTO) mode enables the ALE-47 CMDS to dispense chaff/flare programs continuously in response to RWR-detected threats, without requiring pilot consent for each event. Pilot consent is given once by pressing CMS Aft; dispensing continues until CMS Right is pressed or expendables are exhausted.

\begin{hotastable}{CMS Actuation with CMDS Automatic Mode}
CMDS AUTO & Up & Short & Execute Program 1--4 &
Manual override: runs the selected program once, interrupting any ongoing AUTO dispensing. After the manual program completes, AUTO resumes if threat persists. Useful for pilot override in high-threat scenarios.
& \dashref{2.7.2.2} & \trnref{18 (BARCAP)}, \trnref{28 (SEAD-EW)} \\
CMDS AUTO & Left & Short & Execute Program 6 &
Manual flare-only program, overrides AUTO. After execution, AUTO resumes.
& \dashref{2.7.2.2} & \trnref{19 (GUN AIM)} \\
CMDS AUTO & Aft & Short & Give Consent; Enable AUTO Dispensing &
CMS Aft grants consent for AUTO CMDS. RWR-detected threats trigger automatic program dispensing (selected via PRGM knob). Dispensing continues until threat clears or CMS Right is pressed. Consent state persists even if pilot switches to MAN mode; re-engaging AUTO will resume auto-dispense.
& \dashref{2.7.2.1} & \trnref{18 (BARCAP)}, \trnref{28 (SEAD-EW)} \\
CMDS AUTO & Right & Short & Cancel Consent; Disable AUTO &
CMS Right removes CMDS consent and places the ALE-47 in Standby. Automatic dispensing halts immediately. Pilot must re-issue CMS Aft to resume AUTO operation or use the system manually.
& \dashref{2.7.2.1} & \trnref{18 (BARCAP)}, \trnref{28 (SEAD-EW)} \\
\end{hotastable}

% 5.2.1.3 Semi-Automatic Mode
\paragraph{Semi-Automatic Mode:}
\label{sec:C5-S2-S1-S3-cms-cmds-semi}

Semi-Automatic (SEMI) mode allows the ALE-47 to prompt the pilot for consent on a per-threat basis. When the RWR detects a threat requiring countermeasures, the CMDS displays ``DISPENSE RDY'' on the control unit and sounds a ``COUNTER'' voice message, requesting pilot consent via CMS Aft.

\begin{hotastable}{CMS Actuation with CMDS Semi-Automatic Mode}
CMDS SEMI & Up & Short & Execute Program 1--4 &
Manual override: runs the selected program once, independent of RWR threat state. After execution, CMDS returns to monitoring for threats and issuing ``COUNTER'' prompts.
& \dashref{2.7.2.2} & \trnref{18 (BARCAP)}, \trnref{28 (SEAD-EW)} \\
CMDS SEMI & Left & Short & Execute Program 6 &
Manual flare-only program, independent of SEMI threat detection. After execution, CMDS resumes SEMI monitoring.
& \dashref{2.7.2.2} & \trnref{19 (GUN AIM)} \\
CMDS SEMI & Aft & Short & Give Consent; Dispense One Program &
When CMDS issues ``COUNTER'' (threat detected), pilot presses CMS Aft to execute one instance of the selected program. If threat persists or a new threat appears, CMDS will issue ``COUNTER'' again. Consent state is tracked; switching to AUTO while consent active will trigger immediate AUTO dispensing on next threat.
& \dashref{2.7.2.1} & \trnref{18 (BARCAP)}, \trnref{28 (SEAD-EW)} \\
CMDS SEMI & Right & Short & Cancel Consent; Return to Standby &
CMS Right removes SEMI consent. CMDS halts monitoring and returns to Standby. ``COUNTER'' messages cease.
& \dashref{2.7.2.1} & \trnref{18 (BARCAP)}, \trnref{28 (SEAD-EW)} \\
\end{hotastable}

% 5.2.2 CMS Actuation with ECM
\subsubsection{CMS Actuation with ECM}
\label{sec:C5-S2-S2-cms-ecm}

External ECM pods (ALQ-131, ALQ-184) and internal IDIAS (Improved Defensive Internal Avionic System) provide pilot-controlled jamming across frequency bands. The CMS Aft position controls transmit authority for external pods; CMS Left cycles modes for internal IDIAS. Both systems interact with the RF switch and respect landing gear constraints.

% 5.2.2.1 External ECM Pod
\paragraph{External ECM Pod (ALQ-131 / ALQ-184):}
\label{sec:C5-S2-S2-S1-cms-ecm-external}

External ECM pods provide pilot-controlled jamming across five frequency-band programs. The CMS Aft position grants ``ECM consent,'' enabling the pod to transmit at the XMIT switch setting (modes 1, 2, or 3). The ECM Enable light on the miscellaneous panel indicates consent state.

\begin{hotastable}{CMS Actuation with External ECM Pod}
ECM Pod & Aft & Short & Enable ECM Transmit; Grant Consent &
CMS Aft illuminates the ECM Enable light and permits the external ECM pod to transmit in the mode set by the XMIT switch on the ECM control panel (XMIT 1: AUTO Avionics Priority; XMIT 2: AUTO ECM Priority; XMIT 3: Continuous Jam). Pod continues transmitting as long as ECM is not cancelled by the pilot or RF switch is moved away from NORM.
& \dashref{2.7.4.2.5} & \trnref{28 (SEAD-EW)} \\
ECM Pod & Right & Short & Disable ECM Transmit; Remove Consent &
CMS Right extinguishes the ECM Enable light and places the ECM pod in Standby, halting transmission immediately. Pod will not transmit until CMS Aft is re-issued.
& \dashref{2.7.4.2.5} & \trnref{28 (SEAD-EW)} \\
\end{hotastable}

% 5.2.2.2 Internal ECM (IDIAS)
\paragraph{Internal ECM (IDIAS):}
\label{sec:C5-S2-S2-S2-cms-ecm-idias}

Internal avionics ECM (IDIAS: Improved Defensive Internal Avionic System) automatically selects frequency bands to jam based on RWR threat priority. CMS Left cycles through operational modes (Standby, Avionics Priority, ECM Priority). CMS Aft is not used for IDIAS; mode control is via CMS Left and the XMTR button on the ECM panel.

\begin{hotastable}{CMS Actuation with Internal ECM (IDIAS)}
IDIAS ECM & Left & Short & Cycle ECM Operational Mode &
Each short press of CMS Left advances the ECM mode: STBY $\rightarrow$ AVNC (Avionics Priority) $\rightarrow$ ECM (ECM Priority) $\rightarrow$ AVNC $\rightarrow$ ECM (cycles).
& \dashref{2.7.4.1.2} & \trnref{28 (SEAD-EW)} \\
IDIAS ECM & Right & Short & Set ECM to Standby &
CMS Right forces IDIAS ECM into STBY mode, halting all jamming operations. Requires CMS Left cycling to return to AVNC or ECM mode.
& \dashref{2.7.4.1.2} & \trnref{28 (SEAD-EW)} \\
\end{hotastable}

% 5.2.3 CMS Consent and Constraints
\subsubsection{CMS Consent and Constraints}
\label{sec:C5-S2-S3-cms-constraints}

This subsection clarifies the relationship between CMDS consent (CMS Aft) and ECM transmit authority (CMS Aft for external pod). Understanding these interactions is critical for effective defensive posture management during high-workload combat operations.

\begin{hotastable}{CMS Interaction with CMDS and ECM (Consent and Constraints)}
CMDS AUTO/SEMI + ECM Pod & Aft & Short & Joint Consent (CMDS + ECM) &
Single CMS Aft command grants consent to \textit{both} the ALE-47 CMDS (AUTO/SEMI) and the external ECM pod. There's no distinction between the two; both systems respond to the same CMS Aft press. This unified control maximizes pilot situational awareness and frees workload during combat maneuvering.
& \dashref{2.7.2.1}, \dashref{2.7.4.2.5} & \trnref{18 (BARCAP)}, \trnref{28 (SEAD-EW)} \\
\end{hotastable}

% 5.2.4 Important Operational Notes
\subsubsection{Important Operational Notes}
\label{sec:C5-S2-S4-cms-notes}

The CMS provides rapid, tactile access to CMDS program selection and ECM transmit authority without requiring the pilot to manipulate distant panels during high-G maneuvering. Mastery of CMS actuation across all CMDS modes (MAN, SEMI, AUTO) and ECM configurations (external pod, IDIAS) is essential for effective defensive operations. Pilots must understand the consent state model, RF switch interactions, and inventory management to avoid unintended dispensing or system saturation.

\paragraph*{Consent State Tracking:}
In AUTO and SEMI modes, the CMDS tracks the consent state even if the pilot temporarily switches to MAN mode. If the pilot gives CMS Aft consent in AUTO, then switches the CMDS MODE knob to MAN, the consent state is retained. Upon re-engaging AUTO without issuing CMS Aft again, the CMDS will immediately begin dispensing if a threat is detected. This behavior can be exploited for rapid mode switching during combat but may also lead to unintended dispensing if not carefully managed.

\paragraph*{Bingo Quantity Behavior:}
If expendables (chaff or flare) fall to or below the bingo quantity, the CMDS will still request consent (CMS Aft) and continue dispensing. The ``LOW'' and ``OUT'' voice messages alert the pilot to low or exhausted inventory, but dispensing does not automatically stop. Pilot must monitor EWS upfront pages and manually manage inventory via CMDS MAN or by pressing CMS Right to inhibit AUTO.

\paragraph*{RF Switch Override:}
The RF switch on the throttle is a master control for ECM transmission. Moving the RF switch away from NORM (e.g., to QUIET or SILENT) overrides any previous CMS Aft command and places both the external ECM pod and internal IDIAS in Standby. Returning RF to NORM does \textit{not} automatically restore transmission; the pilot must re-issue CMS Aft.

\paragraph*{ECM Consent vs. CMDS Consent:}
A single CMS Aft press grants consent to both the ALE-47 CMDS and the external ECM pod. The pilot does not issue separate commands; the CMS Aft action is unified. However, internal IDIAS uses CMS Left for mode cycling, not CMS Aft. This distinction is critical for aircraft configured with IDIAS.

\paragraph*{Ground Operations Safety:}
On the ground, ECM pods are held in Standby for safety reasons. Pilots must not hold CMS Aft while on the ground in the vicinity of personnel, as the ECM pod may radiate and pose a hazard. Ground personnel must be clear before the pilot engages ECM for pre-flight high-level BIT (Built-In Test). Once airborne, ECM consent (CMS Aft) can be issued and maintained as tactically required.

% --------------------------------------------------------------------------
% SECTION 5.3: CMS BLOCK / VARIANT NOTES (PLACEHOLDER)
% --------------------------------------------------------------------------
\subsection{CMS -- Block and variant notes}
[Content to be developed in next phase]

% --------------------------------------------------------------------------
% CHAPTER 6: TRAINING REFERENCES (PLACEHOLDER)
% --------------------------------------------------------------------------
\section{Training references and practical flows}
\subsection{How to use this guide with BMS training missions}

[Content to be developed in next phase]

\subsection{Recommended progression}

[Content to be developed in next phase]

\subsection{Example flows for typical missions}

[Content to be developed in next phase]

% --------------------------------------------------------------------------
% CHAPTER 7: VISUAL REFERENCE (PLACEHOLDER)
% --------------------------------------------------------------------------
\section{HOTAS visual reference}
\subsection{F-16 HOTAS overview}

[Content to be developed in next phase]

\subsection{TMS diagrams}

[Content to be developed in next phase]

\subsection{DMS diagrams}

[Content to be developed in next phase]

\subsection{CMS diagrams}

[Content to be developed in next phase]

% --------------------------------------------------------------------------
% APPENDICES
% --------------------------------------------------------------------------
\appendix

\section{Block / variant overview}

\subsection{F-16CM Block 50/52}

[Content to be developed in next phase]

\subsection{F-16C/D Block 40/42}

[Content to be developed in next phase]

\subsection{F-16AM/BM MLU}

[Content to be developed in next phase]

\subsection{F-16I Sufa and Israeli variants}

[Content to be developed in next phase]

\subsection{Other export variants}

[Content to be developed in next phase]

\section{Tables index}

\subsection{TMS tables}

[To be populated]

\subsection{DMS tables}

[To be populated]

\subsection{CMS tables}

[To be populated]

\end{document}
