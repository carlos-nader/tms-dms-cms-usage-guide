\documentclass[11pt,a4paper]{article}

% --------------------------------------------------------------------------
% BASIC ENCODING AND LANGUAGE
% --------------------------------------------------------------------------

\usepackage[utf8]{inputenc}
\usepackage[T1]{fontenc}
\usepackage[english]{babel}

% --------------------------------------------------------------------------
% FONTS AND MICROTYPOGRAPHY
% --------------------------------------------------------------------------

\usepackage{lmodern}
\usepackage{microtype}

% --------------------------------------------------------------------------
% PAGE GEOMETRY AND LAYOUT
% --------------------------------------------------------------------------

\usepackage{geometry}
\geometry{a4paper, left=2.0cm, right=2.0cm, top=2.5cm, bottom=2.5cm}
\usepackage{setspace}
\onehalfspacing

% --------------------------------------------------------------------------
% COLORS AND LINKS
% --------------------------------------------------------------------------

\usepackage[table]{xcolor}
\definecolor{linkblue}{HTML}{004488}
\definecolor{linkred}{HTML}{882222}
\definecolor{headerblue}{HTML}{003366}
\definecolor{rowgray}{HTML}{F5F5F5}
\definecolor{subheadgray}{HTML}{E0E0E0}
\usepackage[pdfencoding=auto, psdextra, colorlinks=true, linkcolor=linkblue, citecolor=linkred, urlcolor=linkblue, breaklinks=true]{hyperref}
\usepackage{bookmark}

% --------------------------------------------------------------------------
% HEADERS AND FOOTERS
% --------------------------------------------------------------------------

\usepackage{fancyhdr}
\setlength{\headheight}{15pt}
\pagestyle{fancy}
\fancyhf{}
\fancyhead[L]{\leftmark}
\fancyhead[R]{\rightmark}
\fancyfoot[C]{\thepage}
\renewcommand{\headrulewidth}{0.4pt}
\renewcommand{\footrulewidth}{0pt}

% --------------------------------------------------------------------------
% TABLES AND MACROS
% --------------------------------------------------------------------------

\usepackage{booktabs}
\usepackage{array}
\usepackage{longtable}
\usepackage{tabularx}

% Custom Columns
\newcolumntype{L}[1]{>{\raggedright\arraybackslash}p{#1}}
\newcolumntype{C}[1]{>{\centering\arraybackslash}p{#1}}
\newcolumntype{R}[1]{>{\raggedleft\arraybackslash}p{#1}}

% Macro for Visual Reference Links
\newcommand{\imglink}[1]{\hspace{2pt}\hyperref[#1]{\scriptsize\textbf{[Fig]}}}

% HOTAS table environment
% CORRECTED: \end commands position corrected
\newenvironment{hotastable}[1]{%
  \small
  \renewcommand{\arraystretch}{1.25}
  \begin{longtable}{L{1.6cm} L{1.0cm} L{1.0cm} L{3.4cm} L{5.8cm} L{1.4cm} L{1.4cm}}
  \caption{#1}\\
  \rowcolor{headerblue}
  \textbf{\color{white}State} &
  \textbf{\color{white}Dir} &
  \textbf{\color{white}Act} &
  \textbf{\color{white}Function} &
  \textbf{\color{white}Effect / Nuance} &
  \textbf{\color{white}Dash34} &
  \textbf{\color{white}Train} \\
  \endfirsthead
  %
  \rowcolor{headerblue}
  \textbf{\color{white}State} &
  \textbf{\color{white}Dir} &
  \textbf{\color{white}Act} &
  \textbf{\color{white}Function} &
  \textbf{\color{white}Effect / Nuance} &
  \textbf{\color{white}Dash34} &
  \textbf{\color{white}Train} \\
  \endhead
  %
  \multicolumn{7}{r}{\small\emph{Continued on next page}}\\
  \endfoot
  %
  \endlastfoot
}{%
  \end{longtable}
}

% --------------------------------------------------------------------------
% SIMPLE REFERENCE MACROS FOR BMS DOCS
% --------------------------------------------------------------------------
% CORRECTED: Changed from \newcommand to \providecommand to enable WIP file integration
% CORRECTED: Formatted with \S symbol per Briefing v0.2.0.1, Section 11.3

\providecommand{\dashref}[1]{Dash-34~\S~#1}
\providecommand{\dashone}[1]{Dash-1~\S~#1}
\providecommand{\trnref}[1]{TRN~#1}
\providecommand{\trnman}{BMS Training Manual 4.38.1}
\providecommand{\bmsver}{Falcon BMS~4.38.1}
\providecommand{\dashrefs}[1]{\textit{TO 1F-16CMAM-34-1-1}, Dash-34, sections \texttt{#1}}

% --------------------------------------------------------------------------
% VERSION CONTROL MACROS
% --------------------------------------------------------------------------

\newcommand{\docversion}{0.2.2.0}
\newcommand{\docbuild}{20260110}
\newcommand{\docstartdate}{05 January 2026}
\newcommand{\docenddate}{DD MMM 2026}
\newcommand{\chapterscompletedof}{2/7}
\newcommand{\tablesfilledpct}{0\%}
\newcommand{\fulldocversion}{\docversion+\docbuild}

% --------------------------------------------------------------------------
% GRAPHICS
% --------------------------------------------------------------------------

\usepackage{graphicx}
\graphicspath{{fig/}}

% --------------------------------------------------------------------------
% TITLE
% --------------------------------------------------------------------------

\title{TMS, DMS and CMS Usage Guide for \bmsver}
\author{Carlos ``Metal'' Nader}
\date{Version \fulldocversion{} | Progress: Chapters \chapterscompletedof{} | Tables \tablesfilledpct{} | January 2026}

% --------------------------------------------------------------------------
% DOCUMENT
% --------------------------------------------------------------------------

\begin{document}

\maketitle

\pagenumbering{roman}

\newpage

\tableofcontents

\newpage

\pagenumbering{arabic}

% --------------------------------------------------------------------------
% CHAPTER 1: INTRODUCTION
% --------------------------------------------------------------------------

\section{Introduction}

This document is a community-made reference guide for \bmsver, focused on practical use of three specific HOTAS controls: the Target Management Switch (TMS), the Display Management Switch (DMS), and the Countermeasures Management Switch (CMS). Although this guide was developed under Falcon BMS 4.38.1, the fundamental behavior of these switches has remained constant since at least Falcon BMS 4.36, making this guide applicable to almost any player of Falcon BMS.

Although other controls exist on the F-16 throttle and stick---such as the Communication Switch, the Dogfight/MRM Override, and the RDR Cursor Enable control---they are mentioned only when essential to understanding the behavior and context of TMS, DMS, and CMS.\footnote{Falcon BMS core avionics and weapons behaviour are documented in \dashone{2} and \dashref{2}. \trnman{} describes how these systems are trained in practice.}

Its goal is to reorganize information that is spread across the Dash-1, Dash-34 and the BMS Training Manual into mode-based tables and short explanations, so that virtual pilots can quickly understand what each switch press does in a given context.\footnote{See \dashref{2.1.5} (Hands-On Controls) and the foreword of \trnman{} for the role of TMS, DMS and CMS in BMS training.}

This is the \emph{TMS, DMS and CMS Usage Guide---Version \fulldocversion}, prepared between \docstartdate{} and \docenddate{}, and created with extensive assistance from an AI language model (Perplexity AI) to help structure, cross-reference and format the material. The human author remains fully responsible for every choice of content, interpretation and final wording, and any mistakes or omissions are attributable to the author alone, not to the AI system.

This work is entirely unofficial. The author is not affiliated with Benchmark Sims, MicroProse, any real-world air force, or any aircraft or weapons manufacturer. All interpretations, simplifications, errors and omissions in this document are solely the responsibility of the author and must not be attributed to the Falcon BMS development team or to any real-world organization.\footnote{Compare the official disclaimer and copyright statements in the foreword of \trnman{}.} Nothing in this document should ever be used for real-world operations, training, or procedures.

% --------------------------------------------------------------------------
% SECTION 1.1: DEVELOPMENT TIMELINE
% --------------------------------------------------------------------------

\subsection{Development timeline and status}

This guide was developed in structured phases, beginning \docstartdate{}. Current development status and targets are shown in Table~\ref{tab:dev_status} below.

\begin{table}[h]
\centering
\begin{tabular}{L{3.2cm} L{2.5cm} L{2.5cm}}
\toprule
\textbf{Metric} & \textbf{Current} & \textbf{Target} \\
\midrule
Start Date & \docstartdate & --- \\
Current Version & \fulldocversion & --- \\
Chapters Complete & \chapterscompletedof & 7/7 \\
Tables Filled & \tablesfilledpct & 100\% \\
\bottomrule
\end{tabular}
\caption{Development Status Snapshot}
\label{tab:dev_status}
\end{table}

The development roadmap is structured in three phases: (1) \emph{Chapter scaffolding} (Versions 0.1.0--0.7.0), during which all chapters receive narrative content and table structures; (2) \emph{Table population} (Versions 1.0.0--1.0.5), during which all tables are filled with complete HOTAS behavior descriptions and diagrams are generated; and (3) \emph{Review and release} (Versions 2.0.0-RC1 through 2.0.0-Stable), during which content is reviewed for accuracy, consistency, and clarity. Each phase produces a versioned PDF artifact, and all versions are archived for traceability.

% --------------------------------------------------------------------------
% SECTION 1.2: SCOPE AND PURPOSE
% --------------------------------------------------------------------------

\subsection{Scope and purpose}

This guide focuses exclusively on three HOTAS switches: the Target Management Switch (TMS), the Display Management Switch (DMS), and the Countermeasures Management Switch (CMS). Although other controls exist on the F-16 throttle and stick---such as the Communication Switch, the Dogfight/MRM Override, and the RDR Cursor Enable control---they are mentioned only when essential to understanding the behavior and context of TMS, DMS, and CMS.

This is not a comprehensive HOTAS or avionics manual. Instead, it is a usage guide organized by context, with emphasis on practical tables that show what each switch input does in specific flight modes, sensor configurations, and weapon employment scenarios. The guide bridges information scattered across official documentation and training missions, making it immediately accessible to pilots who ask: \emph{``In this radar mode, what does TMS Up do?''} or \emph{``How do I cycle through MFD formats with the DMS?''}

The guide assumes knowledge of basic F-16 operation and familiarity with master modes (NAV, A-A, A-G, DGFT). It does not replace the Dash-34 or Training Manual; rather, it complements them by organizing TMS/DMS/CMS behavior into searchable tables with cross-references back to official sources and practical training missions where each behavior can be practiced.

% --------------------------------------------------------------------------
% SECTION 1.3: VERSION, AUTHORSHIP AND AI ASSISTANCE
% --------------------------------------------------------------------------

\subsection{Version, authorship and AI assistance}

\textbf{Document Version:} \fulldocversion{} (Progress: Chapters \chapterscompletedof{} | Tables \tablesfilledpct)

\textbf{Falcon BMS Version:} 4.38.1 (Update 1)

\textbf{Authorship:} This guide was created by a member of the Falcon BMS community with structured assistance from AI language models (Perplexity AI). The human author identified scope, validated content against official Falcon BMS documentation, made all organizational and editorial decisions, and bears full responsibility for the guide's accuracy and presentation. AI tools were used for research organization, cross-referencing, and text generation---not for defining technical correctness.

\textbf{Disclaimer:} This is an unofficial, community-made document not affiliated with, endorsed by, or affiliated with Benchmark Sims, MicroProse, any military organization, or any aircraft or weapons manufacturer. All technical content is paraphrased in original words from official BMS documentation. No copyrighted material is reproduced directly. The guide is provided ``as-is'' for educational and simulation training purposes only.

\textbf{Copyright \& Sharing:} This guide may be freely copied, printed, translated, and shared within the Falcon BMS community for non-commercial use. Derivative works and contributions are encouraged, provided that proper credit is given to the original author and no derivative version claims official status.

% --------------------------------------------------------------------------
% SECTION 1.4: SOURCES AND REFERENCES
% --------------------------------------------------------------------------

\subsection{Sources and references}

This guide is based on the following primary Falcon BMS documents consulted during research and development:

\begin{enumerate}

\item \textbf{TO BMS 1F-16CMAM-34-1-1} (Dash-34, Change 4.38) -- Avionics and Nonnuclear Weapons Delivery Flight Manual

\begin{itemize}
\item Sections 2.1.5 (HOTAS Hands-On Controls)
\item Sections 2.3.1 (AN\-APG-68V5 Fire Control Radar)
\item Section 2.7 (Defensive Avionics -- CMS, ECM, CMDS)
\item Weapon-specific chapters (HARM, Maverick, IAMs, LGBs, Harpoon, SPICE, etc.)
\end{itemize}

\item \textbf{BMS Training Manual 4.38.1} (October 2025) -- Training missions and learning objectives

\begin{itemize}
\item Individual mission descriptions (TRN 11--28, with emphasis on weapons employment and avionics training)
\item Mission learning objectives and practical procedures
\end{itemize}

\item \textbf{TO BMS 1F-16CMAM-1} (Dash-1) -- F-16 Aircraft Systems, Normal and Abnormal Procedures

\begin{itemize}
\item Referenced for overall aircraft context and system interactions
\end{itemize}

\item \textbf{BMS User Manual 4.38} -- BMS user interface and setup

\begin{itemize}
\item Referenced for MFD, ICP, and UFC control information
\end{itemize}

\item \textbf{MCH 11-F16 Vol 5} (May 1996) -- F-16 Flight Manual Vol 5 (Surface Attack and Weapons Delivery)

\begin{itemize}
\item Referenced for operational context and ordnance procedures
\end{itemize}

\item \textbf{Falcon BMS Cockpit Arrangement Diagrams} (Multiple blocks)

\begin{itemize}
\item F-16C Block 50/52, Block 40/42, Block 30/32, F-16A MLU variants
\item Visual reference for HOTAS switch positions across variants
\end{itemize}

\end{enumerate}

\emph{Note:} This reference list will be updated throughout the development of the guide as new sources are consulted. Always refer to the most current version of this document to see the complete list of references.

% --------------------------------------------------------------------------
% SECTION 1.5: DOCUMENT STRUCTURE AND HOW TO READ IT
% --------------------------------------------------------------------------

\subsection{Document structure and how to read it}

\subsubsection{Part A: Foundational Sections}

Foundational sections establish core HOTAS concepts: Sensor of Interest (SOI), short vs. long press timing, master modes, and an overview of TMS/DMS/CMS roles. Read this first if you are new to the F-16 or HOTAS in general.

\subsubsection{Part B: Switch-Specific Sections}

Sections focused on TMS, DMS, and CMS each contain detailed tables using the \texttt{hotastable} environment.

\textbf{Table structure:}

Each table follows a seven-column format:

\begin{center}
\begin{tabular}{L{1.6cm} L{1.0cm} L{1.0cm} L{3.4cm} L{5.8cm} L{1.4cm} L{1.4cm}}
\toprule
\textbf{State} & \textbf{Dir} & \textbf{Act} & \textbf{Function} & \textbf{Effect / Nuance} & \textbf{Dash34} & \textbf{Train} \\
\midrule
Condition & Up & Shrt & Name & Explanation & Ref & TRN \\
\bottomrule
\end{tabular}
\end{center}

\textbf{How to find information:}

\begin{enumerate}
\item Identify the \textbf{master mode and sensor/weapon context} from the section title (e.g., ``TMS in Air-to-Air -- FCR CRM'').
\item Find the \textbf{State} within the table (e.g., ``Search'' vs. ``STT'').
\item Determine the \textbf{Direction} and \textbf{Action} (Short/Long).
\item Read the \textbf{Effect} and check the \textbf{Dash34} or \textbf{Training} reference.
\end{enumerate}

\subsubsection{Part C: Training and Visual Reference}

Training reference section links this guide to the 33 BMS training missions, offering a recommended progression and example tactical flows. Use this section to plan your training sequence.

Visual reference section provides schematic diagrams of the TMS, DMS, and CMS hats with arrows and short labels for each direction in common contexts. These are quick-reference visuals; always consult the tables for complete behavior descriptions.

\subsubsection{Part D: Appendices}

Appendices note any differences in TMS/DMS/CMS behavior across F-16 blocks and variants (e.g., Block 50/52 vs. Block 40/42), and provide a comprehensive index of all major tables and their locations.

% --------------------------------------------------------------------------
% SECTION 2: HOTAS FUNDAMENTALS (PLACEHOLDER)
% --------------------------------------------------------------------------

\section{HOTAS fundamentals}

\subsection{Sensor of Interest (SOI) and display logic}

[Content to be developed in next phase]

\subsection{Short vs long presses and timing}

[Content to be developed in next phase]

\subsection{Master modes and context-sensitive behaviour}

[Content to be developed in next phase]

\subsection{Overview of TMS, DMS and CMS}

[Content to be developed in next phase]

% --------------------------------------------------------------------------
% SECTION 3: TMS (PLACEHOLDER)
% --------------------------------------------------------------------------

\section{TMS -- Target Management Switch}

\subsection{Concept and general behaviour}

[Content to be developed in next phase]

\subsection{TMS and Situational Awareness displays}

[Content to be developed in next phase]

\subsection{TMS in Air-to-Air}

[Content to be developed in next phase]

\subsection{TMS in Air-to-Ground}

[Content to be developed in next phase]

\subsection{TMS in weapon employment}

[Content to be developed in next phase]

\subsection{TMS -- Block / variant notes}

[Content to be developed in next phase]

% --------------------------------------------------------------------------
% SECTION 4: DMS (PLACEHOLDER)
% --------------------------------------------------------------------------

\section{DMS -- Display Management Switch}

\subsection{Concept and Sensor of Interest (SOI)}

[Content to be developed in next phase]

\subsection{DMS in MFDS format selection and SWAP}

[Content to be developed in next phase]

\subsection{DMS in sensor and weapon context}

[Content to be developed in next phase]

\subsection{DMS -- Block and variant notes}

[Content to be developed in next phase]

% --------------------------------------------------------------------------
% SECTION 5: CMS (PLACEHOLDER)
% --------------------------------------------------------------------------

\section{CMS -- Countermeasures Management Switch}

\subsection{Concept and Interaction with CMDS / ECM / RWR}

\subsubsection{Concept}

The Countermeasures Management Switch (CMS) is a four-direction hat switch mounted on the control stick that serves as the pilot's primary control interface to the F-16's integrated electronic warfare (EW) defensive systems: the ALE-47 CMDS (automatic chaff/flare dispenser), ECM systems (external pods or internal avionics), and avionics-based threat defeat systems. The CMS supervises the aircraft's defensive response by controlling defensive program selection, managing ECM operational modes, and granting or withholding consent authority to all defensive subsystems.

Its role is to grant the pilot rapid tactical control over the aircraft's defensive posture. This control is operationally critical because defensive decisions frequently occur during high-G maneuvering when hand position cannot be redirected to distant cockpit panels. A pilot executing a 6-G defensive turn cannot simultaneously reach the CMDS MODE knob on the left console or the ECM control panel without abandoning aircraft control. By placing the CMS within thumb reach during full-stick maneuver, the design ensures that no tactical scenario regardless of G-load or workload forces the pilot to choose between aircraft control and defensive system authority. This design philosophy prioritizes pilot sovereignty: direct access to defensive control is never sacrificed for maneuvering demand.

RWR, although not directly linked to CMS, is more than a display device: it is the decision engine for both CMDS and ECM. The RWR continuously evaluates detected threat radars, classifies them (SEARCH, TRACK, LAUNCH), assigns threat priority, and communicates this information to both the ALE-47 CMDS (in AUTO or SEMI mode) and the ECM system (for band selection or jamming initiation).

For in-depth explanations about CMDS, ECM and RWR operation, see 
\dashrefs{2.7.1, 2.7.2, and 2.7.3}, respectively. This section focuses exclusively on CMS usage and control interface. As presented in the preceding chapters, condensed diagrams of ECM and other throttle and flight stick functionalities can be found on section \dashrefs{2.1.5}. Below is an image of the F-16 Flight Stick, with the CMD swtich location.

\begin{figure}[h]
\centering
\includegraphics[width=0.65\textwidth]{F-16_Side_Stick_Controller-1.jpg}
\caption{F-16 Throttle and Flight Stick. Image by Falconpedia (\url{falcon4.wikidot.com}), via Wikimedia Commons (\url{https://commons.wikimedia.org/wiki/File:F-16_Side_Stick_Controller.jpg}), licensed under the Creative Commons Attribution-Share Alike 3.0 Unported (CC BY-SA 3.0) license.}
\label{fig:f16_hotas_cms_location}
\end{figure}

\subsubsection{Interaction with CMDS / ECM}

Operationally, the CMS manages two distinct defensive layers: CMDS and ECM (in its both configurations: internal avionics or external pod). All CMS button pressings will be detailed in the next section.

Differences in F-16 Blocks or variants, especially regarding ECM, will be discussed in \hyperlink{section 5.3}{Section 5.3}.

\begin{enumerate}

\item \textbf{ECM (External Pod):} Controls the external ECM pod's 
operational state through pilot-directed transmission modes 
and consent authority.

\item \textbf{ECM (Integrated IDIAS):} Controls the integrated ECM 
system through automatic threat-reactive modes.

\item \textbf{CMDS in Manual Mode:} Allows the pilot to execute 
pre-selected dispenser programs on demand, independent of 
automatic systems.

\item \textbf{CMDS in Automatic/Semi-Automatic Modes:} Authorizes 
the ALE-47 CMDS to respond autonomously to RWR-detected threats 
when operating in AUTO or SEMI mode.

\end{enumerate}

\subsection{CMS Switch Actuation}

\subsubsection{CMS Actuation with CMDS}

[Content to be developed in next phase]

\subsubsection{CMS Actuation with ECM}

[Content to be developed in next phase]

\subsubsection{CMS Consent \& Constraints}

[Content to be developed in next phase]

\subsubsection{Important Operational Notes}

[Content to be developed in next phase]

\subsection{CMS -- Block and variant notes}

[Content to be developed in next phase]

% --------------------------------------------------------------------------
% SECTION 6: TRAINING REFERENCES (PLACEHOLDER)
% --------------------------------------------------------------------------

\section{Training references and practical flows}

\subsection{How to use this guide with BMS training missions}

[Content to be developed in next phase]

\subsection{Recommended progression}

[Content to be developed in next phase]

\subsection{Example flows for typical missions}

[Content to be developed in next phase]

% --------------------------------------------------------------------------
% SECTION 7: VISUAL REFERENCE (PLACEHOLDER)
% --------------------------------------------------------------------------

\section{HOTAS visual reference}

\subsection{F-16 HOTAS overview}

[Content to be developed in next phase]

\subsection{TMS diagrams}

[Content to be developed in next phase]

\subsection{DMS diagrams}

[Content to be developed in next phase]

\subsection{CMS diagrams}

[Content to be developed in next phase]

% --------------------------------------------------------------------------
% APPENDICES
% --------------------------------------------------------------------------

\appendix

\section{Block / variant overview}

\subsection{F-16CM Block 50/52}

[Content to be developed in next phase]

\subsection{F-16C/D Block 40/42}

[Content to be developed in next phase]

\subsection{F-16AM/BM MLU}

[Content to be developed in next phase]

\subsection{F-16I Sufa and Israeli variants}

[Content to be developed in next phase]

\subsection{Other export variants}

[Content to be developed in next phase]

\section{Tables index}

\subsection{TMS tables}

[To be populated]

\subsection{DMS tables}

[To be populated]

\subsection{CMS tables}

[To be populated]

\end{document}