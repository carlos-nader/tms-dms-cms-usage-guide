\documentclass{article}

\usepackage[T1]{fontenc}
\usepackage[utf8]{inputenc}
\usepackage{csquotes}
\usepackage{lmodern}
\usepackage{microtype}

\title{How to Resume the TMS/DMS/CMS Guide Project}
\author{}
\date{}

\begin{document}

\maketitle

\section{How to Resume This Project Later}

This section describes a simple, repeatable workflow for resuming work on the TMS/DMS/CMS usage guide after a pause. It assumes that the main LaTeX document, support markdown files, and this how-to file are stored together in a single project directory.

\subsection{Open the correct project context}
\label{sec:howto-context}

\begin{enumerate}
  \item Open Perplexity and navigate to your conversation history (Threads) in the Library.
  \item Locate the thread associated with this project by title, date, or a distinctive keyword (for example, \emph{TMS/DMS/CMS Guide}).
  \item Re-open that thread instead of starting a brand-new chat, so that the previous context and decisions remain available.
\end{enumerate}

If the original thread is not available for any reason, use the fallback procedure in Section~\ref{sec:howto-fallback-thread}.

\subsection{Check the current project state}
\label{sec:howto-state}

In your local project directory, identify and open the following key files:

\begin{itemize}
  \item The main LaTeX document (for example, \texttt{guide-v0.1.0+20260105.tex} or a later version).
  \item \texttt{PROJECT-TRACKING.md} (session log and roadmap).
  \item \texttt{version-system-v2.md} (versioning rules and naming convention).
\end{itemize}

Start by reading \texttt{PROJECT-TRACKING.md} to see the last completed session, the tasks that were finished, and the \emph{Next Steps} that were planned. Then review \texttt{version-system-v2.md} to recall the versioning scheme and current phase (scaffolding, table population, or review).

\subsection{Resume writing in a controlled way}
\label{sec:howto-writing}

To continue the guide itself:

\begin{enumerate}
  \item Open the main LaTeX file (for example, \texttt{guide-v0.1.0+20260105.tex}) in your LaTeX editor.
  \item Navigate to the chapter and section indicated as \enquote{Next Steps} in \texttt{PROJECT-TRACKING.md}.
  \item Optionally, open any supporting markdown files such as \texttt{SECTION-TEMPLATE.md} or specific section drafts (for example, a SOI draft for Section~2.1).
  \item Read a small portion of the previous section to regain context and tone.
  \item Begin writing the next section using the agreed structure (intro paragraph, background, table if applicable, explanation, and cross-references).
\end{enumerate}

\subsection{Working from a new thread (fallback)}
\label{sec:howto-fallback-thread}

If you need to resume the project in a new Perplexity thread:

\begin{enumerate}
  \item Start a new chat.
  \item Upload the key project files:
    \begin{itemize}
      \item The main LaTeX file (latest version).
      \item \texttt{PROJECT-TRACKING.md}.
      \item \texttt{version-system-v2.md}.
    \end{itemize}
  \item Clearly state that these files represent the current state of the TMS/DMS/CMS guide project and that you want to continue from the indicated chapter and section.
\end{enumerate}

This allows the assistant to reconstruct the project context from the sources themselves, even if the original thread is unavailable.

\subsection{End-of-session routine}
\label{sec:howto-eos}

At the end of each working session, follow this checklist:

\begin{enumerate}
  \item Update \texttt{PROJECT-TRACKING.md} with:
    \begin{itemize}
      \item Date and time of the session.
      \item Tasks completed.
      \item Any issues or open questions.
      \item Clear \emph{Next Steps} for the following session.
    \end{itemize}
  \item If you advanced the document meaningfully (for example, completing a chapter or major section), update the version macros in the LaTeX preamble (version number, build date, and progress indicators).
  \item Compile the main LaTeX file to generate an updated PDF snapshot for that version.
  \item If you use markdown helper files, run your conversion script (for example, \texttt{convert-to-word.bat}) to refresh any \texttt{.docx} views used in Word.
\end{enumerate}

By following this routine, you ensure that the project remains easy to resume even after long breaks, with a clear history, consistent versioning, and up-to-date artifacts.

\end{document}