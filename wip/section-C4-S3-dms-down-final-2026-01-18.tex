% ============================================================================
% FALCON BMS TMS/DMS/CMS HOTAS GUIDE
% WIP FILE TEMPLATE V1.0 — FINAL (Briefing v0.2.0.1 + TOC Fix)
% ============================================================================

% IMPORTANTE: Este é um TEMPLATE padronizado para arquivos WIP (Work-In-Progress)
% que serão integrados ao guide.tex.

% Nomenclatura: section-C4-S3-dms-down-review-2026-01-18.tex
% Padrão: section-C{N}-S{M}[-S{K}]-{titulo}-{status}-{data}.tex
% Exemplo: section-C5-S2-cms-actuation-hotas-tables-final-2026-01-10.tex

% Status: dev | review | final | approved | deprecated
% Locação: WIP/ (ativo) | ARCHIVE/ (aprovado/descartado)

\documentclass[11pt,a4paper]{article}

% --------------------------------------------------------------------------
% BASIC ENCODING AND LANGUAGE
% --------------------------------------------------------------------------

\usepackage[utf8]{inputenc}
\usepackage[T1]{fontenc}
\usepackage[english]{babel}

% --------------------------------------------------------------------------
% FONTS AND MICROTYPOGRAPHY
% --------------------------------------------------------------------------

\usepackage{lmodern}
\usepackage{microtype}

% --------------------------------------------------------------------------
% PAGE GEOMETRY AND LAYOUT
% --------------------------------------------------------------------------

\usepackage{geometry}
\geometry{a4paper, left=2.0cm, right=2.0cm, top=2.5cm, bottom=2.5cm}
\usepackage{setspace}
\onehalfspacing

% --------------------------------------------------------------------------
% COLORS AND LINKS
% --------------------------------------------------------------------------

\usepackage[table]{xcolor}
\definecolor{linkblue}{HTML}{004488}
\definecolor{linkred}{HTML}{882222}
\definecolor{headerblue}{HTML}{003366}
\definecolor{rowgray}{HTML}{F5F5F5}
\definecolor{subheadgray}{HTML}{E0E0E0}
\usepackage[pdfencoding=auto, psdextra, colorlinks=true, linkcolor=linkblue, citecolor=linkred, urlcolor=linkblue, breaklinks=true]{hyperref}
\usepackage{bookmark}

% --------------------------------------------------------------------------
% HEADERS AND FOOTERS
% --------------------------------------------------------------------------

\usepackage{fancyhdr}
\setlength{\headheight}{15pt}
\pagestyle{fancy}
\fancyhf{}
\fancyhead[L]{\leftmark}
\fancyhead[R]{\rightmark}
\fancyfoot[C]{\thepage}
\renewcommand{\headrulewidth}{0.4pt}
\renewcommand{\footrulewidth}{0pt}

% --------------------------------------------------------------------------
% TABLES AND MACROS
% CORRECTD: \end commands repositioned --------------------------------------------------------------------------

\usepackage{booktabs}
\usepackage{array}
\usepackage{longtable}
\usepackage{tabularx}

% Custom Columns
\newcolumntype{L}[1]{>{\raggedright\arraybackslash}p{#1}}
\newcolumntype{C}[1]{>{\centering\arraybackslash}p{#1}}
\newcolumntype{R}[1]{>{\raggedleft\arraybackslash}p{#1}}

% HOTAS table environment (Briefing v0.2.0.1, Section 6)
\newenvironment{hotastable}[1]{%
  \small
  \renewcommand{\arraystretch}{1.25}
  \begin{longtable}{L{1.6cm} L{1.0cm} L{1.0cm} L{3.4cm} L{5.8cm} L{1.4cm} L{1.4cm}}
  \caption{#1}\\
  \rowcolor{headerblue}
  \textbf{\color{white}State} &
  \textbf{\color{white}Dir} &
  \textbf{\color{white}Act} &
  \textbf{\color{white}Function} &
  \textbf{\color{white}Effect / Nuance} &
  \textbf{\color{white}Dash34} &
  \textbf{\color{white}Train} \\
  \endfirsthead
  %
  \rowcolor{headerblue}
  \textbf{\color{white}State} &
  \textbf{\color{white}Dir} &
  \textbf{\color{white}Act} &
  \textbf{\color{white}Function} &
  \textbf{\color{white}Effect / Nuance} &
  \textbf{\color{white}Dash34} &
  \textbf{\color{white}Train} \\
  \endhead
  %
  \multicolumn{7}{r}{\small\emph{Continued on next page}}\\
  \endfoot
  %
  \endlastfoot
}{%
  \end{longtable}
}

% --------------------------------------------------------------------------
% SIMPLE REFERENCE MACROS FOR BMS DOCS
% --------------------------------------------------------------------------

\providecommand{\dashref}[1]{Dash-34~\S~#1}
\providecommand{\dashone}[1]{Dash-1~\S~#1}
\providecommand{\trnref}[1]{TRN~#1}
\providecommand{\trnman}{BMS Training Manual 4.38.1}
\providecommand{\bmsver}{Falcon BMS~4.38.1}
\providecommand{\dashrefs}[1]{\textit{TO 1F-16CMAM-34-1-1}, Dash-34, sections \texttt{#1}}

% --------------------------------------------------------------------------
% VERSION CONTROL MACROS
% --------------------------------------------------------------------------

\newcommand{\docversion}{WIP Template v1.0}
\newcommand{\docbuild}{20260110}
\newcommand{\fulldocversion}{\docversion+\docbuild}

% --------------------------------------------------------------------------
% GRAPHICS
% --------------------------------------------------------------------------

\usepackage{graphicx}
\graphicspath{{fig/}}

% --------------------------------------------------------------------------
% TITLE
% --------------------------------------------------------------------------

\title{TMS, DMS and CMS Usage Guide for \bmsver}
\author{Carlos ``Metal'' Nader}
\date{Version \fulldocversion{} | January 2026}

% --------------------------------------------------------------------------
% DOCUMENT BEGIN
% --------------------------------------------------------------------------

\begin{document}

\maketitle

\newpage

\tableofcontents

\newpage


% ============================================================================
% WIP FILE METADATA (NOT RENDERED IN PDF)
% ============================================================================
% File Name: section-C4-S3-dms-down-review-2026-01-18.tex
% WIP Naming Convention: v1.4
% Target Chapter: C4 --- DMS Display Management Switch
% Target Section: S3 --- DMS Down
% Target Subsection: S3.x --- All subsections
% WIP Status: review
% Created: 2026-01-16 (design finalized)
% Last Modified: 2026-01-18 (full WIP generated)
% Integration Status: NOT YET INTEGRATED --- AWAITING HUMAN REVIEW & APPROVAL
% Integration Target: guide-v0.3.2.0 (post-DMS Up and "report" structure change, pre-DMS LeftRight)
% Narrative Completion: 100\% (core content complete, minor refinements pending)
% Table Fill Status: 100\% (all 3 core rows + exception states complete)
% Notes: need human review

% ============================================================================
% SECTION 4.3: DMS DOWN --- TOGGLE SOI BETWEEN DISPLAYS
% ============================================================================

\subsection{DMS Down: Toggle SOI Between Displays}
\label{sec:C4-S3}

% ============================================================================
% INTRO (Unnumbered)
% ============================================================================

DMS Down toggles the Sensor of Interest SOI among the displays available in the current master mode. As established in Section~\ref{sec:C4-S1-S1}, the set of valid SOI displays varies by mode: in NAV and A-G modes, the HUD is available; in air-to-air employment modes (A-A, DGFT, MSL OVRD), it is not

Consequently, DMS Down behaves in two distinct ways:

\begin{itemize}
  \item In \textbf{NAV} and \textbf{A-G} modes, DMS Down toggles SOI through all available displays: HUD $\rightarrow$ L/R MFD $\rightarrow$ l/R MFD $\rightarrow$ HUD.
  \item In \textbf{air-to-air employment modes} (A-A, DGFT, MSL OVRD), DMS Down toggles SOI only between the two MFD (L/R MFD $\leftrightarrow$ l/R MFD), since the HUD is not a valid SOI candidate
\end{itemize}

This design ensures that DMS Up and DMS Down work together to manage SOI across all available displays in each operational context. It is \textbf{important to note} that DMS Down transitions SOI between displays—HUD and the two MFD—without changing which format is currently displayed on any MFD. The pilot executes hands-on commands on whatever format is available at the selected display. Format transitions within an MFD are controlled by DMS Right and DMS Left, covered in Section~\ref{sec:C4-S4}.

% ============================================================================
% 4.3.1: DMS DOWN EFFECTIVENESS IN ALL MASTER MODES
% ============================================================================

\subsubsection{DMS Down Effectiveness in All Master Modes}
\label{sec:C4-S3-S1}

DMS Down effectiveness depends on which displays can serve as SOI in the current master 
mode, as established in Section~\ref{sec:C4-S1-S1}.

\paragraph{Master Modes Where HUD Is a Valid SOI Candidate:}

\subparagraph{NAV (Navigation) Master Mode:}

In NAV, DMS Down toggles SOI through all valid candidates: HUD and both MFD. 
Repeated DMS Down presses create a continuous 3-step sequence: HUD 
$\rightarrow$ L/R MFD $\rightarrow$ L/R MFD $\rightarrow$ HUD. This allows the pilot to 
quickly move hands-on command focus between the HUD and the two MFD sensor displays for 
navigation and sensor management.

\subparagraph{A-G in PRE (Preplanned Air-to-Ground) Mode:}

In A-G PRE, valid SOI candidates are the HUD and both MFD. DMS Down follows the 
same 3-step toggle pattern as NAV, moving SOI among the HUD and the two MFD. This allows the pilot to shift hands-on 
control focus while examining different sensor pages.

\subparagraph{A-G in VIS (Visual Air-to-Ground: CCIP, DTOS, AGM-65 VIS, IAM-VIS):}

In A-G VIS modes, valid SOI candidates are the HUD and both MFD. DMS Down toggles through 
the same 3-step pattern as in NAV. However, in A-G VIS, DMS Down becomes \textbf{operationally 
critical} rather than merely convenient.

A-G VIS delivery is fundamentally HUD-centric: the pilot acquires and designates the target 
visually using the HUD pipper (CCIP) or target designator box (AGM-65 VIS, IAM-VIS). These 
visual cues are controlled by CURSOR and TMS inputs, which are routed to whichever display 
is currently SOI. If SOI migrates to an MFD---such as TGP for sensor refinement, WPN for 
weapon status, or FCR/HSD for situational awareness---those same CURSOR and TMS commands 
will act on the MFD instead of the HUD, and visual designation on the HUD ceases to respond.

Therefore, DMS Down and DMS Up work in tandem in A-G VIS: DMS Down allows the pilot to 
temporarily move SOI to an MFD for sensor work or information review, while DMS Up 
immediately restores HUD SOI to resume visual designation. This up-down alternation is 
fundamental to efficient A-G VIS delivery and cannot be omitted without degrading command 
flow or situational awareness.

\paragraph{Master Modes Where HUD Is NOT a Valid SOI Candidate:}

In air-to-air employment (A-A, DGFT, and MSL OVRD) modes, the avionics architecture 
restricts SOI to the MFD only. The HUD cannot be designated as SOI 
in these modes (see Section~\ref{sec:C4-S1-S3} for the architectural rationale). 
Consequently, DMS Down is limited to toggling SOI between the two MFD (L/R MFD 
$\leftrightarrow$ L/R MFD). This 2-way toggle allows the pilot to select which MFD sensor 
display receives hands-on command priority.

In A-A contexts, this is operationally essential: the pilot uses DMS Down to shift SOI focus between onde MFD and the ohter so he can access and have direct control over whichever format is being actually displayed: FCR for track management 
and missile employment, HSD for tactical picture and threat assessment or between 
the FCR and TGP for situational awareness or supplemental tracking. Efficient air-to-air engagement depends critically on rapid SOI management via 
DMS Down.

% ============================================================================
% 4.3.2: DMS DOWN USAGE TABLE (HOTASTABLE)
% ============================================================================

\subsubsection{DMS Down Usage Table}
\label{sec:C4-S3-S2}

\begin{hotastable}{DMS Down Usage Across NAV, A-A, and A-G Master Modes}
  NAV & Down & Short & Toggle SOI cycle through displays & 
    DMS Down toggles SOI through HUD $\rightarrow$ L/R MFD $\rightarrow$ L/R MFD $\rightarrow$ HUD. With HUD as SOI, hands-on commands (CURSOR/ENABLE, TMS) manage HUD navigation symbology. Pressing DMS Down transfers SOI to the next display; the pilot can rotate through all three displays sequentially. &
    --- \\
  
  A-A & Down & Short & Toggle SOI between MFD only & 
    DMS Down toggles SOI only between the two MFD. The HUD cannot be SOI in A-A and remains a passive display. This is the primary HOTAS method for selecting which MFD sensor page receives hands-on command priority for track management, situational awareness, and weapons employment. & 
    2.1.1.2.3, 2.1.6.3 & 
    \trnref{18 BARCAP}, \trnref{17B IFF Intercept} \\
  
  A-G & Down & Short & Toggle SOI between HUD and MFD sensor pages & 
    DMS Down toggles SOI through HUD $\rightarrow$ L/R MFD $\rightarrow$ L/R MFD $\rightarrow$ HUD. In A-G PRE, DMS Down is a convenience tool for shifting hands-on focus between HUD and MFD sensor pages. \textbf{In A-G VIS (CCIP, DTOS, AGM-65 VIS, IAM-VIS), DMS Down is operationally critical:} the HUD is the primary visual designation interface. DMS Down allows the pilot to alternate between HUD visual cueing (TMS/CURSOR steering, pipper control, TD box positioning) and MFD sensor work (TGP search/refine, WPN status, FCR A-G ranging). Loss of HUD SOI in A-G VIS prevents proper visual designation and must be recovered with DMS Up. & 
    2.1.1.2.3, 2.1.6.3 & 
    \trnref{10 GP Bombs}, \trnref{11 LGB}, \trnref{13 Maverick}, \trnref{14 Maverick Adv}, \trnref{15 IAM} \\
\end{hotastable}

% ============================================================================
% 4.3.3: DMS DOWN EXCEPTION STATES
% ============================================================================

\subsubsection{DMS Down Exception States}
\label{sec:C4-S3-S3}

In certain special states and submodes, DMS Down may be temporarily ineffective, even in master modes where SOI toggling normally works, as a reflection of DMS Up, as seen on Section~\ref{sec:C4-S2-S3}

\begin{itemize}
  \item \textbf{Snowplow (SP) PRE State (Unstabilised)}
  
    When the pilot enters Snowplow mode (a specialised A-G ground-stabilisation mode for slewing to arbitrary ground positions) and the SP position has not yet been stabilised with TMS Up, the SOI is effectively ``nowhere.'' Both the A-G radar and TGP MFD displays show \texttt{NOT SOI}, and neither display is designated as SOI. As a result, \textbf{DMS Down has no effect} in this state: the toggle mechanism has nowhere to advance SOI to.
    
    Once the SP position is stabilised with TMS Up (pressing TMS Up on the HUD), SOI returns to its previous designated display, and DMS Down resumes normal toggling behaviour.

  \item \textbf{MARK/OFLY Submode}
  
    In the MARK/OFLY submode (a specialised target-acquisition context documented in \dashref{2.1.1.2.3}), the SOI cannot be designated or changed at all. Consequently, \textbf{DMS Down has no effect} in MARK/OFLY: you cannot toggle SOI when SOI designation itself is locked. This submode is rare in normal operations but is important to recognise if you encounter it during unusual procedures or system states.
\end{itemize}

\end{document}