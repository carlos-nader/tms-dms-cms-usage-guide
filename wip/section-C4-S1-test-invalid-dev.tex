% ============================================================================

% FALCON BMS TMS/DMS/CMS HOTAS GUIDE

% WIP FILE TEMPLATE V1.0 — Section 4.1.3 (Concept and SOI / Clarification)

% ============================================================================

% IMPORTANTE: Este é um arquivo WIP (Work-In-Progress) a ser integrado em guide.tex.

% Nomenclatura: Siga wip-naming-v1.4

% Padrão: section-C{N}-S{M}-S{K}-{titulo}-{status}-{data}.tex

% Exemplo: section-C4-S1-S3-hud-soi-hmcs-capabilities-dev-2026-01-14.tex

% Status: dev | review | final | approved | deprecated

% Locação: WIP/ (ativo) | ARCHIVE/ (aprovado/descartado)

\documentclass[11pt,a4paper]{article}

% --------------------------------------------------------------------------

% BASIC ENCODING AND LANGUAGE

% --------------------------------------------------------------------------

\usepackage[utf8]{inputenc}

\usepackage[T1]{fontenc}

\usepackage[english]{babel}

% --------------------------------------------------------------------------

% FONTS AND MICROTYPOGRAPHY

% --------------------------------------------------------------------------

\usepackage{lmodern}

\usepackage{microtype}

% --------------------------------------------------------------------------

% PAGE GEOMETRY AND LAYOUT

% --------------------------------------------------------------------------

\usepackage{geometry}

\geometry{a4paper, left=2.0cm, right=2.0cm, top=2.5cm, bottom=2.5cm}

\usepackage{setspace}

\onehalfspacing

% --------------------------------------------------------------------------

% COLORS AND LINKS

% --------------------------------------------------------------------------

\usepackage[table]{xcolor}

\definecolor{linkblue}{HTML}{004488}

\definecolor{linkred}{HTML}{882222}

\definecolor{headerblue}{HTML}{003366}

\definecolor{rowgray}{HTML}{F5F5F5}

\definecolor{subheadgray}{HTML}{E0E0E0}

\usepackage[pdfencoding=auto, psdextra, colorlinks=true, linkcolor=linkblue, citecolor=linkred, urlcolor=linkblue, breaklinks=true]{hyperref}

\usepackage{bookmark}

% --------------------------------------------------------------------------

% HEADERS AND FOOTERS

% --------------------------------------------------------------------------

\usepackage{fancyhdr}

\setlength{\headheight}{15pt}

\pagestyle{fancy}

\fancyhf{}

\fancyhead[L]{\leftmark}

\fancyhead[R]{\rightmark}

\fancyfoot[C]{\thepage}

\renewcommand{\headrulewidth}{0.4pt}

\renewcommand{\footrulewidth}{0pt}

% --------------------------------------------------------------------------

% TABLES AND MACROS

% --------------------------------------------------------------------------

\usepackage{booktabs}

\usepackage{array}

\usepackage{longtable}

\usepackage{tabularx}

% Custom Columns

\newcolumntype{L}[1]{>{\\raggedright\\arraybackslash}p{#1}}

\newcolumntype{C}[1]{>{\\centering\\arraybackslash}p{#1}}

\newcolumntype{R}[1]{>{\\raggedleft\\arraybackslash}p{#1}}

% HOTAS table environment (Briefing v0.2.0.1, Section 6)

\newenvironment{hotastable}[1]{%

\small

\renewcommand{\arraystretch}{1.25}

\begin{longtable}{L{1.6cm} L{1.0cm} L{1.0cm} L{3.4cm} L{5.8cm} L{1.4cm} L{1.4cm}}

\caption{#1}\\

\rowcolor{headerblue}

\textbf{\color{white}State} &

\textbf{\color{white}Dir} &

\textbf{\color{white}Act} &

\textbf{\color{white}Function} &

\textbf{\color{white}Effect / Nuance} &

\textbf{\color{white}Dash34} &

\textbf{\color{white}Train} \\

\endfirsthead

\rowcolor{headerblue}

\textbf{\color{white}State} &

\textbf{\color{white}Dir} &

\textbf{\color{white}Act} &

\textbf{\color{white}Function} &

\textbf{\color{white}Effect / Nuance} &

\textbf{\color{white}Dash34} &

\textbf{\color{white}Train} \\

\endhead

\multicolumn{7}{r}{\small\emph{Continued on next page}}\\

\endfoot

\endlastfoot

}{%

\end{longtable}

% --------------------------------------------------------------------------

% SIMPLE REFERENCE MACROS FOR BMS DOCS

% --------------------------------------------------------------------------

\providecommand{\dashref}[1]{Dash-34~\S~#1}

\providecommand{\dashone}[1]{Dash-1~\S~#1}

\providecommand{\trnref}[1]{TRN~#1}

\providecommand{\trnman}{BMS Training Manual 4.38.1}

\providecommand{\bmsver}{Falcon BMS~4.38.1}

\providecommand{\dashrefs}[1]{\textit{TO 1F-16CMAM-34-1-1}, Dash-34, sections \texttt{#1}}

% --------------------------------------------------------------------------

% VERSION CONTROL MACROS

% --------------------------------------------------------------------------

\newcommand{\docversion}{C4-S1-S3}

\newcommand{\docbuild}{20260114}

\newcommand{\fulldocversion}{\docversion+\docbuild}

% --------------------------------------------------------------------------

% GRAPHICS

% --------------------------------------------------------------------------

\usepackage{graphicx}

\graphicspath{{fig/}}

% --------------------------------------------------------------------------

% TITLE

% --------------------------------------------------------------------------

\title{TMS, DMS and CMS Usage Guide for \bmsver}

\author{Carlos ``Metal'' Nader}

\date{Version \fulldocversion{} | January 2026}

% --------------------------------------------------------------------------

% DOCUMENT BEGIN

% --------------------------------------------------------------------------

\begin{document}

\maketitle

\newpage

\tableofcontents

\newpage

% ============================================================================

% WIP FILE METADATA (NOT RENDERED IN PDF)

% ============================================================================

% File Name: section-C4-S1-S3-hud-soi-hmcs-capabilities-dev-2026-01-14.tex

% WIP Naming Convention: v1.4

% Target Chapter: C4 (DMS --- Display Management Switch)

% Target Section: S1 (Concept and Sensor of Interest)

% Target Subsection: S3 (HUD as SOI in A-A and HMCS Capabilities) — NEW CLARIFICATION SECTION

%

% WIP Status: dev

% Created: 2026-01-14

% Last Modified: 2026-01-14

% Integration Status: NOT YET INTEGRATED | INTEGRATION TARGET: v0.3.0.0

%

% Narrative Completion: 100% (concise clarification section)

% Table Fill Status: 0% (no tables in this section)

%

% Notes:

% - Clarification section added per request to Section 4.1, resolves ambiguity between SOI status and display functional capability

% - Addresses reader confusion: "If HUD cannot be SOI in A-A, can HMCS still acquire targets?"

% - Answer: YES. SOI routing mechanism ≠ Display functional capability. SOI controls HOTAS input delivery; functional capability operates independently.

% - HMCS exemplifies this distinction with off-boresight targeting, AIM-9 BORE, AATLL, FCR ACM BORE slaving

% - Provides theoretical foundation for understanding why HUD/HMCS have limitations in SOI designation but retain full targeting capability

% - All statements validated against Dash-34 sections 2.1.1.2.3, 2.5 (HMCS operations), with narrative references to Dash-34 Section 2.5

% - Drafted per requirement: ~220 words, conceptual level (not technical details), one Dash-34 reference at end

% - Cross-reference to 4.2.1 (DMS Up) planned for later; deferred pending 4.2.1 revision

% ============================================================================

% ============================================================================

% APPROVED CONTENT --- HUD as SOI in A-A and HMCS Capabilities (Section 4.1.3)

% ============================================================================

\subsubsection{HUD as SOI in A-A and HMCS Capabilities}

\label{sec:C4-S1-S3}

The restriction that HUD cannot be designated as SOI in A-A master mode applies exclusively to the \textbf{SOI routing architecture} — the mechanism by which HOTAS inputs (cursor slew, TMS commands) are delivered to a specific display. This architectural constraint \textbf{does not eliminate the functional capability} of the HUD or related displays to acquire, track, or cue targets in A-A operations.

The \textbf{Helmet Mounted Cueing System (HMCS)} exemplifies this distinction. Although HMCS is an extension of the HUD display system and shares the HUD's architectural restriction in A-A mode (neither can be designated as SOI via DMS), HMCS retains \textbf{independent off-boresight targeting capability}. For example, the pilot can slave an AIM-9 seeker to the HMCS visor line-of-sight without regard to which display (FCR, HSD, or TGP) is currently the SOI, and can employ HMCS-derived target acquisition cues that function independently of SOI status.

This principle reflects the fundamental architectural design: the SOI mechanism manages \textbf{HOTAS input routing}, while \textbf{display functional capabilities operate orthogonally}. Displays designated as SOI in A-A (FCR, HSD, TGP) receive these HOTAS inputs; displays not designated as SOI (HUD, HMCS) provide cueing and targeting functions through independent mechanisms (e.g., helmet line-of-sight, derived sensor information). Both types of functionality are essential to A-A operations, despite the SOI designation limitation.

For further technical details on HMCS capabilities and behavior in A-A contexts, refer to \dashref{2.5} (Helmet Mounted Cueing System).

% ============================================================================

% END OF SECTION 4.1.3

% ============================================================================

\end{document}