%============================================================================
% LICENSING NOTICE
%============================================================================
% This WIP file, upon integration into guide-v.tex, becomes part of the
% TMS/DMS/CMS Usage Guide and is released under CC BY-NC 4.0.
% For details, see LICENSE in the repository root.
%============================================================================

%============================================================================
% FALCON BMS TMS/DMS/CMS HOTAS GUIDE
% WIP FILE TEMPLATE V1.0 — FINAL (Briefing v0.2.0.1 + TOC Fix)
%============================================================================

% IMPORTANTE: Este é um TEMPLATE padronizado para arquivos WIP (Work-In-Progress)
% que serão integrados ao guide.tex.

% Nomenclatura: Siga wip-naming-v1.3
% Padrão: section-C{N}-S{M}[-S{K}]-{titulo}-{status}-{data}.tex
% Exemplo: section-C5-S2-cms-actuation-hotas-tables-final-2026-01-10.tex

% Status: dev | review | final | approved | deprecated
% Locação: WIP/ (ativo) | ARCHIVE/ (aprovado/descartado)

%============================================================================
% PREAMBLE COMPLETO — TMS/DMS/CMS Usage Guide for Falcon BMS 4.38.1
% Gerado: 17 January 2026
% Status: Novo preâmbulo (report + twoside + titlesec + fancyhdr melhorado)
%============================================================================

\documentclass[11pt, a4paper, twoside]{report}

%--------------------------------------------------------------------------
% BASIC ENCODING AND LANGUAGE
%--------------------------------------------------------------------------

\usepackage[utf8]{inputenc}
\usepackage[T1]{fontenc}
\usepackage[english]{babel}

%--------------------------------------------------------------------------
% FONTS AND MICROTYPOGRAPHY
%--------------------------------------------------------------------------

\usepackage{lmodern}
\usepackage{microtype}

%--------------------------------------------------------------------------
% PAGE GEOMETRY AND LAYOUT
%--------------------------------------------------------------------------

\usepackage{geometry}
\geometry{a4paper, left=2.0cm, right=2.0cm, top=2.5cm, bottom=2.5cm}
\usepackage{setspace}
\onehalfspacing

% --------------------------------------------------------------------------
% COLORS AND LINKS
% --------------------------------------------------------------------------

\usepackage[table]{xcolor}
\definecolor{linkblue}{HTML}{004488}
\definecolor{linkred}{HTML}{882222}
\definecolor{headerblue}{HTML}{003366}
\definecolor{rowgray}{HTML}{F5F5F5}
\definecolor{subheadgray}{HTML}{E0E0E0}

\usepackage{soul}
\usepackage[pdfencoding=auto, psdextra, colorlinks=true, linkcolor=linkblue, citecolor=linkred, urlcolor=linkblue, breaklinks=true]{hyperref}
\usepackage{bookmark}

% ============================================================================
% CAPTION SETUP BY TYPE
% ============================================================================

\usepackage{caption}

% GLOBAL PATTERN
\captionsetup{font=small, labelfont=bf, justification=centering, singlelinecheck=true}
% FIGURES
\captionsetup[figure]{font=footnotesize, labelfont=bf, justification=centering, singlelinecheck=true}
% % TABLE/LONGTABLE/hotastable:
\captionsetup[table]{font=small, labelfont=bf, justification=centering, singlelinecheck=true}

%--------------------------------------------------------------------------
% HEADERS AND FOOTERS (IMPROVED for report + twoside)
%--------------------------------------------------------------------------

\usepackage{fancyhdr}
\setlength{\headheight}{25pt}                    % Increased (was 15pt) for long names
\pagestyle{fancy}
\fancyhf{}                                        % Clear all
\fancyhead[LO,RE]{\small\textit{\leftmark}}     % Outer edge (odd left, even right): chapter name
\fancyhead[RO,LE]{\small\thepage}               % Inner edge (odd right, even left): page number
\fancyfoot{}                                      % No footer (page number in header)
\renewcommand{\headrulewidth}{0.4pt}
\renewcommand{\footrulewidth}{0pt}

%-------------------------------------------------------------------------
% CHAPTER FORMATTING AND SPACING (via titlesec)
%--------------------------------------------------------------------------

\usepackage{titlesec}

% --------------------------------------------------------------------------
% CHAPTER - Nível 0 (Destaque Máximo)
% --------------------------------------------------------------------------
% Shape: display → standalone em nova linha
% Font: \Large\bfseries --- label e título ambos em bold para destaque máximo
% Numeração: SIM
% Espaçamento: 15pt antes, 28pt depois
% --------------------------------------------------------------------------

\titleformat{\chapter}[display]
{\normalfont\Large\bfseries}           % Formato: LARGE + bold (label)
{\chaptertitlename~\thechapter}        % Rótulo: "Chapter 1" (bold também)
{20pt}                                 % Espaço vertical antes do corpo
{\Large\bfseries}                      % Corpo do título: LARGE + bold

\titlespacing{\chapter}
{0pt}       % Margem esquerda (sem indent)
{15pt}      % Espaço ANTES do título do capítulo
{28pt}      % Espaço DEPOIS do título do capítulo (AUMENTADO)
[0pt]       % Margem direita

% --------------------------------------------------------------------------
% SECTION - Nível 1
% --------------------------------------------------------------------------
% Shape: hang → label à esquerda, corpo indentado (compacto)
% Font: \large\bfseries (mesmo destaque que chapter para coerência)
% Numeração: SIM
% Espaçamento: 15pt antes, 8pt depois (REDUZIDO de 15pt/10pt anteriores)
% --------------------------------------------------------------------------

\titleformat{\section}[hang]
{\normalfont\large\bfseries}           % Formato: large + bold
{\thesection}                          % Rótulo: "1", "2", etc
{1em}                                  % Espaço entre rótulo e corpo
{}                                     % Corpo do título (herda do format)

\titlespacing{\section}
{0pt}       % Margem esquerda (sem indent)
{15pt}      % Espaço ANTES do título da seção (REDUZIDO)
{8pt}       % Espaço DEPOIS do título da seção (REDUZIDO)
[0pt]       % Margem direita

% --------------------------------------------------------------------------
% SUBSECTION - Nível 2
% --------------------------------------------------------------------------
% Shape: hang → label esquerda, corpo indentado
% Font: \normalsize\bfseries (redução visual, ainda robusto)
% Numeração: SIM
% Espaçamento: 12pt antes, 6pt depois (compacto)
% --------------------------------------------------------------------------

\titleformat{\subsection}[hang]
{\normalfont\normalsize\bfseries}      % Formato: tamanho normal + bold
{\thesubsection}                       % Rótulo: "1.1", "1.2", etc
{1em}                                  % Espaço entre rótulo e corpo
{}                                     % Corpo do título

\titlespacing{\subsection}
{0pt}       % Margem esquerda (sem indent)
{12pt}      % Espaço ANTES do título da subseção
{6pt}       % Espaço DEPOIS do título da subseção
[0pt]       % Margem direita

% --------------------------------------------------------------------------
% SUBSUBSECTION - Nível 3
% --------------------------------------------------------------------------
% Shape: hang → label esquerda, máximo recuo (bem compacto)
% Font: \normalsize\bfseries (mesmo tamanho que subsection, padrão)
% Numeração: SIM (configurable com secnumdepth)
% Espaçamento: 10pt antes, 4pt depois (bem comprimido)
% --------------------------------------------------------------------------

\titleformat{\subsubsection}[hang]
{\normalfont\normalsize\bfseries}      % Formato: tamanho normal + bold
{\thesubsubsection}                    % Rótulo: "1.1.1", "1.1.2", etc
{1em}                                  % Espaço entre rótulo e corpo
{}                                     % Corpo do título

\titlespacing{\subsubsection}
{0pt}       % Margem esquerda (sem indent)
{10pt}       % Espaço ANTES do título
{4pt}       % Espaço DEPOIS do título (bem comprimido)
[0pt]       % Margem direita

% --------------------------------------------------------------------------
% PARAGRAPH - Nível 4 (CRÍTICO: Ilusão Óptica Resolvida)
% --------------------------------------------------------------------------
% Shape: runin → inline, integrado ao texto (seu requisito)
% Font: \small\bfseries (SOLUÇÃO PARA ILUSÃO ÓPTICA DO NEGRITO)%       
% Numeração: NÃO (padrão para \paragraph)
% Espaçamento: runin (sem espaço vertical antes, integrado ao texto)
% --------------------------------------------------------------------------

\titleformat{\paragraph}[runin]
{\normalfont\small\bfseries}           % Formato: SMALL + bold (compensa ilusão)
{}                                     % Rótulo: vazio (sem numeração)
{0em}                                  % Espaço entre rótulo e corpo (zero, runin)
{}                                     % Corpo do título

\titlespacing{\paragraph}
{0pt}       % Margem esquerda (sem indent)
{8pt}       % Espaço ANTES do título do parágrafo (ADICIONADO - cria separação)
{1em}       % Espaço DEPOIS do título (ADICIONADO - espaço após ":")
[0pt]       % Margem direita

% --------------------------------------------------------------------------
% SUBPARAGRAPH - Nível 5 (Preparação Futura)
% --------------------------------------------------------------------------
% Shape: runin → inline, integrado ao texto
% Font: \small (MENOS destaque que \paragraph, sem bold)
% Numeração: NÃO (padrão para \subparagraph)
% Espaçamento: runin (integrado)
% Uso: Se usado, aparecerá com destaque menor que \paragraph
% --------------------------------------------------------------------------

\titleformat{\subparagraph}[runin]
{\normalfont\small}                    % Formato: small, SEM bold (menos destaque)
{}                                     % Rótulo: vazio (sem numeração)
{0em}                                  % Espaço entre rótulo e corpo (zero, runin)
{}                                     % Corpo do título

\titlespacing{\subparagraph}
{0pt}       % Margem esquerda (sem indent)
{8pt}       % Espaço ANTES do título (ADICIONADO)
{1em}       % Espaço DEPOIS do título (ADICIONADO)
[0pt]       % Margem direita

%--------------------------------------------------------------------------
% TABLES AND MACROS
%--------------------------------------------------------------------------

\usepackage{booktabs}
\usepackage{array}
\usepackage{longtable}
\usepackage{tabularx}

% Custom Columns
\newcolumntype{L}[1]{>{\raggedright\arraybackslash}p{#1}}
\newcolumntype{C}[1]{>{\centering\arraybackslash}p{#1}}
\newcolumntype{R}[1]{>{\raggedleft\arraybackslash}p{#1}}

% Macro for Visual Reference Links
\newcommand{\imglink}[1]{\hspace{2pt}\hyperref[#1]{\scriptsize\textbf{[Fig]}}}

%============================================================================
% HOTAS table environment (per Briefing v0.2.0.1)
%============================================================================

\newenvironment{hotastable}[1]{%
  \small
  \setlength{\tabcolsep}{2pt}
  \renewcommand{\arraystretch}{1.25}
  \begin{longtable}{L{1.00cm} L{0.90cm} L{0.90cm} L{3.30cm} L{6.40cm} L{1.40cm} L{2.10cm}}
  	\caption{#1}\\
  	\rowcolor{headerblue}
  	\multicolumn{1}{>{\centering\arraybackslash}p{1.00cm}}{\textbf{\color{white}Mode}} &
  	\multicolumn{1}{>{\centering\arraybackslash}p{0.90cm}}{\textbf{\color{white}Dir.}} &
  	\multicolumn{1}{>{\centering\arraybackslash}p{0.90cm}}{\textbf{\color{white}Act.}} &
  	\multicolumn{1}{>{\centering\arraybackslash}p{3.30cm}}{\textbf{\color{white}Function}} &
  	\multicolumn{1}{>{\centering\arraybackslash}p{6.40cm}}{\textbf{\color{white}Effect / Nuance}} &
  	\multicolumn{1}{>{\centering\arraybackslash}p{1.40cm}}{\textbf{\color{white}Dash34}} &
  	\multicolumn{1}{>{\centering\arraybackslash}p{2.10cm}}{\textbf{\color{white}Train.}} \\
  	\endfirsthead
  	\rowcolor{headerblue}
  	\multicolumn{1}{>{\centering\arraybackslash}p{1.00cm}}{\textbf{\color{white}Mode}} &
  	\multicolumn{1}{>{\centering\arraybackslash}p{0.90cm}}{\textbf{\color{white}Dir.}} &
  	\multicolumn{1}{>{\centering\arraybackslash}p{0.90cm}}{\textbf{\color{white}Act.}} &
  	\multicolumn{1}{>{\centering\arraybackslash}p{3.30cm}}{\textbf{\color{white}Function}} &
  	\multicolumn{1}{>{\centering\arraybackslash}p{6.40cm}}{\textbf{\color{white}Effect / Nuance}} &
  	\multicolumn{1}{>{\centering\arraybackslash}p{1.40cm}}{\textbf{\color{white}Dash34}} &
  	\multicolumn{1}{>{\centering\arraybackslash}p{2.10cm}}{\textbf{\color{white}Train.}} \\
  	\endhead
  	\multicolumn{7}{r}{\small\emph{Continued on next page}}\\
  	\endfoot
  	\endlastfoot
  }{%
  \end{longtable}
  }

%--------------------------------------------------------------------------
% SIMPLE REFERENCE MACROS FOR BMS DOCS
%--------------------------------------------------------------------------

\providecommand{\dashref}[1]{Dash-34~\S~#1}
\providecommand{\dashone}[1]{Dash-1~\S~#1}
\providecommand{\trnref}[1]{TRN~#1}
\providecommand{\trnman}{BMS Training Manual 4.38.1}
\providecommand{\bmsver}{Falcon BMS~4.38.1}
\providecommand{\dashrefs}[1]{\textit{TO 1F-16CMAM-34-1-1}, Dash-34, sections \texttt{#1}}

%--------------------------------------------------------------------------
% VERSION CONTROL MACROS
%--------------------------------------------------------------------------

\newcommand{\docversion}{0.3.2.0}
\newcommand{\docbuild}{20260120}
\newcommand{\docstartdate}{05 January 2026}
\newcommand{\docenddate}{20012026}
\newcommand{\chapterscompletedof}{2/7}
\newcommand{\tablesfilledpct}{0}
\newcommand{\fulldocversion}{\docversion+\docbuild}

%--------------------------------------------------------------------------
% GRAPHICS
%--------------------------------------------------------------------------

\usepackage{graphicx}
\graphicspath{{fig/}}
\usepackage{float}

%--------------------------------------------------------------------------
% TITLE
%--------------------------------------------------------------------------

\title{TMS, DMS and CMS Usage Guide for \bmsver}
\author{Carlos ``Metal'' Nader}
\date{Version \fulldocversion{} | Progress: Chapters \chapterscompletedof{} | Tables \tablesfilledpct{} | January 2026}

%============================================================================
% DOCUMENT BEGIN
%============================================================================

\begin{document}

\maketitle

\pagenumbering{roman}

%============================================================================
% TOC DEPTH CONFIGURATION (CRITICAL for report)
%============================================================================
\setcounter{tocdepth}{3}       % Show up to \subsubsection in TOC
\setcounter{secnumdepth}{3}    % Number up to \subsubsection

\newpage

\tableofcontents

\newpage

\pagenumbering{arabic}

%============================================================================
% WIP FILE METADATA (NOT RENDERED IN PDF)
%============================================================================

% File Name: chapter-C4-dms-review-2026-01-30.tex
% WIP Naming Convention: v1.4
% Target Chapter: C{N} (Chapter Title)
% Target Section: S{M} (Section Title)
% [Target Subsection: S{K} (Subsection Title) — OPTIONAL]
%
% WIP Status: review --- dev | review | final | approved | deprecated
% Created: 2026-01-30
% Last Modified: 2026-01-30
% Integration Status: NOT YET INTEGRATED | INTEGRATION TARGET: v0.3.3.0
%
% Narrative Completion: 0-100%
% Table Fill Status: 0-100%
%
% Notes: revision of already integrated C4:S1-S3 + new C4:S4
% - [Describe what's done, what's pending, open questions]
% - [Cross-references to other WIP files or guide.tex sections]
% - [Any divergences from briefing or validation issues]

%============================================================================
%============================================================================
% TEMPLATE STRUCTURE — Replace section/subsection headers with your content
%============================================================================
% CONTENT BEGINS HERE

%============================================================================
%--------------------------------------------------------------------------
% CHAPTER 4: DMS
%--------------------------------------------------------------------------
\chapter{DMS -- Display Management Switch}
\label{chap:C4}

%--------------------------------------------------------------------------
% CHAPTER 4: SECTION 4.1
%--------------------------------------------------------------------------
\section{Concept and Sensor of Interest (SOI)}
\label{sec:C4-S1}

The \textbf{Display Management Switch (DMS)} is a four-direction spring-loaded hat located on the flight stick. Its primary role is to manage which display or sensor receives hands-on control inputs, known as the Sensor of Interest (SOI), and to cycle through the Multifunction Display (MFD) formats.

Unlike the Target Management Switch (TMS), which performs tactical functions such as target designation and data management (see Chapter~\ref{chap:C3}), the DMS is a transversal SOI manager. It does not designate targets or change radar modes directly; instead, it selects \textit{which display or sensor} the pilot is currently controlling with other HOTAS inputs (such as CURSOR/ENABLE or TMS) --- Sections \ref{sec:C4-S2} and \ref{sec:C4-S3} --- and cycles \textit{MFD format pages} --- Section \ref{sec:C4-S4}.

As presented in Chapter~\ref{chap:C3}, condensed diagrams for throttle and flight stick switches functionalities can be found on \dashref{2.1.5}. Below is an image of the F-16 Flight Stick, with the DMS switch location.

\begin{figure}[H]
	\centering
	\includegraphics[width=0.65\textwidth]{F-16_Side_Stick_Controller-1.jpg}
	\caption{F-16 Throttle and Flight Stick. Image by Falconpedia (\url{falcon4.wikidot.com}), via Wikimedia Commons (\url{https://commons.wikimedia.org/wiki/File:F-16_Side_Stick_Controller.jpg}), licensed under the Creative Commons Attribution-Share Alike 3.0 Unported (CC BY-SA 3.0) license.}
	\label{fig:f16_hotas_dms_location}
\end{figure}

\paragraph*{DMS Across F-16 Blocks and Variants:}

The functionality of the DMS --- SOI selection, MFD format cycling and all associated behaviors --- is identical across all F-16 blocks and variants available in Falcon BMS. Differences in aircraft avionics do not alter DMS switch usage. For this reason, all DMS procedures in this chapter apply universally to the entire F-16 family.

%--------------------------------------------------------------------------
% CHAPTER 4: SECTION 4.1.1
%--------------------------------------------------------------------------
\subsection{SOI Definition}
\label{sec:C4-S1-S1}

The Sensor of Interest (SOI) is the display or sensor that currently receives HOTAS cursor slew commands and, where applicable, TMS actions. At any moment, only one display can be the SOI. Valid SOI displays include any of the F-16's displays: the HUD, the right MFD, and the left MFD. However, some MFD formats are not valid SOI because they provide information or control but do not accept sensor-like slew or targeting inputs. The table below summarizes valid and invalid SOI MFD formats.

\begin{longtable}{C{4.5cm} C{4.5cm}}
	\caption{Valid and Invalid SOI MFD Format \label{tab:C4-S1-MFD-format}}\\
	
	\rowcolor{headerblue}
	\textbf{\color{white}Valid} & \textbf{\color{white}Invalid}\\
	\endfirsthead
	
	\rowcolor{headerblue}
	\textbf{\color{white}Valid} & \textbf{\color{white}Invalid}\\
	\endhead
	
	\multicolumn{2}{r}{\emph{Continues on next page}}\\
	\endfoot
	
	\endlastfoot
	
	Fire Control Radar (FCR); Targeting Pod (TGP); Horizontal Situation Display (HSD); HARM (HAD); Weapon (WPN) & Stores Management (SMS); Data Transfer Equipment (DTE); TEST; Blank/Inactive formats; Digital Flight Control System (FLCS); TACAN (TCN); Forward Looking Infrared (FLIR); Terrain Following Radar (TFR)
\end{longtable}

Visually, The currently active SOI is easily recognized:
\begin{itemize}
	\item On the \textbf{HUD}: an asterisk (\texttt{*}) appears in the upper left corner when HUD is SOI.
	\item On an \textbf{MFD}: a border outline appears around the edges of the display when it is SOI. When an MFD format is \textit{not} SOI, the text \texttt{NOT SOI} may appear on the format.
\end{itemize}

%--------------------------------------------------------------------------
%CHAPTER 4: SECTION 4.1.1.1
%--------------------------------------------------------------------------
\subsubsection{Valid SOI Displays by Master Mode}
\label{sec:C4-S1-S1-S1}

The availability of displays as valid SOI varies by master mode. Table~\ref{tab:C4-S1-SOI-by-mode} shows which displays can serve as SOI in the primary operational contexts. Note that in air-to-air employment modes (A-A, DGFT, MSL OVRD), the HUD is \textbf{never} available as SOI; A-A modes restrict the pilot to left and right MFD (FCR, HSD, or TGP formats) as the SOI. This constraint ensures that radar and tactical displays remain the primary source of truth in air-to-air engagements.

\small
\renewcommand{\arraystretch}{1.2}

\begin{longtable}{L{2.5cm} L{6.5cm} L{6.0cm}}
	\caption{Valid SOI Displays by Master Mode\label{tab:C4-S1-SOI-by-mode}}\\
	
	\rowcolor{headerblue}
	\textbf{\color{white}Master Mode} & \textbf{\color{white}Valid SOI Displays} & \textbf{\color{white}Constraints \& Notes} \\
	\endfirsthead
	
	\rowcolor{headerblue}
	\textbf{\color{white}Master Mode} & \textbf{\color{white}Valid SOI Displays} & \textbf{\color{white}Constraints \& Notes} \\
	\endhead
	
	\multicolumn{3}{r}{\emph{Continues on next page}}\\
	\endfoot
	
	\endlastfoot
	
	NAV (Navigation) & HUD and MFD (FCR, TGP, HSD, WPN, HAD formats) & All displays available. HUD is primary choice for situational awareness and NAV-specific tasks. \\
	\midrule
	A-A (Air-to-Air) & MDF only (FCR, HSD, TGP formats) & \textbf{HUD cannot be SOI.} SOI limited to radar and tactical displays. \\
	\midrule
	A-G (PRE) & HUD and MFD (FCR, TGP, WPN, HAD, HSD formats) & All displays available. \\
	\midrule
	A-G (VIS) & HUD and MFD (FCR, TGP, WPN formats) & Restricted to visual-capable displays. HUD used for target acquisition (e.g., AGM-65 VIS EO, DTOS, CCIP). \\
	\midrule
	DGFT (Dogfight) & MFD (FCR, HSD, TGP formats) & \textbf{HUD cannot be SOI}. \\
	\midrule
	MSL OVRD (Missile Override) & MFD (FCR, HSD, TGP formats) & \textbf{HUD cannot be SOI}. \\
\end{longtable}

As shown in Table~\ref{tab:C4-S1-SOI-by-mode}, the availability of SOI displays is strategically constrained by master mode to align pilot attention with the operational context. Navigation and air-to-ground modes offer maximum display flexibility, allowing the pilot to transition between HUD and MFD with radar, targeting pod, and weapon pages formats as needed.

Conversely, air-to-air and associated override modes eliminate HUD as a selectable SOI. This design reflects the fundamental principle that air-to-air engagements must be driven by primary sensor information rather than HUD-derived or symbology-based cues. However, HMCS provides an independent off-boresight capability (see Section~\ref{sec:C4-S1-S3} for details), independent of HUD not being a valid SOI.

For visual air-to-ground delivery (VIS modes), practical SOI control --- HUD in conjunction with MFD --- focuses on visual-capable sensors: the HUD for target designation and the MFD for optical tracking (TGP page) and weapon-specific control (WPN page), for instance. Although the radar and optical trackings may continue to provide ranging data in the background, HUD and sensor-video-driven symbology drive the attack in VIS delivery. See Section~\ref{sec:C4-S2-S1} for specific weapon examples (CCIP, DTOS, AGM-65 VIS EO, IAM VIS).

%--------------------------------------------------------------------------
% CHAPTER 4: SECTION 4.1.2
%--------------------------------------------------------------------------
\subsection{Role of DMS in SOI Selection}
\label{sec:C4-S1-S2}

The DMS manages SOI selection through two orthogonal axes of control:

\begin{itemize}
	\item \textbf{Vertical (Up / Down):} Selects \textit{which display} is SOI.
	\begin{itemize}
		\item \textbf{DMS Up:} Transfers SOI to the HUD (when permitted by master mode), detailed in Section~\ref{sec:C4-S2}.
		\item \textbf{DMS Down:} Cycles SOI between MFD, or from HUD to an MFD, detailed in Section~\ref{sec:C4-S3}.
	\end{itemize}
	\item \textbf{Horizontal (Left / Right):} Steps through MFD formats on the left or right MFD, independently of which display is the SOI (detailed in Section~\ref{sec:C4-S4}).
	\begin{itemize}
		\item \textbf{DMS Left:} Cycles the left MFD format (primary $\to$ secondary $\to$ tertiary).
		\item \textbf{DMS Right:} Cycles the right MFD format (primary $\to$ secondary $\to$ tertiary).
	\end{itemize}
\end{itemize}

%--------------------------------------------------------------------------
% CHAPTER 4: SECTION 4.1.3
%--------------------------------------------------------------------------
\subsection{HUD as SOI in A-A and HMCS Capabilities}
\label{sec:C4-S1-S3}

The restriction that HUD cannot be designated as SOI in A-A master mode applies exclusively to the \textbf{SOI routing architecture} — the mechanism by which HOTAS inputs (cursor slew, TMS commands) are delivered to a specific display. This architectural constraint \textbf{does not eliminate the functional capability} of the HUD or related displays to acquire, track, or cue targets in A-A operations.

The \textbf{Helmet Mounted Cueing System (HMCS)} exemplifies this distinction. Although HMCS is an extension of the HUD display system and shares the HUD's architectural restriction in A-A mode (neither can be designated as SOI via DMS), HMCS retains \textbf{independent off-boresight targeting capability}. For example, the pilot can slave an AIM-9 seeker to the HMCS visor line-of-sight without regard to which display (FCR, HSD, or TGP) is currently the SOI, and can employ HMCS-derived target acquisition cues that function independently of SOI status.

This principle reflects the fundamental architectural design: the SOI mechanism manages \textbf{HOTAS input routing}, while \textbf{display functional capabilities operate orthogonally}. Displays designated as SOI in A-A (FCR, HSD, TGP) receive these HOTAS inputs; displays not designated as SOI (HUD, HMCS) provide cueing and targeting functions through independent mechanisms (e.g., helmet line-of-sight, derived sensor information). Both types of functionality are essential to A-A operations, despite the SOI designation limitation.

For further technical details on HMCS capabilities and behavior in A-A contexts, refer to \dashref{2.5} (Helmet Mounted Cueing System).

%--------------------------------------------------------------------------
% CHAPTER 4: SECTION 4.2
%--------------------------------------------------------------------------
\section{DMS Up: HUD Designation as SOI}
\label{sec:C4-S2}

The \textbf{DMS Up} command attempts to designate the HUD as the Sensor of Interest (SOI). When the current master mode permits the HUD to be SOI (see Table~\ref{tab:C4-S1-SOI-by-mode} in Section~\ref{sec:C4-S1}): a short press of DMS Up immediately transfers SOI to the HUD.

When DMS Up successfully designates the HUD as SOI, the HUD SOI asterisk appears and any previous MFD SOI border is removed, as described in Section~\ref{sec:C4-S1-S1}. From that moment, all SOI-dependent HOTAS inputs (such as CURSOR/ENABLE for symbology positioning and TMS for target or waypoint designation) act on HUD symbology rather than on any MFD format. This simple visual feedback --- the HUD asterisk and loss of MFD SOI borders --- allows the pilot to confirm at a glance that all SOI-dependent commands are now applied to HUD-level cueing rather than to an MFD sensor page.

%--------------------------------------------------------------------------
% CHAPTER 4: SECTION 4.2.1
%--------------------------------------------------------------------------
\subsection{DMS Up Effectiveness}
\label{sec:C4-S2-S1}

DMS Up is only effective in master modes where the HUD is a valid SOI candidate. Table~\ref{tab:C4-S1-SOI-by-mode} in Section~\ref{sec:C4-S1} summarises these constraints. This subsection focuses on the modes where HUD SOI is both permitted and operationally significant, and then contrasts them with modes where DMS Up has no effect.

%--------------------------------------------------------------------------
% CHAPTER 4: SECTION 4.2.1.1
%--------------------------------------------------------------------------
\subsubsection{Master Modes Where DMS Up is Effective (HUD as SOI Permitted)}
\label{sec:C4-S2-S1-S1}

In master modes where the HUD can be SOI, DMS Up is the hands‑on command used to designate the HUD as SOI, enabling HUD‑based visual cueing in those modes.

\paragraph*{NAV (Navigation) Master Mode:}
In Navigation mode, the HUD is the primary reference for flight path, steering, and basic situational awareness. A short press of DMS Up immediately designates the HUD as SOI. With HUD as SOI, the pilot can use SOI‑dependent HOTAS inputs to interact with navigation‑related symbology on the HUD or HMCS: CURSOR/ENABLE slews the HUD/HMCS cursor or designator, and in specific functions such as HUD or HMCS MARK, TMS Up is used to stabilise the line of sight and create markpoints, while the MFD continue to provide background information.

The exact set of displays that may become SOI in NAV, and how they compare to the HUD, is documented in Table~\ref{tab:C4-S1-SOI-by-mode}; DMS Up simply selects the HUD within that set.

\paragraph*{A-G in VIS (Visual Air-to-Ground)---CCIP, DTOS, AGM-65 VIS, IAM-VIS:}
In air-to-ground visual delivery modes, targets are identified and designated visually by the pilot. The HUD becomes the \textbf{primary command interface} for visual cueing, and HUD as SOI is often a prerequisite for correct TMS and CURSOR behaviour. DMS Up is therefore \textbf{operationally critical}: whenever SOI has migrated to an MFD (for example, to TGP or WPN), a short press of DMS Up restores HUD as SOI and returns HOTAS inputs to the HUD.

HUD as SOI in typical A-G VIS contexts:

\begin{itemize}
	\item \textbf{CCIP visual deliveries:} The HUD displays a pipper (computed impact point or bullet track line). The pilot maneuvers the aircraft to place the pipper on the intended impact point and commands weapon release with the weapon release button. When the fire control system allows, CURSOR/ENABLE inputs referenced to HUD SOI can be used to refine the visual aimpoint or adjust reference cues without leaving the HUD-centric view.
	\item \textbf{AGM-65 Maverick VIS:} In AGM-65 VIS, the HUD shows a target designator (TD) box that slaves the Maverick seeker. With HUD as SOI (via DMS Up), CURSOR/ENABLE slews the TD box over the intended target, and TMS Up commands seeker lock. If SOI is inadvertently left on an MFD (for example, the WPN page), TMS inputs are routed to that display instead of to the HUD TD box, and visual target rejection or re-designation through the HUD will not work as intended until DMS Up restores HUD SOI.
	\item \textbf{IAM (JSOW/JDAM) visual deliveries (IAM-VIS):} In IAM-VIS, the HUD presents a TD box and associated A-G solution cues for visual designation. With HUD as SOI, the pilot refines the TD box position by aircraft manoeuvre and, when appropriate, by CURSOR/ENABLE inputs. TMS Up then designates and ground-stabilises the target. If SOI is on an MFD, these TMS commands act on the MFD sensor page instead, and the HUD cueing will not update as expected until HUD SOI is re-established with DMS Up.
\end{itemize}

DMS Up is also valid in non‑visual A‑G modes (such as CCRP or preplanned IAM deliveries), but in those cases HUD SOI is a convenience rather than a strict requirement, since targeting, cursor management, sighting‑point control, and sensor designation can be accomplished entirely with MFD‑centric SOI (FCR, TGP, HSD). The visual modes described earlier (DTOS, AGM‑65 VIS, HUD/HMCS MARK, IAM‑VIS) are where the coupling between DMS Up and TMS/CURSOR behaviour on HUD/HMCS symbology becomes operationally critical, as these modes rely on HUD/HMCS line‑of‑sight cueing and ground‑stabilization as primary designation methods.

%--------------------------------------------------------------------------
% CHAPTER 4: SECTION 4.2.1.2
%--------------------------------------------------------------------------
\subsubsection{Modes Where DMS Up is Ineffective (HUD as SOI Prohibited)}
\label{sec:C4-S2-S1-S2}

In air-to-air employment modes (A-A, DGFT, and MSL OVRD), pressing DMS Up has no effect on SOI selection, because the HUD is not a valid SOI candidate in these modes (see Section~\ref{sec:C4-S1-S1} and Table~\ref{tab:C4-S1-SOI-by-mode}). SOI remains on one of the MFD formats, such as the FCR, HSD, or TGP. The architectural rationale for this restriction, and the complementary role of HMCS in providing high off-boresight cueing in air-to-air, are developed in Section~\ref{sec:C4-S1-S3}.

%--------------------------------------------------------------------------
% CHAPTER 4: SECTION 4.2.2
%--------------------------------------------------------------------------
\subsection{DMS Up Usage Table}
\label{sec:C4-S2-S2}

The table below summarises DMS Up behaviour across representative master modes. It should be read together with Table~\ref{tab:C4-S1-SOI-by-mode}, which documents which displays are valid SOI candidates in each mode.

Each table entry specifies:

\begin{itemize}
	\item \textbf{State}: The operational context (Master Mode).
	\item \textbf{Direction}: The physical direction for pressing the DMS hat (Up, Down, Left, Right).
	\item \textbf{Action}: The press type (Short, Long, Long Hold).
	\item \textbf{Function}: What the DMS command activates or controls.
	\item \textbf{Effect / Nuance}: The resulting system behavior, including tactics and constraints.
	\item \textbf{Dash34}: Reference section in the Dash-34 manual.
	\item \textbf{Training}: Recommended BMS training missions for hands-on practice.
\end{itemize}

\begin{hotastable}{DMS Up Usage Across NAV, A-A and A-G Master Modes}
	NAV & Up & Short & Designate HUD as SOI & DMS Up is fully effective in NAV master mode. Pressing DMS Up immediately designates the HUD as SOI, placing the SOI asterisk on the HUD. With HUD/HMCS as SOI, CURSOR/ENABLE slews the HUD/HMCS cursor or designator, and in functions such as HUD or HMCS MARK, TMS Up is used to stabilise the line of sight and create markpoints. MFDs remain available for background navigation and systems information. & 2.1.1.2.3, 2.1.7.5.1, 2.1.7.5.4, 2.5.6.1 & \\\\
	A-A & Up & Short & Designate HUD as SOI & DMS Up is \textbf{ineffective} in A-A master mode. The avionics architecture restricts SOI to FCR, HSD, or TGP only. HUD cannot be SOI in this mode and functions purely as a passive display. & 2.1.1.2.3 & \\
	A-G & Up & Short & Designate HUD as SOI & DMS Up is fully effective in A-G master modes. Pressing DMS Up immediately designates HUD as SOI, and an asterisk appears in the upper left corner of the HUD. \textbf{In A-G visual modes (VIS)}, HUD is the operationally critical command interface for visual target designation and rejection via CURSOR and TMS inputs. If HUD loses SOI, visual cueing control is lost and must be recovered with DMS Up. & 2.1.1.2.3, 4.2.2.1, 4.2.2.1.1 & \trnref{10 (GP Bombs)}, \trnref{11 (LGB)}, \trnref{13 (Maverick)}, \trnref{14 (Maverick Adv)}, \trnref{15 (IAM)} \\
\end{hotastable}
\label{tab:C4-S2-DMS-Up-Usage}

%--------------------------------------------------------------------------
% CHAPTER 4: SECTION 4.2.3
%--------------------------------------------------------------------------
\subsection{DMS Up Exception States}
\label{sec:C4-S2-S3}

In certain states, DMS Up may be temporarily ineffective even in modes where HUD is normally a valid SOI:

\begin{itemize}
	\item \textbf{Snowplow (SP) PRE state (unstabilised):} When the pilot enters Snowplow mode (a specialised ground-stabilisation mode for slewing to arbitrary ground positions) and the SP position has not yet been stabilised with TMS Up, the SOI is effectively ``nowhere''. Both the A-G radar and TGP display \texttt{NOT SOI} on the MFDs, and DMS Up/Down commands are ineffective until the SP position is stabilised. Once stabilised, SOI returns to its previous state, and DMS Up becomes effective again (\dashref{4.2.1.4}).
	\item \textbf{MARK/OFLY Submode:} In the MARK/OFLY submode (a specialised target-acquisition context documented in \dashref{2.1.1.2.3}), the SOI cannot be designated at all. As a result, DMS inputs that would normally change the SOI have no effect in this state. This exception is rare in normal operations.
\end{itemize}

%--------------------------------------------------------------------------
% CHAPTER 4: SECTION 4.3
%--------------------------------------------------------------------------
\section{DMS Down: Toggle SOI Between Displays}
\label{sec:C4-S3}

DMS Down toggles the Sensor of Interest SOI among the displays available in the current master mode. As established in Section~\ref{sec:C4-S1-S1}, the set of valid SOI displays varies by mode: in NAV and A-G modes, the HUD is available; in air-to-air employment modes (A-A, DGFT, MSL OVRD), it is not.

Consequently, DMS Down behaves in two distinct ways:

\begin{itemize}
	\item In \textbf{NAV} and \textbf{A-G} modes, DMS Down toggles SOI through all available displays: HUD $\rightarrow$ L/R MFD $\rightarrow$ L/R MFD $\rightarrow$ HUD.
	\item In \textbf{air-to-air employment modes} (A-A, DGFT, MSL OVRD), DMS Down toggles SOI only between the two MFD (L/R MFD $\leftrightarrow$ L/R MFD), since the HUD is not a valid SOI candidate
\end{itemize}

This design ensures that DMS Up and DMS Down work together to manage SOI across all available displays in each operational context. It is \textbf{important to note} that \textbf{DMS Down} transitions SOI between displays—HUD and the two MFD—without changing which format is currently displayed on any MFD. The pilot executes hands-on commands on whatever format is available at the selected display. Format transitions \textbf{within an MFD} are controlled by DMS Right and DMS Left, covered in Section~\ref{sec:C4-S4}.

% CHAPTER 4: SECTION 4.3.1
\subsection{DMS Down Effectiveness}
\label{sec:C4-S3-S1}
DMS Down effectiveness depends on which displays can serve as SOI in the current master 
mode, as established in Section~\ref{sec:C4-S1-S1}.

% CHAPTER 4: SECTION 4.3.1.1
\subsubsection{Master Modes Where HUD is a Valid SOI Candidate}
\label{sec:C4-S3-S1-S1}

\paragraph{NAV (Navigation) Master Mode:}
In NAV, DMS Down toggles SOI through all valid candidates: HUD and both MFD. 
Repeated DMS Down presses create a continuous 3-step sequence: HUD 
$\rightarrow$ L/R MFD $\rightarrow$ L/R MFD $\rightarrow$ HUD. This allows the pilot to quickly move hands-on command focus between the HUD and the two MFD sensor displays for navigation and sensor management.

\paragraph{A-G in PRE (Preplanned Air-to-Ground) Mode:}
In A-G PRE, valid SOI candidates are the HUD and both MFD. DMS Down follows the 
same 3-step toggle pattern as NAV, moving SOI among the HUD and the two MFD. This allows the pilot to shift hands-on control focus while examining different sensor pages.

\paragraph{A-G in VIS (Visual Air-to-Ground)---CCIP, DTOS, AGM-65 VIS, IAM-VIS:}
In A-G VIS modes, valid SOI candidates are the HUD and both MFD. DMS Down toggles through the same 3-step pattern as in NAV. However, in A-G VIS, DMS Down becomes \textbf{operationally critical} rather than merely convenient.

A-G VIS delivery is fundamentally HUD-centric: the pilot acquires and designates the target visually using the HUD pipper (CCIP) or target designator box (AGM-65 VIS, IAM-VIS). These visual cues are controlled by CURSOR and TMS inputs, which are routed to whichever display is currently SOI. If SOI migrates to an MFD---such as TGP for sensor refinement, WPN for weapon status, or FCR/HSD for situational awareness---those same CURSOR and TMS commands will act on the MFD instead of the HUD, and visual designation on the HUD ceases to respond.

Therefore, \textbf{DMS Down and DMS Up} work in tandem in A-G VIS: DMS Down allows the pilot to temporarily move SOI to an MFD for sensor work or information review, while DMS Up immediately restores HUD SOI to resume visual designation. This up-down alternation is fundamental to efficient A-G VIS delivery and cannot be omitted without degrading command flow or situational awareness.

% CHAPTER 4: SECTION 4.3.1.2
\subsubsection{Modes Where HUD is NOT a Valid SOI Candidate}
\label{sec:C4-S3-S1-S2}

In air-to-air employment (A-A, DGFT, and MSL OVRD) modes, the avionics architecture restricts SOI to the MFD only. The HUD cannot be designated as SOI 
in these modes (see Sections~\ref{sec:C4-S1-S1} and \ref{sec:C4-S1-S3}). Consequently, DMS Down is limited to toggling SOI between the two MFD (L/R MFD $\leftrightarrow$ L/R MFD). This 2-way toggle allows the pilot to select which MFD sensor display receives hands-on command priority.

In A-A contexts, this is \textbf{operationally essential}: the pilot uses DMS Down to shift SOI focus between one MFD and the other so the pilot can access and have direct control over whichever format is being actually displayed: FCR for track management and missile employment, HSD for tactical picture and threat assessment or between the FCR and TGP for situational awareness or supplemental tracking. Efficient air-to-air engagement depends critically on rapid SOI management via DMS Down.

% CHAPTER 4: SECTION 4.3.2
\subsection{DMS Down Usage Table}
\label{sec:C4-S3-S2}

The table below summarises DMS Down behaviour across representative master modes. It should be read together with Table~\ref{tab:C4-S1-SOI-by-mode}, which documents which displays are valid SOI candidates in each mode.

Each table entry specifies:

\begin{itemize}
	\item \textbf{State}: The operational context (Master Mode).
	\item \textbf{Direction}: The physical direction for pressing the DMS hat (Up, Down, Left, Right).
	\item \textbf{Action}: The press type (Short, Long, Long Hold).
	\item \textbf{Function}: What the DMS command activates or controls.
	\item \textbf{Effect / Nuance}: The resulting system behavior, including tactics and constraints.
	\item \textbf{Dash34}: Reference section in the Dash-34 manual.
	\item \textbf{Training}: Recommended BMS training missions for hands-on practice.
\end{itemize}

\begin{hotastable}{DMS Down Usage Across NAV, A-A, and A-G Master Modes}
	NAV & Down & Short & Toggle SOI cycle through displays & 
	DMS Down toggles SOI through HUD $\rightarrow$ L/R MFD $\rightarrow$ L/R MFD $\rightarrow$ HUD. With HUD as SOI, hands-on commands (CURSOR/ENABLE, TMS) manage HUD navigation symbology. Pressing DMS Down transfers SOI to the next display; the pilot can rotate through all three displays sequentially. &
	--- \\
	
	A-A & Down & Short & Toggle SOI between MFD only & 
	DMS Down toggles SOI only between the two MFD. The HUD cannot be SOI in A-A and remains a passive display. This is the primary HOTAS method for selecting which MFD sensor page receives hands-on command priority for track management, situational awareness, and weapons employment. & 
	2.1.1.2.3, 2.1.6.3 & 
	\trnref{18 BARCAP}, \trnref{17B IFF Intercept} \\
	
	A-G & Down & Short & Toggle SOI between HUD and MFD sensor pages & 
	DMS Down toggles SOI through HUD $\rightarrow$ L/R MFD $\rightarrow$ L/R MFD $\rightarrow$ HUD. In A-G PRE, DMS Down is a convenience tool for shifting hands-on focus between HUD and MFD sensor pages. \textbf{In A-G VIS (CCIP, DTOS, AGM-65 VIS, IAM-VIS), DMS Down is operationally critical:} the HUD is the primary visual designation interface. DMS Down allows the pilot to alternate between HUD visual cueing (TMS/CURSOR steering, pipper control, TD box positioning) and MFD sensor work (TGP search/refine, WPN status, FCR A-G ranging). Loss of HUD as SOI in A-G VIS prevents proper visual designation and must be recovered with DMS Up. & 
	2.1.1.2.3, 2.1.6.3 & 
	\trnref{10 GP Bombs}, \trnref{11 LGB}, \trnref{13 Maverick}, \trnref{14 Maverick Adv}, \trnref{15 IAM} \\
\end{hotastable}

% CHAPTER 4: SECTION 4.3.3
\subsection{DMS Down Exception States}
\label{sec:C4-S3-S3}

In certain special states and submodes, DMS Down may be temporarily ineffective, as a direct reflection of DMS Up (see Section~\ref{sec:C4-S2-S3}).

\begin{itemize}
	\item \textbf{Snowplow (SP) PRE State (not stabilised):} When the pilot enters Snowplow mode (a specialised A-G ground-stabilisation mode for slewing to arbitrary ground positions) and the SP position has not yet been stabilised with TMS Up, the SOI is effectively ``nowhere.'' Both the A-G radar and TGP MFD displays show \texttt{NOT SOI}, and neither display is designated as SOI. As a result, \textbf{DMS Down has no effect} in this state: the toggle mechanism has nowhere to advance SOI to. Once the SP position is stabilised with TMS Up (pressing TMS Up on the HUD), SOI returns to its previous designated display, and DMS Down resumes normal toggling behaviour.
	
	\item \textbf{MARK/OFLY Submode:} In the MARK/OFLY submode (a specialised target-acquisition context documented in \dashref{2.1.1.2.3}), the SOI cannot be designated or changed at all. Consequently, \textbf{DMS Down has no effect} in MARK/OFLY: you cannot toggle SOI when SOI designation itself is locked. This submode is rare in normal operations but is important to recognise if you encounter it during unusual procedures or system states.
\end{itemize}
%--------------------------------------------------------------------------
%CHAPTER 4: SECTION 4.4
%--------------------------------------------------------------------------
\section{DMS Left/Right: Multifunction Display Format Cycling}
\label{sec:C4-S4}

%--------------------------------------------------------------------------
% SECTION 4.4.1
%--------------------------------------------------------------------------

\subsection{Concept}
\label{sec:C4-S4-S1}

The DMS Left and DMS Right commands are fundamentally orthogonal to the DMS Up and DMS Down controls described in Sections~\ref{sec:C4-S2} and~\ref{sec:C4-S3}. Whereas DMS Up and DMS Down select \textit{which display} (HUD, Left MFD or Right MFD) becomes the Sensor of Interest (SOI), DMS Left and DMS Right cycle through different \textit{format pages} displayed on \textit{each MFD}, independently of which display is currently designated as SOI --- \textbf{even if the HUD is the actual SOI, DMS Left/Right will actuate on each MFD}.

\paragraph{Definition of Format Cycling:}

Each MFD can display up to three different format pages, pre-configured during mission planning via the Data Transfer Cartridge (DTC) or directly by the pilot, in-flight. These three format pages are designated as PRIMARY, SECONDARY, and TERTIARY (see section \ref{sec:C4-S4-S2}). DMS Left and DMS Right allow the pilot to cycle through these slots, advancing to the next format page with each button press.
%--------------------------------------------------------------------------
% SECTION 4.4.1.1
%--------------------------------------------------------------------------
\subsubsection{Format Cycling and SOI Selection --- Distinctions}
\label{sec:C4-S4-S1-S1}

The critical distinction is this: DMS Up/Down operates on the \textbf{display selection axis} (designation of which display is SOI), whereas DMS Left/Right operates on the \textbf{format pages axis} (which page is shown on an MFD). A pilot can simultaneously manage both axes:

\begin{itemize}

\item Press DMS Down to transfer SOI from the Left MFD to the Right MFD (changes which display receives HOTAS commands).

\item Press DMS Right to cycle the Right MFD to a different format page (changes what is displayed, independent of SOI).

\end{itemize}

This orthogonality is operationally powerful: the pilot can organize the MFD for increased situational awareness while simultaneously managing which display receives HOTAS inputs --- which display is SOI. So, pressing DMS Right changes the Right MFD format even if it is not SOI; the same applies to the Left MFD by pressing DMS Left.

The two mechanisms (SOI definition and MFD format cycling) do not interfere with each other and DMS Left/Right is not restricted by the current Master Mode.

Beyond that, DMS Left and DMS Right are \textbf{completely independent} from each other. DMS Left controls the Left MFD \textit{only}; DMS Right controls the Right MFD \textit{only}: pressing DMS Right won't, for instance, go aback to the previous left MFD format.

\begin{center}
	$\mathtt{DMS\ LEFT} \longrightarrow \mathtt{LEFT\ MFD}$\\[1em]
	$\mathtt{DMS\ RIGHT} \longrightarrow \mathtt{RIGHT\ MFD}$
\end{center}

In summary, \textbf{both MFD cycle independently and DMS Left/Right pressings don't affect SOI designation}, this is accomplished by DMS Up/Down. This independence allows the pilot to organize a visual workspace suited to the mission, so they never have to ``choose'' which display to look at or which to control; both are available simultaneously via independent mechanisms.

%--------------------------------------------------------------------------
% SECTION 4.4.2
%--------------------------------------------------------------------------
\subsection{MFD Configuration}
\label{sec:C4-S4-S2}
%--------------------------------------------------------------------------
% SECTION 4.4.2.1
%--------------------------------------------------------------------------
\subsubsection{Display Format Configuration via DTC}
\label{sec:C4-S4-S2-S1}

Each Master Mode (A-A, A-G, NAV) and also DGFT and MSL OVRD has its own independent three-slot configuration, as preset in the DTC. When the pilot switches Master Modes in-flight, the avionics automatically load the format configuration for that mode and display the chosen format. Every press of DMS Left/Right will act upon the new set of three-slot formats.

The formats can also be reconfigured by the pilot in flight, by pressing again the OSB corresponding to the actual format being displayed. See \dashref{2.1.6.2} for a comprehensive explanation.

Below is a list of every possible display pages currently present in Falcon BMS that could be configured as an MFD format, either through the DTC or by the pilot in-flight. Note that not all of them can be SOI. When selecting, in any MFD, a format that can't be designated SOI, the \textit{SOI is automatically transferred to the other MFD}.

\small
\renewcommand{\arraystretch}{1.2}

\begin{longtable}{L{1.8cm} L{5.5cm} L{6.5cm} L{2.5cm}}
	\caption{Falcon BMS Possible MFD Formats\label{table:C4-S4-S2-S1}}\\
	
	\rowcolor{headerblue}
	\textbf{\color{white}Acronym} & \textbf{\color{white}Full Name} & \textbf{\color{white}Definition} & \textbf{\color{white}Can be SOI} \\
	\midrule
	\endfirsthead
	
	\rowcolor{headerblue}
	\textbf{\color{white}Acronym} & \textbf{\color{white}Full Name} & \textbf{\color{white}Definition} & \textbf{\color{white}Can be SOI} \\
	\midrule
	\endhead
	
	\midrule
	\multicolumn{4}{r}{\emph{Continues on next page}}\\
	\endfoot
	
	\bottomrule
	\endlastfoot
	
	FCR & Fire Control Radar & Provides air-to-air and air-to-ground radar detection, tracking, and targeting data with multiple search and track modes for weapons employment & YES \\
	
	HSD & Horizontal Situation Display & Presents tactical navigation, situational awareness, and positioning information on a moving map display for mission planning & YES \\
	
	TGP & Targeting Pod & Displays targeting pod imagery for target acquisition, tracking, identification, and laser designation of targets & YES \\
	
	WPN & Weapon Management & Shows weapons status, aircraft ordnance configuration, and munitions management for air-to-ground missions & YES \\
	
	HAD & HARM Attack Display & Provides detection and targeting information from air defense radar sources for anti-radiation warfare missions & YES \\
	
	FLIR & Forward Looking Infrared Navigation Pod & Displays thermal imaging data for navigation, target detection, and low-level flight operations in degraded visibility & NO \\
	
	TFR & Terrain Following Radar Navigation Pod & Presents terrain elevation and clearance data for automated low-level navigation and terrain avoidance & NO \\
	
	SMS & Stores Management System & Displays current weapons configuration, loadout, and stores management parameters and status information & NO \\
	
	TCN & TACAN Format & Shows TACAN navigation aid position and bearing information for tactical air navigation and station keeping & NO \\
	
	DTE & Data Transfer Equipment & Provides interface and status for external data link communications with ground stations and other aircraft & NO \\
	
	FLCS & Digital Flight Control System & Displays flight control system parameters, status, and diagnostics for aircraft control system monitoring & NO \\
	
	TEST & Test Format & Provides system test and diagnostic pages for built-in test (BIT) functions and aircraft system verification & NO \\
	
	BLANK & Blank Format & Displays an empty/blank page with no symbology or information for display configuration flexibility & NO \\
	
\end{longtable}
%--------------------------------------------------------------------------
% SECTION 4.4.2.2
%-------------------------------------------------------------------------
\subsubsection{DTC Customization in Falcon BMS:}
\label{sec:C4-S4-S2-S2}

During mission planning in Falcon BMS (2D map screen) the pilot can access the DTC configuration by pressing its corresponding button on the right side bar (see User Manual §§ 5.1 and 9.3.4.2 for extensive explanations on DTC use in-game). In the MODES tab of the DTC configuration menu (see User Manual § 5.1.4 for extensive explanations on hot to set the formats), the pilot can assign any valid format to the three available slots of all suported modes (A-A, A-G, NAV, DGFT and MSL OVRD), and define the format (not necessarily the PRIMARY one) that will be firstly displayed on both MFD when entering that specific Master Mode in-flight.

All modifications to the DTC must be saved and loaded in the 2D map screen before taking off for any mission, as stated in the User Manual §§ 5.1.1, 5.1.9 e 9.4. These customizations are then stored in the DTC and loaded either manually or automatically when the pilot enters the 3D. They persist for the duration of the flight, unless the pilot changes them in-flight through the OSB buttons.
%--------------------------------------------------------------------------
% SECTION 4.4.2.3
%-------------------------------------------------------------------------
\subsubsection{Primary, Secondary, and Tertiary format pages}
\label{sec:C4-S4-S2-S3}

The PRIMARY, SECONDARY, and TERTIARY format pages configured by the pilot are accessed by pressing the corresponding button on the bottom row of each MFD. Each format page corresponds to one of the three central buttons in the lower OSB row of each MFD:

\begin{itemize}
    \item \textbf{OSB 14:} PRIMARY slot
    \item \textbf{OSB 13:} SECONDARY slot
    \item \textbf{OSB 12:} TERTIARY slot
\end{itemize}

The diagram below illustrates the OSB layout in the F-16 MFD:

\begin{figure}[H]
	\centering
	\includegraphics[width=0.36\textwidth]{MFD.jpg}
	\caption{F-16 MFD Representation. Adapted from an AI-generated image by Perplexity AI. Free to use and modify per Perplexity Terms of Service, Section 2.3.1 (\url{https://www.perplexity.ai/hub/legal/perplexity-api-terms-of-service}).}
	\label{fig:f16_MFD}
\end{figure}

Pressing the OSB corresponding to each format page is \textit{equivalent} to pressing DMS Left for the Left MDF and DMS Right for the Right MFD. When using DMS Left/Right, the formats will cycle from the inside to the outside of the MFD, in a constant sequencing, allowing pilots to develop muscle memory:

\begin{itemize}
	\item \textbf{Left MFD:} OSB 12 → OSB 13 → OSB 14
	\item \textbf{Right MFD:} OSB 14 → OSB 13 → OSB 12	
\end{itemize}

The three-slot architecture provides mission planning flexibility. During mission planning in the BMS 2D map screen, the pilot pre-configures which format pages are most useful for a given Master Mode.
%--------------------------------------------------------------------------
% SECTION 4.4.2.4
%-------------------------------------------------------------------------
\subsubsection{Cycling Constraints and Edge Cases}
\label{sec:C4-S4-S2-S4}

\paragraph{BLANK Format Skipping:}

If one or more format slots are configured as \texttt{BLANK} (meaning no format is assigned to that slot, either by choice or by default), DMS Left/Right pressings will automatically skip the BLANK format slot and advance to the next non-BLANK format slot.

\paragraph{Non-SOI-Candidate Formats:}

Some format pages are \textit{not valid} candidates for SOI designation. For example, the SMS (Stores Management System) format is not a valid SOI candidate. If a pilot customizes an MFD to include SMS in one of the three slots, and that specific MFD was SOI before cycling formats, SOI will be automatically transferred to the other MFD.

\paragraph{Format Persistence Across Master Mode Change:}

When the pilot changes Master Mode, the displayed formats on both MFD reset to the preferred formats of the actual chosen mode. There is \textit{no carryover} of the previously displayed slot.

%--------------------------------------------------------------------------
% SECTION 4.4.3: USAGE TABLE
%--------------------------------------------------------------------------

\subsection{DMS Left/Right Usage Table}
\label{sec:C4-S4-S3}

The table below summarizes DMS Left and DMS Right behavior across all Master Modes. Because format cycling is \textit{identical} in all modes, the table shows a single row for each DMS direction, applicable to every Master Mode: A-A (including DGFT and MSL OVRD), A-G and NAV.

Each table entry specifies:

\begin{itemize}
	\item \textbf{State}: The operational context (Master Mode).
	\item \textbf{Direction}: The physical direction for pressing the CMS hat (Left or Right).
	\item \textbf{Action}: The press type (Short, Long, Long Hold).
	\item \textbf{Function}: What the DMS command activates or controls.
	\item \textbf{Effect / Nuance}: The resulting system behavior, including tactics and constraints.
	\item \textbf{Dash34}: Reference section in the Dash-34 manual.
	\item \textbf{Training}: Recommended BMS training missions for hands-on practice.
\end{itemize}

\paragraph{Training Mission Selection Rationale:} To demonstrate the operational utility of DMS Left/Right without overloading the HOTAS table, one representative mission was selected from each Master Mode (NAV, A-A, and A-G). The selection prioritized missions requiring frequent alternation between MFD formats during critical tactical phases. These three missions span the full spectrum of scenarios where DMS Left/Right provides measurable operational advantage over OSB-based navigation.

Notwithstanding the decision to present only three representative training missions, any complex mission involving A-A and A-G weapons employment will benefit from the application of DMS Left/Right.

\begin{hotastable}{DMS Left/Right Format Cycling Across All Master Modes}

A-A, A-G, NAV & Left & Short & Cycle Left MFD format & DMS Left cycles the Left MFD through its configured 3-slot sequence: PRIMARY → SECONDARY → TERTIARY → PRIMARY (wrap-around). If BLANK slots are present, they are skipped automatically. Each press advances one step; no continuous cycling on hold. SOI designation of any valid display is unaffected. & 2.1.1.2.1, 2.1.6.3 & \trnref{8 (TFR/FLIR)}, \trnref{28 (SEAD-EW)}, \trnref{18 (BARCAP)} \\

\hline

A-A, A-G, NAV & Right & Short & Cycle Right MFD format & DMS Right cycles the Right MFD through its configured 3-slot sequence: PRIMARY → SECONDARY → TERTIARY → PRIMARY (wrap-around). If BLANK slots are present, they are skipped automatically. Each press advances one step; no continuous cycling on hold. SOI designation of any valid display is unaffected. & 2.1.1.2.1, 2.1.6.3 & \trnref{8 (TFR/FLIR)}, \trnref{28 (SEAD-EW)}, \trnref{18 (BARCAP)} \\

\end{hotastable}

%--------------------------------------------------------------------------
% END OF SECTION
%--------------------------------------------------------------------------

\end{document}
