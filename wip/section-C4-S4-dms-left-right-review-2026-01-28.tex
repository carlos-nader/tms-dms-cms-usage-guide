%============================================================================
% LICENSING NOTICE
%============================================================================
% This WIP file, upon integration into guide-v.tex, becomes part of the
% TMS/DMS/CMS Usage Guide and is released under CC BY-NC 4.0.
% For details, see LICENSE in the repository root.
%============================================================================

%============================================================================
% FALCON BMS TMS/DMS/CMS HOTAS GUIDE
% WIP FILE TEMPLATE V1.0 — FINAL (Briefing v0.2.0.1 + TOC Fix)
%============================================================================

% IMPORTANTE: Este é um TEMPLATE padronizado para arquivos WIP (Work-In-Progress)
% que serão integrados ao guide.tex.

% Nomenclatura: Siga wip-naming-v1.3
% Padrão: section-C{N}-S{M}[-S{K}]-{titulo}-{status}-{data}.tex
% Exemplo: section-C5-S2-cms-actuation-hotas-tables-final-2026-01-10.tex

% Status: dev | review | final | approved | deprecated
% Locação: WIP/ (ativo) | ARCHIVE/ (aprovado/descartado)

%============================================================================
% PREAMBLE COMPLETO — TMS/DMS/CMS Usage Guide for Falcon BMS 4.38.1
% Gerado: 17 January 2026
% Status: Novo preâmbulo (report + twoside + titlesec + fancyhdr melhorado)
%============================================================================

\documentclass[11pt, a4paper, twoside]{report}

%--------------------------------------------------------------------------
% BASIC ENCODING AND LANGUAGE
%--------------------------------------------------------------------------

\usepackage[utf8]{inputenc}
\usepackage[T1]{fontenc}
\usepackage[english]{babel}

%--------------------------------------------------------------------------
% FONTS AND MICROTYPOGRAPHY
%--------------------------------------------------------------------------

\usepackage{lmodern}
\usepackage{microtype}

%--------------------------------------------------------------------------
% PAGE GEOMETRY AND LAYOUT
%--------------------------------------------------------------------------

\usepackage{geometry}
\geometry{a4paper, left=2.0cm, right=2.0cm, top=2.5cm, bottom=2.5cm}
\usepackage{setspace}
\onehalfspacing

%--------------------------------------------------------------------------
% COLORS AND LINKS
%--------------------------------------------------------------------------

\usepackage[table]{xcolor}
\definecolor{linkblue}{HTML}{004488}
\definecolor{linkred}{HTML}{882222}
\definecolor{headerblue}{HTML}{003366}
\definecolor{rowgray}{HTML}{F5F5F5}
\definecolor{subheadgray}{HTML}{E0E0E0}

\usepackage{soul}
\usepackage[pdfencoding=auto, psdextra, colorlinks=true, linkcolor=linkblue, citecolor=linkred, urlcolor=linkblue, breaklinks=true]{hyperref}
\usepackage{bookmark}
\usepackage{caption}
\captionsetup{font=footnotesize, labelfont=bf, justification=raggedright, singlelinecheck=false}

%--------------------------------------------------------------------------
% HEADERS AND FOOTERS (IMPROVED for report + twoside)
%--------------------------------------------------------------------------

\usepackage{fancyhdr}
\setlength{\headheight}{25pt}                    % Increased (was 15pt) for long names
\pagestyle{fancy}
\fancyhf{}                                        % Clear all
\fancyhead[LO,RE]{\small\textit{\leftmark}}     % Outer edge (odd left, even right): chapter name
\fancyhead[RO,LE]{\small\thepage}               % Inner edge (odd right, even left): page number
\fancyfoot{}                                      % No footer (page number in header)
\renewcommand{\headrulewidth}{0.4pt}
\renewcommand{\footrulewidth}{0pt}

%-------------------------------------------------------------------------
% CHAPTER FORMATTING AND SPACING (via titlesec)
%--------------------------------------------------------------------------

\usepackage{titlesec}

% Chapter format: display style with custom spacing
\titleformat{\chapter}[display]
  {\normalfont\Large\bfseries}
  {\chaptertitlename~\thechapter}
  {20pt}
  {\Large}

% Chapter spacing: before=10pt (was 50pt), after=20pt (was 40pt)
\titlespacing{\chapter}
  {0pt}
  {10pt}      % Space BEFORE chapter title
  {20pt}      % Space AFTER chapter title
  [0pt]

% Section spacing (optional, for consistency)
\titlespacing{\section}
  {0pt}
  {15pt}
  {10pt}
  [0pt]

%--------------------------------------------------------------------------
% TABLES AND MACROS
%--------------------------------------------------------------------------

\usepackage{booktabs}
\usepackage{array}
\usepackage{longtable}
\usepackage{tabularx}

% Custom Columns
\newcolumntype{L}[1]{>{\raggedright\arraybackslash}p{#1}}
\newcolumntype{C}[1]{>{\centering\arraybackslash}p{#1}}
\newcolumntype{R}[1]{>{\raggedleft\arraybackslash}p{#1}}

% Macro for Visual Reference Links
\newcommand{\imglink}[1]{\hspace{2pt}\hyperref[#1]{\scriptsize\textbf{[Fig]}}}

%============================================================================
% HOTAS table environment (per Briefing v0.2.0.1)
%============================================================================

\newenvironment{hotastable}[1]{%
  \small
  \renewcommand{\arraystretch}{1.25}
  \begin{longtable}{L{1.6cm} L{1.0cm} L{1.0cm} L{3.4cm} L{5.8cm} L{1.4cm} L{1.4cm}}
  \caption{#1}\\
  \rowcolor{headerblue}
  \textbf{\color{white}State} &
  \textbf{\color{white}Dir} &
  \textbf{\color{white}Act} &
  \textbf{\color{white}Function} &
  \textbf{\color{white}Effect / Nuance} &
  \textbf{\color{white}Dash34} &
  \textbf{\color{white}Train} \\
  \endfirsthead
  \rowcolor{headerblue}
  \textbf{\color{white}State} &
  \textbf{\color{white}Dir} &
  \textbf{\color{white}Act} &
  \textbf{\color{white}Function} &
  \textbf{\color{white}Effect / Nuance} &
  \textbf{\color{white}Dash34} &
  \textbf{\color{white}Train} \\
  \endhead
  \multicolumn{7}{r}{\small\emph{Continued on next page}}\\
  \endfoot
  \endlastfoot
}{%
  \end{longtable}
}

%--------------------------------------------------------------------------
% SIMPLE REFERENCE MACROS FOR BMS DOCS
%--------------------------------------------------------------------------

\providecommand{\dashref}[1]{Dash-34~\S~#1}
\providecommand{\dashone}[1]{Dash-1~\S~#1}
\providecommand{\trnref}[1]{TRN~#1}
\providecommand{\trnman}{BMS Training Manual 4.38.1}
\providecommand{\bmsver}{Falcon BMS~4.38.1}
\providecommand{\dashrefs}[1]{\textit{TO 1F-16CMAM-34-1-1}, Dash-34, sections \texttt{#1}}

%--------------------------------------------------------------------------
% VERSION CONTROL MACROS
%--------------------------------------------------------------------------

\newcommand{\docversion}{0.3.2.0}
\newcommand{\docbuild}{20260120}
\newcommand{\docstartdate}{05 January 2026}
\newcommand{\docenddate}{20012026}
\newcommand{\chapterscompletedof}{2/7}
\newcommand{\tablesfilledpct}{0}
\newcommand{\fulldocversion}{\docversion+\docbuild}

%--------------------------------------------------------------------------
% GRAPHICS
%--------------------------------------------------------------------------

\usepackage{graphicx}
\graphicspath{{fig/}}
\usepackage{float}

%--------------------------------------------------------------------------
% TITLE
%--------------------------------------------------------------------------

\title{TMS, DMS and CMS Usage Guide for \bmsver}
\author{Carlos ``Metal'' Nader}
\date{Version \fulldocversion{} | Progress: Chapters \chapterscompletedof{} | Tables \tablesfilledpct{} | January 2026}

%============================================================================
% DOCUMENT BEGIN
%============================================================================

\begin{document}

\maketitle

\pagenumbering{roman}

%============================================================================
% TOC DEPTH CONFIGURATION (CRITICAL for report)
%============================================================================
\setcounter{tocdepth}{3}       % Show up to \subsubsection in TOC
\setcounter{secnumdepth}{3}    % Number up to \subsubsection

\newpage

\tableofcontents

\newpage

\pagenumbering{arabic}

%============================================================================
% WIP FILE METADATA (NOT RENDERED IN PDF)
%============================================================================

% File Name: section-C4-S4-dms-left-right-review-20260121.tex
% WIP Naming Convention: v1.3
% Target Chapter: C{N} (Chapter Title)
% Target Section: S{M} (Section Title)
% [Target Subsection: S{K} (Subsection Title) — OPTIONAL]
%
% WIP Status: review --- dev | review | final | approved | deprecated
% Created: 2026-01-20
% Last Modified: 2026-01-28
% Integration Status: NOT YET INTEGRATED | INTEGRATION TARGET: v0.{MINOR}.{PATCH}
%
% Narrative Completion: 0-100%
% Table Fill Status: 0-100%
%
% Notes: first draft - review stopped on section 4.4.2
% - [Describe what's done, what's pending, open questions]
% - [Cross-references to other WIP files or guide.tex sections]
% - [Any divergences from briefing or validation issues]

%============================================================================
%============================================================================
% TEMPLATE STRUCTURE — Replace section/subsection headers with your content
%============================================================================
% CONTENT BEGINS HERE

%============================================================================

\section{DMS Left/Right: Multifunction Display Format Cycling}

\label{sec:C4-S4}

%--------------------------------------------------------------------------

% SECTION 4.4.1: CONCEPT AND ORTHOGONALITY

%--------------------------------------------------------------------------

\subsection{Concept}

\label{sec:C4-S4-S1}

The DMS Left and DMS Right commands are fundamentally orthogonal to the DMS Up and DMS Down controls described in Sections~\ref{sec:C4-S2} and~\ref{sec:C4-S3}. Whereas DMS Up and DMS Down select \textit{which display} (HUD, Left MFD or Right MFD) becomes the Sensor of Interest (SOI), DMS Left and DMS Right cycle through different \textit{format pages} displayed on \textit{each MFD}, independently of which display is currently designated as SOI --- \textbf{even if the HUD is the actual SOI, DMS Left/Right will actuate on each MFD}.

\paragraph{Definition of Format Cycling:}

Each MFD can display up to three different format pages, pre-configured during mission planning via the Data Transfer Cartridge (DTC) or directly by the pilot, in-flight. These three format pages are designated as PRIMARY, SECONDARY, and TERTIARY (see section \ref{sec:C4-S4-S2}). DMS Left and DMS Right allow the pilot to cycle through these slots, advancing to the next format page with each button press.

\subsubsection{Format Cycling and SOI Selection --- Distinctions}
\label{sec:C4-S4-S1-S1}

The critical distinction is this: DMS Up/Down operates on the \textbf{display selection axis} (designation of which display is SOI), whereas DMS Left/Right operates on the \textbf{format pages axis} (which page is shown on an MFD). A pilot can simultaneously manage both axes:

\begin{itemize}

\item Press DMS Down to transfer SOI from the Left MFD to the Right MFD (changes which display receives HOTAS commands).

\item Press DMS Right to cycle the Right MFD to a different format page (changes what is displayed, independent of SOI).

\end{itemize}

This orthogonality is operationally powerful: the pilot can organize the MFD for increased situational awareness while simultaneously managing which display receives HOTAS inputs --- which display is SOI. So, pressing DMS Right changes the Right MFD format even if it is not SOI; the same applies to the Left MFD by pressing DMS Left.

The two mechanisms (SOI definition and MFD format cycling) do not interfere eith each other and DMS Rigth/Left is not restricted by the current Master Mode.

Beyond that, DMS Left and DMS Right are \textbf{completely independent} from each other. DMS Left controls the Left MFD \textit{only}; DMS Right controls the Right MFD \textit{only}: pressing DMS Right won't, for instance, go aback to the previous left MFD format.

\begin{center}
	$\mathtt{DMS\ LEFT} \longrightarrow \mathtt{LEFT\ MFD}$\\[1em]
	$\mathtt{DMS\ RIGHT} \longrightarrow \mathtt{RIGHT\ MFD}$
\end{center}

In summary, \textbf{both MFD cycle independently and DMS Left/Right pressings don't affect SOI designation}, this is accomplished by DMS Up/Down. This independence allows the pilot to organize a visual workspace suited to the mission, so they never have to ``choose'' which display to look at or which to control; both are available simultaneously via independent mechanisms.

%--------------------------------------------------------------------------

% SECTION 4.4.2: OPERATING PRINCIPLES

%--------------------------------------------------------------------------

\subsection{MFD Configuration}

\label{sec:C4-S4-S2}

\subsubsection{Display Format Configuration (DTC)}

\label{sec:C4-S4-S2-S1}

Each Master Mode (A-A, A-G, NAV) and also DGFT and MSL OVRD has its own independent three-slot configuration, as preset in the DTC. When the pilot switches Master Modes in-flight, the avionics automatically load the format configuration for that mode and display the chosen format. Every press of DMS Left/Right will act upon the new set of three-slot formats.

The formats can also be reconfigured by the pilot in flight, by pressing again the OSB corresponding to the actual format being displayed. See \dashref{2.1.6.2} for a comprehensive explanation.

Below is a list of every possible display pages currently present in Falcon BMS that could be configured as an MFD format, either through the DTC or by the pilot in-flight. Note that not all of them can be SOI. When selecting, in any MFD, a format that can't be designated SOI, the \textit{SOI is automatically transferred to the other MFD}.

\small
\renewcommand{\arraystretch}{1.2}

\begin{longtable}{L{1.8cm} L{5.5cm} L{6.5cm} L{2.5cm}}
	\caption{Falcon BMS Possible MFD Formats\label{table:C4-S4-S2-S1}}\\
	
	\rowcolor{headerblue}
	\textbf{\color{white}Acronym} & \textbf{\color{white}Full Name} & \textbf{\color{white}Definition} & \textbf{\color{white}Can be SOI} \\
	\midrule
	\endfirsthead
	
	\rowcolor{headerblue}
	\textbf{\color{white}Acronym} & \textbf{\color{white}Full Name} & \textbf{\color{white}Definition} & \textbf{\color{white}Can be SOI} \\
	\midrule
	\endhead
	
	\midrule
	\multicolumn{4}{r}{\emph{Continues on next page}}\\
	\endfoot
	
	\bottomrule
	\endlastfoot
	
	FCR & Fire Control Radar & Provides air-to-air and air-to-ground radar detection, tracking, and targeting data with multiple search and track modes for weapons employment & YES \\
	
	HSD & Horizontal Situation Display & Presents tactical navigation, situational awareness, and positioning information on a moving map display for mission planning & YES \\
	
	TGP & Targeting Pod & Displays targeting pod imagery for target acquisition, tracking, identification, and laser designation of air-to-ground targets & YES \\
	
	WPN & Weapon/Stores Management & Shows weapons status, aircraft ordnance configuration, and munitions management for air-to-air and air-to-ground missions & YES \\
	
	HAD & HARM Attack Display & Provides detection and targeting information from air defense radar sources for anti-radiation warfare missions & NO \\
	
	FLIR & Forward Looking Infrared Navigation Pod & Displays thermal imaging data for navigation, target detection, and low-level flight operations in degraded visibility & NO \\
	
	TFR & Terrain Following Radar Navigation Pod & Presents terrain elevation and clearance data for automated low-level navigation and terrain avoidance & NO \\
	
	SMS & Stores Management System & Displays current weapons configuration, loadout, and stores management parameters and status information & NO \\
	
	TCN & TACAN Format & Shows TACAN navigation aid position and bearing information for tactical air navigation and station keeping & NO \\
	
	DTE & Data Transfer Equipment & Provides interface and status for external data link communications with ground stations and other aircraft & NO \\
	
	FLCS & Digital Flight Control System & Displays flight control system parameters, status, and diagnostics for aircraft control system monitoring & NO \\
	
	TEST & Test Format & Provides system test and diagnostic pages for built-in test (BIT) functions and aircraft system verification & NO \\
	
	BLANK & Blank Format & Displays an empty/blank page with no symbology or information for display configuration flexibility & NO \\
	
\end{longtable}

\paragraph{DTC Customization in Falcon BMS:}

During mission planning in Falcon BMS (2D map screen) the pilot can access the DTC configuration by pressing its corresponding button on the right side bar (see User Manual §§ 5.1 and 9.3.4.2 for extensive explanations on DTC use in-game). In the MODES tab of the DTC configuration menu (see User Manual § 5.1.4 for extensive explanations on hot to set the formats), the pilot can assign any valid format to the three available slots of all suported modes (A-A, A-G, NAV, DGFT and MSL OVRD), and define the format (not necessarily the PRIMARY one) that will be firstly displayed on the MFD in use when entering that specific Master Mode in-flight.

All modifications to the DTC must be saved and loaded in the 2D map screen before taking off for any mission, as stated in the User Manual §§ 5.1.1, 5.1.9 e 9.4. These customizations are then stored in the DTC and loaded automatically when the pilot takes off. They persist for the duration of the flight, unless the pilot changes them in-flight through the OSB buttons.

\subsubsection{Primary, Secondary, and Tertiary format pages}

\label{sec:C4-S4-S2-S2}

\hl{The PRIMARY, SECONDARY, and TERTIARY format pages configured by the pilot are accessed by pressing the corresponding button on the bottom row of each MFD. Each format page corresponds to one of the three central buttons in the lower OSB row of each MFD}:

\begin{itemize}

\item \textbf{OSB 14 (left button):} PRIMARY slot

\item \textbf{OSB 13 (center button):} SECONDARY slot

\item \textbf{OSB 12 (right button):} TERTIARY slot

\end{itemize}

\hl{The diagram below illustrates the OSB layout in the F-16 MFD}:

\begin{figure}[H]
	\centering
	\includegraphics[width=0.36\textwidth]{MFD.jpg}
	\caption{F-16 MFD Representation. Adapted from an AI-generated image by Perplexity AI. Free to use and modify per Perplexity Terms of Service, Section 2.3.1 (\url{https://www.perplexity.ai/hub/legal/perplexity-api-terms-of-service}).}
	\label{fig:f16_MFD}
\end{figure}

\hl{The three-slot architecture provides mission planning flexibility. During mission planning in the BMS Briefing, the pilot pre-configures which format pages are most useful for a given Master Mode. For instance, in air-to-air mode, the pilot might configure}:

\begin{itemize}

\item PRIMARY (OSB 14) = FCR (Fire Control Radar page)

\item SECONDARY (OSB 13) = HSD (Horizontal Situation Display)

\item TERTIARY (OSB 12) = TGP (Targeting Pod page, for auxiliary visual cues)

\end{itemize}

\hl{With three slots, the pilot can quickly access the most relevant pages via DMS Left/Right presses. This is especially important in time-critical engagements}.

\subsubsection{Format Cycling Mechanism: The Wrap-Around Sequence}

\label{sec:C4-S4-S2-S3}

\paragraph*{Cycling Direction:}

DMS Left and DMS Right advance through format slots in an anti-clockwise direction relative to the OSB button layout. For the Left MFD:

\begin{itemize}

\item DMS Left press 1: PRIMARY (OSB 14) → SECONDARY (OSB 13)

\item DMS Left press 2: SECONDARY (OSB 13) → TERTIARY (OSB 12)

\item DMS Left press 3: TERTIARY (OSB 12) → PRIMARY (OSB 14) [wrap-around]

\item DMS Left press 4: PRIMARY → SECONDARY again [cycle repeats]

\end{itemize}

The same cyclic sequence applies to DMS Right for the Right MFD. The direction is consistent and predictable, allowing pilots to develop muscle memory.

\paragraph*{Wrap-Around Behavior:}

After cycling through TERTIARY, the next DMS press returns to PRIMARY, creating an infinite loop. There is no ``off'' state or ``hold'' position. The cycling is continuous and wrap-around is instantaneous. If a slot is configured as BLANK (unused), cycling skips it automatically and advances to the next occupied slot (see Section~\ref{sec:C4-S4-S3-S1}).

\paragraph*{Press Type: Short Press (Tap Only):}

DMS Left/Right respond to \textbf{short press only} (tap). There is no long-press or continuous-hold variant. A single tap advances to the next format; holding the button does \textit{not} cycle continuously. This behavior is consistent across all Master Modes and all configurations. If the pilot needs to cycle through multiple slots, separate taps are required.

\paragraph*{Example Scenario:}

A pilot in air-to-air mode has customized the Left MFD as FCR/HSD/TGP. Initially, the Left MFD displays FCR (PRIMARY). The pilot wants to check the tactical picture on HSD. The pilot presses DMS Left once. The Left MFD now displays HSD (SECONDARY). The pilot presses DMS Left again. The Left MFD now displays TGP (TERTIARY). The pilot decides to return to FCR. The pilot presses DMS Left once more. The Left MFD displays FCR again (wrap-around from TERTIARY back to PRIMARY). The SOI designation, if Left MFD was SOI, remains unchanged throughout these cycling operations.

%--------------------------------------------------------------------------

% SECTION 4.4.3: CYCLING CONSTRAINTS AND EDGE CASES

%--------------------------------------------------------------------------

\subsection{Cycling Constraints and Edge Cases}

\label{sec:C4-S4-S3}

\subsubsection{BLANK Format Skipping}

\label{sec:C4-S4-S3-S1}

If one or more format slots are configured as BLANK (meaning no format is assigned to that slot, either by choice or by default), cycling automatically skips over BLANK slots and advances to the next non-BLANK slot.

\paragraph*{Example:}

Configuration: PRIMARY = FCR, SECONDARY = BLANK, TERTIARY = HSD.

Cycling sequence:
\begin{itemize}

\item Press DMS Left (once): FCR → [skips BLANK] → HSD

\item Press DMS Left (again): HSD → [skips BLANK] → FCR [wrap-around]

\item Result: Only two visible formats cycle; BLANK is transparent.

\end{itemize}

This automatic skipping means that BLANK slots do \textit{not} create ``pauses'' or ``dead stops'' in cycling. The pilot never sees a BLANK page; cycling flows smoothly from one occupied slot to the next.

---

\subsubsection{Non-SOI-Candidate Formats (Edge Case)}

\label{sec:C4-S4-S3-S2}

In certain Master Modes, some format pages are \textit{not valid} candidates for SOI designation. For example, SMS (Stores Management System) is not a valid SOI candidate in air-to-air mode. If a pilot customizes the air-to-air configuration to include SMS in one of the three slots, cycling can advance to SMS. The MFD will display SMS, but the page will show ``NOT SOI'' text, indicating that HOTAS commands routed to this MFD will not affect the SMS page.

\paragraph*{Operational Implication:}

This is an edge case that arises from unusual customizations. For normal operations, pilots should configure the three slots with formats that are valid SOI candidates in the intended Master Mode (e.g., in A-A: FCR, HSD, TGP; in A-G: FCR, TGP, WPN). If a non-candidate format is encountered, the pilot simply cycles to another slot; the behavior is well-defined, and no malfunction occurs.

\textit{Note:} This section flags the edge case for completeness. Detailed behavior (e.g., whether SOI is automatically removed, whether HOTAS inputs are blocked) is beyond the scope of this guide and depends on implementation details documented in \dashref{2.1.1.2.3}.

---

\subsubsection{Format Persistence Across Master Mode Change}

\label{sec:C4-S4-S3-S3}

When the pilot changes Master Mode, the displayed format resets to the PRIMARY slot of the new mode. There is \textit{no carryover} of the previously viewed slot.

\paragraph*{Example:}

Pilot is in A-A mode, Left MFD is cycling through FCR/HSD/TGP, and the pilot has advanced to TERTIARY (TGP). The pilot then presses the A-G button to switch to air-to-ground mode. Immediately, the Left MFD resets to PRIMARY of the A-G configuration (let's say it's FCR or SMS, depending on customization). The fact that the pilot was viewing TERTIARY in A-A is not remembered in A-G.

\paragraph*{Why This Design:}

Each Master Mode has a distinct operational context and a distinct three-slot configuration. Automatically resetting to PRIMARY ensures the pilot starts from a known, predictable state in the new mode. This prevents confusion and aligns with the principle that Master Mode transitions should reset the pilot's display focus to the primary sensor of that mode.

%--------------------------------------------------------------------------

% SECTION 4.4.4: USAGE TABLE

%--------------------------------------------------------------------------

\subsection{DMS Left/Right Usage Table}

\label{sec:C4-S4-S4}

The table below summarizes DMS Left and DMS Right behavior across all Master Modes. Because format cycling is \textit{identical} in all modes, the table shows a single row for each DMS direction, applicable to every Master Mode: A-A, A-G and NAV.

\begin{hotastable}{DMS Left/Right Format Cycling Across All Master Modes}

A-A, A-G, NAV & Left & Short & Cycle Left MFD format & DMS Left cycles the Left MFD through its configured 3-slot sequence: PRIMARY → SECONDARY → TERTIARY → PRIMARY (wrap-around). If BLANK slots are present, they are skipped automatically. Each press advances one step; no continuous cycling on hold. SOI designation to any MFD is unaffected. & 2.1.1.2.1, 2.1.6.3 & — \\

\hline

A-A, A-G, NAV & Right & Short & Cycle Right MFD format & DMS Right cycles the Right MFD through its configured 3-slot sequence: PRIMARY → SECONDARY → TERTIARY → PRIMARY (wrap-around). If BLANK slots are present, they are skipped automatically. Each press advances one step; no continuous cycling on hold. Cycling Right does not affect Left MFD or SOI. & 2.1.1.2.1, 2.1.6.3 & — \\

\end{hotastable}

\paragraph*{How to Use This Table:}

\begin{enumerate}

\item Identify the Master Mode you are operating in (A-A, A-G, etc.). \textit{All modes use the same cycling rules.}

\item Determine which MFD you want to cycle: Left or Right.

\item Press DMS in the appropriate direction (Left or Right) for as many steps as needed to reach the desired format.

\item Check which format is now displayed. Refer to Section~\ref{sec:C4-S1} to understand valid SOI candidates if you need to manage hands-on control.

\end{enumerate}

\paragraph*{Special Notes:}

\begin{itemize}

\item \textbf{Short Press Only:} DMS Left/Right do not support long-press or hold variants. Each tap advances one slot.

\item \textbf{Training Column Empty:} The Training (Train) column is left blank as per author guidance; training mission references will be populated in a future update.

\item \textbf{Consistency Across Modes:} Unlike DMS Up/Down (Sections~\ref{sec:C4-S2} and~\ref{sec:C4-S3}), DMS Left/Right behavior is \textit{identical} in all Master Modes. Format cycling is a simple, mode-agnostic operation. Customization via DTC is the only variable.

\end{itemize}

---

%--------------------------------------------------------------------------

% END OF SECTION

%--------------------------------------------------------------------------

\end{document}
