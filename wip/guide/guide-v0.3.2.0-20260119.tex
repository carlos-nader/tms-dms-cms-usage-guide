% ============================================================================
% PREAMBLE COMPLETO — TMS/DMS/CMS Usage Guide for Falcon BMS 4.38.1
% Gerado: 17 January 2026
% Status: Novo preâmbulo (report + twoside + titlesec + fancyhdr melhorado)
% ============================================================================

\documentclass[11pt, a4paper, twoside]{report}

% --------------------------------------------------------------------------
% BASIC ENCODING AND LANGUAGE
% --------------------------------------------------------------------------

\usepackage[utf8]{inputenc}
\usepackage[T1]{fontenc}
\usepackage[english]{babel}

% --------------------------------------------------------------------------
% FONTS AND MICROTYPOGRAPHY
% --------------------------------------------------------------------------

\usepackage{lmodern}
\usepackage{microtype}

% --------------------------------------------------------------------------
% PAGE GEOMETRY AND LAYOUT
% --------------------------------------------------------------------------

\usepackage{geometry}
\geometry{a4paper, left=2.0cm, right=2.0cm, top=2.5cm, bottom=2.5cm}
\usepackage{setspace}
\onehalfspacing

% --------------------------------------------------------------------------
% COLORS AND LINKS
% --------------------------------------------------------------------------

\usepackage[table]{xcolor}
\definecolor{linkblue}{HTML}{004488}
\definecolor{linkred}{HTML}{882222}
\definecolor{headerblue}{HTML}{003366}
\definecolor{rowgray}{HTML}{F5F5F5}
\definecolor{subheadgray}{HTML}{E0E0E0}

\usepackage[pdfencoding=auto, psdextra, colorlinks=true, linkcolor=linkblue, citecolor=linkred, urlcolor=linkblue, breaklinks=true]{hyperref}
\usepackage{bookmark}

% --------------------------------------------------------------------------
% HEADERS AND FOOTERS (IMPROVED for report + twoside)
% --------------------------------------------------------------------------

\usepackage{fancyhdr}
\setlength{\headheight}{25pt}                    % Increased (was 15pt) for long names
\pagestyle{fancy}
\fancyhf{}                                        % Clear all
\fancyhead[LO,RE]{\small\textit{\leftmark}}     % Outer edge (odd left, even right): chapter name
\fancyhead[RO,LE]{\small\thepage}               % Inner edge (odd right, even left): page number
\fancyfoot{}                                      % No footer (page number in header)
\renewcommand{\headrulewidth}{0.4pt}
\renewcommand{\footrulewidth}{0pt}

% --------------------------------------------------------------------------
% CHAPTER FORMATTING AND SPACING (via titlesec)
% --------------------------------------------------------------------------

\usepackage{titlesec}

% Chapter format: display style with custom spacing
\titleformat{\chapter}[display]
  {\normalfont\Large\bfseries}
  {\chaptertitlename~\thechapter}
  {20pt}
  {\Large}

% Chapter spacing: before=10pt (was 50pt), after=20pt (was 40pt)
\titlespacing{\chapter}
  {0pt}
  {10pt}      % Space BEFORE chapter title
  {20pt}      % Space AFTER chapter title
  [0pt]

% Section spacing (optional, for consistency)
\titlespacing{\section}
  {0pt}
  {15pt}
  {10pt}
  [0pt]

% --------------------------------------------------------------------------
% TABLES AND MACROS
% --------------------------------------------------------------------------

\usepackage{booktabs}
\usepackage{array}
\usepackage{longtable}
\usepackage{tabularx}

% Custom Columns
\newcolumntype{L}[1]{>{\raggedright\arraybackslash}p{#1}}
\newcolumntype{C}[1]{>{\centering\arraybackslash}p{#1}}
\newcolumntype{R}[1]{>{\raggedleft\arraybackslash}p{#1}}

% Macro for Visual Reference Links
\newcommand{\imglink}[1]{\hspace{2pt}\hyperref[#1]{\scriptsize\textbf{[Fig]}}}

% ============================================================================
% HOTAS table environment (per Briefing v0.2.0.1)
% ============================================================================

\newenvironment{hotastable}[1]{%
  \small
  \renewcommand{\arraystretch}{1.25}
  \begin{longtable}{L{1.6cm} L{1.0cm} L{1.0cm} L{3.4cm} L{5.8cm} L{1.4cm} L{1.4cm}}
  \caption{#1}\\
  \rowcolor{headerblue}
  \textbf{\color{white}State} &
  \textbf{\color{white}Dir} &
  \textbf{\color{white}Act} &
  \textbf{\color{white}Function} &
  \textbf{\color{white}Effect / Nuance} &
  \textbf{\color{white}Dash34} &
  \textbf{\color{white}Train} \\
  \endfirsthead
  \rowcolor{headerblue}
  \textbf{\color{white}State} &
  \textbf{\color{white}Dir} &
  \textbf{\color{white}Act} &
  \textbf{\color{white}Function} &
  \textbf{\color{white}Effect / Nuance} &
  \textbf{\color{white}Dash34} &
  \textbf{\color{white}Train} \\
  \endhead
  \multicolumn{7}{r}{\small\emph{Continued on next page}}\\
  \endfoot
  \endlastfoot
}{%
  \end{longtable}
}

% --------------------------------------------------------------------------
% SIMPLE REFERENCE MACROS FOR BMS DOCS
% --------------------------------------------------------------------------

\providecommand{\dashref}[1]{Dash-34~\S~#1}
\providecommand{\dashone}[1]{Dash-1~\S~#1}
\providecommand{\trnref}[1]{TRN~#1}
\providecommand{\trnman}{BMS Training Manual 4.38.1}
\providecommand{\bmsver}{Falcon BMS~4.38.1}
\providecommand{\dashrefs}[1]{\textit{TO 1F-16CMAM-34-1-1}, Dash-34, sections \texttt{#1}}

% --------------------------------------------------------------------------
% VERSION CONTROL MACROS
% --------------------------------------------------------------------------

\newcommand{\docversion}{0.3.2.0}
\newcommand{\docbuild}{20260119}
\newcommand{\docstartdate}{05 January 2026}
\newcommand{\docenddate}{19 January 2026}
\newcommand{\chapterscompletedof}{3/7}
\newcommand{\tablesfilledpct}{Chapter 5 and Chapter 4 - 4.1, 4.2 and 4.3}
\newcommand{\fulldocversion}{\docversion+\docbuild}

% --------------------------------------------------------------------------
% GRAPHICS
% --------------------------------------------------------------------------

\usepackage{graphicx}
\graphicspath{{fig/}}
\usepackage{float}

% --------------------------------------------------------------------------
% TITLE
% --------------------------------------------------------------------------

\title{TMS, DMS and CMS Usage Guide for \bmsver}
\author{Carlos ``Metal'' Nader}
\date{Version \fulldocversion{} | Progress: Chapters \chapterscompletedof{} | Tables \tablesfilledpct{} | January 2026}

% ============================================================================
% DOCUMENT BEGIN
% ============================================================================

\begin{document}

\maketitle

\pagenumbering{roman}

% ============================================================================
% TOC DEPTH CONFIGURATION (CRITICAL for report)
% ============================================================================
\setcounter{tocdepth}{3}       % Show up to \subsubsection in TOC
\setcounter{secnumdepth}{3}    % Number up to \subsubsection

\newpage
\tableofcontents
\newpage
\pagenumbering{arabic}

% ============================================================================
% CONTENT BEGINS HERE (use \chapter{}, \section{}, \subsection{}, etc.)
% ============================================================================
%--------------------------------------------------------------------------
% CHAPTER 1: INTRODUCTION
%--------------------------------------------------------------------------
\chapter{Introduction}
\label{chap:C1}

This document is a community-made reference guide for \bmsver, focused on practical use of three specific HOTAS controls: the Target Management Switch (TMS), the Display Management Switch (DMS), and the Countermeasures Management Switch (CMS). Although this guide was developed under Falcon BMS 4.38.1, the fundamental behavior of these switches has remained constant since at least Falcon BMS 4.36, making this guide applicable to almost any player of Falcon BMS.

Although other controls exist on the F-16 throttle and stick---such as the Communication Switch, the Dogfight/MRM Override, and the RDR Cursor Enable control---they are mentioned only when essential to understanding the behavior and context of TMS, DMS, and CMS.\footnote{Falcon BMS core avionics and weapons behaviour are documented in \dashone{2} and \dashref{2}. \trnman{} describes how these systems are trained in practice.}

Its goal is to reorganize information that is spread across the Dash-1, Dash-34 and the BMS Training Manual into mode-based tables and short explanations, so that virtual pilots can quickly understand what each switch press does in a given context.\footnote{See \dashref{2.1.5} (Hands-On Controls) and the foreword of \trnman{} for the role of TMS, DMS and CMS in BMS training.}

This is the \emph{TMS, DMS and CMS Usage Guide---Version \fulldocversion}, prepared between \docstartdate{} and \docenddate{}, and created with extensive assistance from an AI language model (Perplexity AI) to help structure, cross-reference and format the material. The human author remains fully responsible for every choice of content, interpretation and final wording, and any mistakes or omissions are attributable to the author alone, not to the AI system.

This work is entirely unofficial. The author is not affiliated with Benchmark Sims, MicroProse, any real-world air force, or any aircraft or weapons manufacturer. All interpretations, simplifications, errors and omissions in this document are solely the responsibility of the author and must not be attributed to the Falcon BMS development team or to any real-world organization.\footnote{Compare the official disclaimer and copyright statements in the foreword of \trnman{}.} Nothing in this document should ever be used for real-world operations, training, or procedures.

%--------------------------------------------------------------------------
% SECTION 1.1: DEVELOPMENT TIMELINE
%--------------------------------------------------------------------------
\section{Development timeline and status}
\label{sec:C1-S1}

This guide was developed in structured phases, beginning \docstartdate{}. Current development status and targets are shown in Table 1 below.

\begin{table}[h]
\centering
\caption{Development Status Snapshot}
\begin{tabular}{L{3.2cm} L{2.5cm} L{2.5cm}}
\toprule
\textbf{Metric} & \textbf{Current} & \textbf{Target} \\
\midrule
Start Date & \docstartdate & --- \\
Current Version & \fulldocversion & --- \\
Chapters Complete & \chapterscompletedof & 7/7 \\
Tables Filled & \tablesfilledpct & 100\% \\
\bottomrule
\end{tabular}
\end{table}

The development roadmap is structured in three phases: (1) \emph{Chapter scaffolding} (Versions 0.1.0--0.7.0), during which all chapters receive narrative content and table structures; (2) \emph{Table population} (Versions 1.0.0--1.0.5), during which all tables are filled with complete HOTAS behavior descriptions and diagrams are generated; and (3) \emph{Review and release} (Versions 2.0.0-RC1 through 2.0.0-Stable), during which content is reviewed for accuracy, consistency, and clarity. Each phase produces a versioned PDF artifact, and all versions are archived for traceability.

%--------------------------------------------------------------------------
% SECTION 1.2: SCOPE AND PURPOSE
%--------------------------------------------------------------------------
\section{Scope and purpose}
\label{sec:C1-S2}

This guide focuses exclusively on three HOTAS switches: the Target Management Switch (TMS), the Display Management Switch (DMS), and the Countermeasures Management Switch (CMS). Although other controls exist on the F-16 throttle and stick---such as the Communication Switch, the Dogfight/MRM Override, and the RDR Cursor Enable control---they are mentioned only when essential to understanding the behavior and context of TMS, DMS, and CMS.

This is not a comprehensive HOTAS or avionics manual. Instead, it is a usage guide organized by context, with emphasis on practical tables that show what each switch input does in specific flight modes, sensor configurations, and weapon employment scenarios. The guide bridges information scattered across official documentation and training missions, making it immediately accessible to pilots who ask: \emph{``In this radar mode, what does TMS Up do?''} or \emph{``How do I cycle through MFD formats with the DMS?''}

The guide assumes knowledge of basic F-16 operation and familiarity with master modes (NAV, A-A, A-G, DGFT). It does not replace the Dash-34 or Training Manual; rather, it complements them by organizing TMS/DMS/CMS behavior into searchable tables with cross-references back to official sources and practical training missions where each behavior can be practiced.

%--------------------------------------------------------------------------
% SECTION 1.3: VERSION, AUTHORSHIP AND AI ASSISTANCE
%--------------------------------------------------------------------------
\section{Version, authorship and AI assistance}

\label{sec:C1-S3}

\textbf{Document Version:} \fulldocversion{} (Progress: Chapters \chapterscompletedof{} | Tables \tablesfilledpct)

\textbf{Falcon BMS Version:} 4.38.1 (Update 1)

\textbf{Authorship:} This guide was created by a member of the Falcon BMS community with structured assistance from AI language models (Perplexity AI). The human author identified scope, validated content against official Falcon BMS documentation, made all organizational and editorial decisions, and bears full responsibility for the guide's accuracy and presentation. AI tools were used for research organization, cross-referencing, and text generation---not for defining technical correctness.

\textbf{Copyright and License:}

This guide is released under the \textit{Creative Commons Attribution-NonCommercial 4.0 International} (CC BY-NC 4.0) license.

\textbf{You are free to:}

\begin{itemize}
  \item Share, copy, distribute, and print this guide.
  \item Translate into other languages.
  \item Adapt and create derivative works (with attribution).
  \item Create enhanced or corrected versions.
\end{itemize}

\textbf{Restrictions:}

\begin{itemize}
  \item No commercial use. This guide must remain free to all users.
  \item Derivatives must also be CC BY-NC.
  \item Derivatives cannot claim official status; must indicate source and author.
\end{itemize}

\textbf{Attribution Required:} When sharing, translating, or adapting, include the title, author (Carlos ``Metal'' Nader), source repository (\url{https://github.com/carlos-nader/tms-dms-cms-usage-guide}), and this license.

For full legal terms, visit \url{https://creativecommons.org/licenses/by-nc/4.0/}.

\textbf{Disclaimer:} This is an unofficial, community-made document not affiliated with, endorsed by, or affiliated with Benchmark Sims, MicroProse, any military organization, or any aircraft or weapons manufacturer. No copyrighted material is reproduced directly. The guide is provided ``as-is'' for educational and simulation training purposes only.

%--------------------------------------------------------------------------
% SECTION 1.4: SOURCES AND REFERENCES
%--------------------------------------------------------------------------
\section{Sources and references}
\label{sec:C1-S4}

This guide is based on the following primary Falcon BMS documents consulted during research and development:

\begin{enumerate}
  \item \textbf{TO BMS 1F-16CMAM-34-1-1} (Dash-34, Change 4.38) -- Avionics and Nonnuclear Weapons Delivery Flight Manual
  \item \textbf{BMS Training Manual 4.38.1} (October 2025) -- Training missions and learning objectives
  \item \textbf{TO BMS 1F-16CMAM-1} (Dash-1) -- F-16 Aircraft Systems, Normal and Abnormal Procedures
  \item \textbf{BMS User Manual 4.38} -- BMS user interface and setup
  \item \textbf{MCH 11-F16 Vol 5} (May 1996) -- F-16 Flight Manual Vol 5 (Surface Attack and Weapons Delivery)
  \item \textbf{Falcon BMS Cockpit Arrangement Diagrams} (Multiple blocks)
\end{enumerate}

%--------------------------------------------------------------------------
% SECTION 1.5: DOCUMENT STRUCTURE AND HOW TO READ IT
%--------------------------------------------------------------------------
\section{Document structure and how to read it}
\label{sec:C1-S5}

\subsection{Part A: Foundational Sections}
\label{sec:C1-S5-S1}

Foundational sections establish core HOTAS concepts: Sensor of Interest (SOI), short vs. long press timing, master modes, and an overview of TMS/DMS/CMS roles. Read this first if you are new to the F-16 or HOTAS in general.

\subsection{Part B: Switch-Specific Sections}
\label{sec:C1-S5-S2}


Sections focused on TMS, DMS, and CMS each contain detailed tables using the \texttt{hotastable} environment.
\\\\
\textbf{Table structure:}

Each table follows a seven-column format:

\begin{center}
\begin{tabular}{L{1.6cm} L{1.0cm} L{1.0cm} L{3.4cm} L{5.8cm} L{1.4cm} L{1.4cm}}
\toprule
\textbf{State} & \textbf{Dir} & \textbf{Act} & \textbf{Function} & \textbf{Effect / Nuance} & \textbf{Dash34} & \textbf{Train} \\
\midrule
Condition & Up & Shrt & Name & Explanation & Ref & TRN \\
\bottomrule
\end{tabular}
\end{center}

\textbf{How to find information:}

\begin{enumerate}
\item Identify the \textbf{master mode and sensor/weapon context} from the section title (e.g., ``TMS in Air-to-Air -- FCR CRM'').
\item Find the \textbf{State} within the table (e.g., ``Search'' vs. ``STT'').
\item Determine the \textbf{Direction} and \textbf{Action} (Short/Long).
\item Read the \textbf{Effect} and check the \textbf{Dash34} or \textbf{Training} reference.
\end{enumerate}

\subsection{Part C: Training and Visual Reference}
\label{sec:C1-S5-S3}

Training reference section links this guide to the 33 BMS training missions, offering a recommended progression and example tactical flows. Use this section to plan your training sequence.

Visual reference section provides schematic diagrams of the TMS, DMS, and CMS hats with arrows and short labels for each direction in common contexts. These are quick-reference visuals; always consult the tables for complete behavior descriptions.

\subsection{Part D: Appendices}
\label{sec:C1-S5-S4}

Appendices note any differences in TMS/DMS/CMS behavior across F-16 blocks and variants (e.g., Block 50/52 vs. Block 40/42), and provide a comprehensive index of all major tables and their locations.
\newpage
%--------------------------------------------------------------------------
% CHAPTER 2: HOTAS FUNDAMENTALS (PLACEHOLDER)
%--------------------------------------------------------------------------
\chapter{HOTAS fundamentals}
\label{chap:C2}

\section{Sensor of Interest (SOI) and display logic}
\label{sec:C2-S1}
[Content to be developed in next phase]

\section{Short vs long presses and timing}
\label{sec:C2-S2}
[Content to be developed in next phase]

\section{Master modes and context-sensitive behaviour}
\label{sec:C2-S3}
[Content to be developed in next phase]

\section{Overview of TMS, DMS and CMS}
\label{sec:C2-S4}
[Content to be developed in next phase]
\newpage
%--------------------------------------------------------------------------
% CHAPTER 3: TMS (PLACEHOLDER)
%--------------------------------------------------------------------------
\chapter{TMS -- Target Management Switch}
\label{chap:C3}

\section{Concept and general behaviour}
\label{sec:C3-S1}

[Content to be developed in next phase]

\section{TMS and Situational Awareness displays}
\label{sec:C3-S2}

[Content to be developed in next phase]

\section{TMS in Air-to-Air}
\label{sec:C3-S3}

[Content to be developed in next phase]

\section{TMS in Air-to-Ground}
\label{sec:C3-S4}

[Content to be developed in next phase]

\section{TMS in weapon employment}
\label{sec:C3-S5}

[Content to be developed in next phase]

\section{TMS -- Block / variant notes}
\label{sec:C3-S6}
[Content to be developed in next phase]
\newpage
%--------------------------------------------------------------------------
% CHAPTER 4: DMS
%--------------------------------------------------------------------------
\chapter{DMS -- Display Management Switch}
\label{chap:C4}

%--------------------------------------------------------------------------
% CHAPTER 4: SECTION 4.1
%--------------------------------------------------------------------------
\section{Concept and Sensor of Interest (SOI)}
\label{sec:C4-S1}

The Display Management Switch (DMS) is a four-direction spring-loaded hat located on the flight stick. Its primary role is to manage which display or sensor receives hands-on control inputs, known as the Sensor of Interest (SOI), and to cycle through the Multifunction Display Set (MFDS) formats.

Unlike the Target Management Switch (TMS), which performs tactical functions such as target designation and data management (see Chapter~\ref{chap:C3}), the DMS is a transversal SOI manager. It does not designate targets or change radar modes directly; instead, it selects \textit{which display or sensor} the pilot is currently controlling with other HOTAS inputs (such as CURSOR/ENABLE or TMS).

As presented in Chapter~\ref{chap:C3}, condensed diagrams for throttle and flight stick switches functionalities can be found on \dashref{2.1.5}. Below is an image of the F-16 Flight Stick, with the DMS switch location.

\begin{figure}[H]
\centering
\includegraphics[width=0.65\textwidth]{F-16_Side_Stick_Controller-1.jpg}
\caption{F-16 Throttle and Flight Stick. Image by Falconpedia (\url{falcon4.wikidot.com}), via Wikimedia Commons (\url{https://commons.wikimedia.org/wiki/File:F-16_Side_Stick_Controller.jpg}), licensed under the Creative Commons Attribution-Share Alike 3.0 Unported (CC BY-SA 3.0) license.}
\label{fig:f16_hotas_dms_location}
\end{figure}

\paragraph*{DMS Across F-16 Blocks and Variants:}

The functionality of the DMS --- SOI selection, MFD format cycling and all associated behaviors --- is identical across all F-16 blocks and variants available in Falcon BMS. Differences in aircraft avionics do not alter DMS switch usage. For this reason, all DMS procedures in this chapter apply universally to the entire F-16 family.

%--------------------------------------------------------------------------
% CHAPTER 4: SECTION 4.1.1
%--------------------------------------------------------------------------
\subsection{SOI Definition and Scope Across Displays}
\label{sec:C4-S1-S1}

The Sensor of Interest (SOI) is the display or sensor that currently receives HOTAS cursor slew commands and, where applicable, TMS actions. At any moment, only one display can be the SOI. Valid SOI displays include any of the F-16's displays: the HUD, the right MFD, and the left MFD. However, some MFD formats are not valid SOI because they provide information or control but do not accept sensor-like slew or targeting inputs. The table below summarizes valid and invalid MFD formats.

\begin{longtable}{C{4.5cm} C{4.5cm}}
\caption{Valid and Invalid SOI MFD Format \label{tab:C4-S1-MFD-format}}\\

\rowcolor{headerblue}
\textbf{\color{white}Valid} & \textbf{\color{white}Invalid}\\
\midrule
\endfirsthead

\rowcolor{headerblue}
\textbf{\color{white}Valid} & \textbf{\color{white}Invalid}\\
\midrule
\endhead

\midrule
\multicolumn{2}{r}{\emph{Continues on next page}}\\
\endfoot

\bottomrule
\endlastfoot

Fire Control Radar (FCR); Targeting Pod (TGP); Horizontal Situation Display (HSD); HARM Pages; Weapon (WPN) & Stores Management (SMS); Data Transfer Equipment (DTE); TEST; Blank/Inactive formats; Digital Flight Control System (FLCS); TACAN (TCN); Forward Looking Infrared (FLIR); Terrain Following Radar (TFR)
\end{longtable}

Visually, The currently active SOI is easily recognized:
\begin{itemize}
    \item On the \textbf{HUD}: an asterisk (\texttt{*}) appears in the upper left corner when HUD is SOI.
    \item On an \textbf{MFD}: a border outline appears around the edges of the display when it is SOI. When an MFD format is \textit{not} SOI, the text \texttt{NOT SOI} may appear on the format.
\end{itemize}

%--------------------------------------------------------------------------
%CHAPTER 4: SECTION 4.1.1.1
%--------------------------------------------------------------------------
\subsubsection{Valid SOI Displays by Master Mode}
\label{sec:C4-S1-S1-S1}

The availability of displays as valid SOI varies by master mode. Table~\ref{tab:C4-S1-SOI-by-mode} shows which displays can serve as SOI in the primary operational contexts. Note that in air-to-air empolyment modes (A-A, DGFT, MSL OVRD), the HUD is \textbf{never} available as SOI; A-A modes restrict the pilot to left and right MFD (FCR, HSD, or TGP formats) as the SOI. This constraint ensures that radar and tactical displays remain the primary source of truth in air-to-air engagements.

\small
\renewcommand{\arraystretch}{1.2}

\begin{longtable}{L{2.5cm} L{6.5cm} L{6.0cm}}
\caption{Valid SOI Displays by Master Mode\label{tab:C4-S1-SOI-by-mode}}\\

\rowcolor{headerblue}
\textbf{\color{white}Master Mode} & \textbf{\color{white}Valid SOI Displays} & \textbf{\color{white}Constraints \& Notes} \\
\midrule
\endfirsthead

\rowcolor{headerblue}
\textbf{\color{white}Master Mode} & \textbf{\color{white}Valid SOI Displays} & \textbf{\color{white}Constraints \& Notes} \\
\midrule
\endhead

\midrule
\multicolumn{3}{r}{\emph{Continues on next page}}\\
\endfoot

\bottomrule
\endlastfoot

NAV (Navigation) & HUD and MFD (FCR, TGP, HSD, WPN, HARM Pages formats) & All displays available. HUD is primary choice for situational awareness and NAV-specific tasks. \\
\midrule
A-A (Air-to-Air) & MDF only (FCR, HSD, TGP formats) & \textbf{HUD cannot be SOI.} SOI limited to radar and tactical displays. \\
\midrule
A-G (PRE) & HUD and MFD (FCR, TGP, WPN, HARM Pages, HSD formtas) & All displays available. \\
\midrule
A-G (VIS) & HUD and MFD (FCR, TGP, WPN formats) & Restricted to visual-capable displays. HUD used for target acquisition (e.g., AGM-65 VIS EO, DTOS, CCIP). \\
\midrule
DGFT (Dogfight) & MFD (FCR, HSD, TGP formats) & \textbf{HUD cannot be SOI}. \\
\midrule
MSL OVRD (Missile Override) & MFD (FCR, HSD, TGP formats) & \textbf{HUD cannot be SOI}. \\
\end{longtable}

As shown in Table~\ref{tab:C4-S1-SOI-by-mode}, the availability of SOI displays is strategically constrained by master mode to align pilot attention with the operational context. Navigation and air-to-ground modes offer maximum display flexibility, allowing the pilot to transition between HUD and MFD with radar, targeting pod, and weapon pages formats as needed.

Conversely, air-to-air and associated override modes eliminate HUD as a selectable SOI. This design reflects the fundamental principle that air-to-air engagements must be driven by primary sensor information rather than HUD-derived or symbology-based cues. However, HMCS provides an independent off-boresight capability (see Section~\ref{sec:C4-S1-S3} for details), independent of HUD not being a valid SOI.

For visual air-to-ground delivery (VIS modes), practical SOI control --- HUD in conjunction with MFD --- focuses on visual-capable sensors: the HUD for target designation and the MFD for optical tracking (TGP page) and weapon-specific control (WPN page), ofr instance. Although the radar and optical trackings may continue to provide ranging data in the background, HUD and sensor-video-driven symbology drive the attack in VIS delivery. See Section~\ref{sec:C4-S2-S1} for specific weapon examples (CCIP, DTOS, AGM-65 VIS EO, IAM VIS).

%--------------------------------------------------------------------------
% CHAPTER 4: SECTION 4.1.2
%--------------------------------------------------------------------------
\subsection{Role of the DMS in SOI Selection}
\label{sec:C4-S1-S2}

The DMS manages SOI selection through two orthogonal axes of control:

\begin{itemize}
    \item \textbf{Vertical (Up / Down):} Selects \textit{which display} is SOI.
        \begin{itemize}
            \item \textbf{DMS Up:} Transfers SOI to the HUD (when permitted by master mode), detailed in Section~\ref{sec:C4-S2}.
            \item \textbf{DMS Down:} Cycles SOI between MFD, or from HUD to an MFD, detailed in Section~\ref{sec:C4-S3}.
        \end{itemize}
    \item \textbf{Horizontal (Left / Right):} Steps through MFD formats on the left or right MFD, independently of which display is the SOI (detailed in Section~\ref{sec:C4-S4}).
        \begin{itemize}
            \item \textbf{DMS Left:} Cycles the left MFD format (primary $\to$ secondary $\to$ tertiary).
            \item \textbf{DMS Right:} Cycles the right MFD format (primary $\to$ secondary $\to$ tertiary).
        \end{itemize}
\end{itemize}

%--------------------------------------------------------------------------
% CHAPTER 4: SECTION 4.1.3
%--------------------------------------------------------------------------
\subsection{HUD as SOI in A-A and HMCS Capabilities}
\label{sec:C4-S1-S3}

The restriction that HUD cannot be designated as SOI in A-A master mode applies exclusively to the \textbf{SOI routing architecture} — the mechanism by which HOTAS inputs (cursor slew, TMS commands) are delivered to a specific display. This architectural constraint \textbf{does not eliminate the functional capability} of the HUD or related displays to acquire, track, or cue targets in A-A operations.

The \textbf{Helmet Mounted Cueing System (HMCS)} exemplifies this distinction. Although HMCS is an extension of the HUD display system and shares the HUD's architectural restriction in A-A mode (neither can be designated as SOI via DMS), HMCS retains \textbf{independent off-boresight targeting capability}. For example, the pilot can slave an AIM-9 seeker to the HMCS visor line-of-sight without regard to which display (FCR, HSD, or TGP) is currently the SOI, and can employ HMCS-derived target acquisition cues that function independently of SOI status.

This principle reflects the fundamental architectural design: the SOI mechanism manages \textbf{HOTAS input routing}, while \textbf{display functional capabilities operate orthogonally}. Displays designated as SOI in A-A (FCR, HSD, TGP) receive these HOTAS inputs; displays not designated as SOI (HUD, HMCS) provide cueing and targeting functions through independent mechanisms (e.g., helmet line-of-sight, derived sensor information). Both types of functionality are essential to A-A operations, despite the SOI designation limitation.

For further technical details on HMCS capabilities and behavior in A-A contexts, refer to \dashref{2.5} (Helmet Mounted Cueing System).

%--------------------------------------------------------------------------
% CHAPTER 4: SECTION 4.2
%--------------------------------------------------------------------------
\section{DMS Up: HUD Designation as SOI}
\label{sec:C4-S2}

The \textbf{DMS Up} command attempts to designate the HUD as the Sensor of Interest (SOI). When the current master mode permits the HUD to be SOI (see Table~\ref{tab:C4-S1-SOI-by-mode} in Section~\ref{sec:C4-S1}), a short press of DMS Up immediately transfers SOI to the HUD.

When DMS Up successfully designates the HUD as SOI, the HUD SOI asterisk appears and any previous MFD SOI border is removed, as described in Section~\ref{sec:C4-S1-S1}. From that moment, all SOI-dependent HOTAS inputs (such as CURSOR/ENABLE for symbology positioning and TMS for target or waypoint designation) act on HUD symbology rather than on any MFD format. This simple visual feedback --- the HUD asterisk and loss of MFD SOI borders --- allows the pilot to confirm at a glance that all SOI-dependent commands are now applied to HUD-level cueing rather than to an MFD sensor page.

%--------------------------------------------------------------------------
% CHAPTER 4: SECTION 4.2.1
%--------------------------------------------------------------------------
\subsection{DMS Up Effectiveness in All Master Modes}
\label{sec:C4-S2-S1}

DMS Up is only effective in master modes where the HUD is a valid SOI candidate. Table~\ref{tab:C4-S1-SOI-by-mode} in Section~\ref{sec:C4-S1} summarises these constraints. This subsection focuses on the modes where HUD SOI is both permitted and operationally significant, and then contrasts them with modes where DMS Up has no effect.

%--------------------------------------------------------------------------
% CHAPTER 4: SECTION 4.2.1.1
%--------------------------------------------------------------------------
\subsubsection{Master Modes Where DMS Up is Effective (HUD as SOI Permitted)}
\label{sec:C4-S2-S1-S1}

In master modes where the HUD can be SOI, DMS Up is the hands‑on command used to designate the HUD as SOI, enabling HUD‑based visual cueing in those modes.

\paragraph*{NAV (Navigation) Master Mode:}
In Navigation mode, the HUD is the primary reference for flight path, steering, and basic situational awareness. A short press of DMS Up immediately designates the HUD as SOI. With HUD as SOI, the pilot can use SOI‑dependent HOTAS inputs to interact with navigation‑related symbology on the HUD or HMCS: CURSOR/ENABLE slews the HUD/HMCS cursor or designator, and in specific functions such as HUD or HMCS MARK, TMS Up is used to stabilise the line of sight and create markpoints, while the MFD continue to provide background information.

The exact set of displays that may become SOI in NAV, and how they compare to the HUD, is documented in Table~\ref{tab:C4-S1-SOI-by-mode}; DMS Up simply selects the HUD within that set.

\paragraph*{A-G in VIS (Visual Air-to-Ground)---CCIP, DTOS, AGM-65 VIS, IAM-VIS:}
In air-to-ground visual delivery modes, targets are identified and designated visually by the pilot. The HUD becomes the \textbf{primary command interface} for visual cueing, and HUD as SOI is often a prerequisite for correct TMS and CURSOR behaviour. DMS Up is therefore \textbf{operationally critical}: whenever SOI has migrated to an MFD (for example, to TGP or WPN), a short press of DMS Up restores HUD as SOI and returns HOTAS inputs to the HUD.

HUD as SOI in typical A-G VIS contexts:

\begin{itemize}
  \item \textbf{CCIP visual deliveries:} The HUD displays a pipper (computed impact point or bullet track line). The pilot maneuvers the aircraft to place the pipper on the intended impact point and commands weapon release with the weapon release button. When the fire control system allows, CURSOR/ENABLE inputs referenced to HUD SOI can be used to refine the visual aimpoint or adjust reference cues without leaving the HUD-centric view.
  \item \textbf{AGM-65 Maverick VIS:} In AGM-65 VIS, the HUD shows a target designator (TD) box that slaves the Maverick seeker. With HUD as SOI (via DMS Up), CURSOR/ENABLE slews the TD box over the intended target, and TMS Up commands seeker lock. If SOI is inadvertently left on an MFD (for example, the WPN page), TMS inputs are routed to that display instead of to the HUD TD box, and visual target rejection or re-designation through the HUD will not work as intended until DMS Up restores HUD SOI.
  \item \textbf{IAM (JSOW/JDAM) visual deliveries (IAM-VIS):} In IAM-VIS, the HUD presents a TD box and associated A-G solution cues for visual designation. With HUD as SOI, the pilot refines the TD box position by aircraft manoeuvre and, when appropriate, by CURSOR/ENABLE inputs. TMS Up then designates and ground-stabilises the target. If SOI is on an MFD, these TMS commands act on the MFD sensor page instead, and the HUD cueing will not update as expected until HUD SOI is re-established with DMS Up.
\end{itemize}

DMS Up is also valid in non‑visual A‑G modes (such as CCRP or preplanned IAM deliveries), but in those cases HUD SOI is a convenience rather than a strict requirement, since targeting, cursor management, sighting‑point control, and sensor designation can be accomplished entirely with MFD‑centric SOI (FCR, TGP, HSD). The visual modes described earlier (DTOS, AGM‑65 VIS, HUD/HMCS MARK, IAM‑VIS) are where the coupling between DMS Up and TMS/CURSOR behaviour on HUD/HMCS symbology becomes operationally critical, as these modes rely on HUD/HMCS line‑of‑sight cueing and ground‑stabilization as primary designation methods.

%--------------------------------------------------------------------------
% CHAPTER 4: SECTION 4.2.1.2
%--------------------------------------------------------------------------
\subsubsection{Modes Where DMS Up is Ineffective (HUD as SOI Prohibited)}
\label{sec:C4-S2-S1-S2}

In air-to-air employment modes (A-A, DGFT, and MSL OVRD), pressing DMS Up has no effect on SOI selection, because the HUD is not a valid SOI candidate in these modes (see Section~\ref{sec:C4-S1-S1} and Table~\ref{tab:C4-S1-SOI-by-mode}). SOI remains on one of the MFD formats, such as the FCR, HSD, or TGP. The architectural rationale for this restriction, and the complementary role of HMCS in providing high off-boresight cueing in air-to-air, are developed in Section~\ref{sec:C4-S1-S3}.

%--------------------------------------------------------------------------
% CHAPTER 4: SECTION 4.2.2
%--------------------------------------------------------------------------
\subsection{DMS Up Usage Table}
\label{sec:C4-S2-S2}

The table below summarises DMS Up behaviour across representative master modes. It should be read together with Table~\ref{tab:C4-S1-SOI-by-mode}, which documents which displays are valid SOI candidates in each mode.

Each table entry specifies:

\begin{itemize}
  \item \textbf{State}: The operational context (master mode, CMDS mode, sensor state).
  \item \textbf{Direction}: The physical direction for pressing the CMS hat (Up, Down, Left, Right).
  \item \textbf{Action}: The press type (Short, Long, Long Hold).
  \item \textbf{Function}: What the CMS command activates or controls.
  \item \textbf{Effect / Nuance}: The resulting system behavior, including tactics and constraints.
  \item \textbf{Dash34}: Reference section in the Dash-34 manual.
  \item \textbf{Training}: Recommended BMS training missions for hands-on practice.
\end{itemize}

\begin{hotastable}{DMS Up Usage Across NAV, A-A and A-G Master Modes}
NAV & Up & Short & Designate HUD as SOI & DMS Up is fully effective in NAV master mode. Pressing DMS Up immediately designates the HUD as SOI, placing the SOI asterisk on the HUD. With HUD/HMCS as SOI, CURSOR/ENABLE slews the HUD/HMCS cursor or designator, and in functions such as HUD or HMCS MARK, TMS Up is used to stabilise the line of sight and create markpoints. MFDs remain available for background navigation and systems information. & 2.1.1.2.3, 2.1.7.5.1, 2.1.7.5.4, 2.5.6.1 & \\\\
A-A & Up & Short & Designate HUD as SOI & DMS Up is \textbf{ineffective} in A-A master mode. The avionics architecture restricts SOI to FCR, HSD, or TGP only. HUD cannot be SOI in this mode and functions purely as a passive display. & 2.1.1.2.3 & \\
A-G & Up & Short & Designate HUD as SOI & DMS Up is fully effective in A-G master modes. Pressing DMS Up immediately designates HUD as SOI, and an asterisk appears in the upper left corner of the HUD. \textbf{In A-G visual modes (VIS)}, HUD is the operationally critical command interface for visual target designation and rejection via CURSOR and TMS inputs. If HUD loses SOI, visual cueing control is lost and must be recovered with DMS Up. & 2.1.1.2.3, 4.2.2.1, 4.2.2.1.1 & \trnref{10 (GP Bombs)}, \trnref{11 (LGB)}, \trnref{13 (Maverick)}, \trnref{14 (Maverick Adv)}, \trnref{15 (IAM)} \\
\end{hotastable}
\label{tab:C4-S2-DMS-Up-Usage}

%--------------------------------------------------------------------------
% CHAPTER 4: SECTION 4.2.3
%--------------------------------------------------------------------------
\subsection{DMS Up Exception States}
\label{sec:C4-S2-S3}

In certain states, DMS Up may be temporarily ineffective even in modes where HUD is normally a valid SOI:

\begin{itemize}
  \item \textbf{Snowplow (SP) PRE state (unstabilised):} When the pilot enters Snowplow mode (a specialised ground-stabilisation mode for slewing to arbitrary ground positions) and the SP position has not yet been stabilised with TMS Up, the SOI is effectively ``nowhere''. Both the A-G radar and TGP display \texttt{NOT SOI} on the MFDs, and DMS Up/Down commands are ineffective until the SP position is stabilised. Once stabilised, SOI returns to its previous state, and DMS Up becomes effective again (\dashref{4.2.1.4}).
  \item \textbf{MARK/OFLY Submode:} In the MARK/OFLY submode (a specialised target-acquisition context documented in \dashref{2.1.1.2.3}), the SOI cannot be designated at all. As a result, DMS inputs that would normally change the SOI have no effect in this state. This exception is rare in normal operations.
\end{itemize}

%--------------------------------------------------------------------------
% CHAPTER 4: SECTION 4.3
%--------------------------------------------------------------------------
\section{DMS Down: Toggle SOI Between Displays}
\label{sec:C4-S3}

DMS Down toggles the Sensor of Interest SOI among the displays available in the current master mode. As established in Section~\ref{sec:C4-S1-S1}, the set of valid SOI displays varies by mode: in NAV and A-G modes, the HUD is available; in air-to-air employment modes (A-A, DGFT, MSL OVRD), it is not.

Consequently, DMS Down behaves in two distinct ways:

\begin{itemize}
  \item In \textbf{NAV} and \textbf{A-G} modes, DMS Down toggles SOI through all available displays: HUD $\rightarrow$ L/R MFD $\rightarrow$ L/R MFD $\rightarrow$ HUD.
  \item In \textbf{air-to-air employment modes} (A-A, DGFT, MSL OVRD), DMS Down toggles SOI only between the two MFD (L/R MFD $\leftrightarrow$ L/R MFD), since the HUD is not a valid SOI candidate
\end{itemize}

This design ensures that DMS Up and DMS Down work together to manage SOI across all available displays in each operational context. It is \textbf{important to note} that \textbf{DMS Down} transitions SOI between displays—HUD and the two MFD—without changing which format is currently displayed on any MFD. The pilot executes hands-on commands on whatever format is available at the selected display. Format transitions \textbf{within an MFD} are controlled by DMS Right and DMS Left, covered in Section~\ref{sec:C4-S4}.

% CHAPTER 4: SECTION 4.3.1
\subsection{DMS Down Effectiveness in All Master Modes}
\label{sec:C4-S3-S1}
DMS Down effectiveness depends on which displays can serve as SOI in the current master 
mode, as established in Section~\ref{sec:C4-S1-S1}.

% CHAPTER 4: SECTION 4.3.1.1
\subsubsection{Master Modes Where HUD is a Valid SOI Candidate}
\label{sec:C4-S3-S1-S1}

\paragraph{NAV (Navigation) Master Mode:}
In NAV, DMS Down toggles SOI through all valid candidates: HUD and both MFD. 
Repeated DMS Down presses create a continuous 3-step sequence: HUD 
$\rightarrow$ L/R MFD $\rightarrow$ L/R MFD $\rightarrow$ HUD. This allows the pilot to quickly move hands-on command focus between the HUD and the two MFD sensor displays for navigation and sensor management.

\paragraph{A-G in PRE (Preplanned Air-to-Ground) Mode:}
In A-G PRE, valid SOI candidates are the HUD and both MFD. DMS Down follows the 
same 3-step toggle pattern as NAV, moving SOI among the HUD and the two MFD. This allows the pilot to shift hands-on control focus while examining different sensor pages.

\paragraph{A-G in VIS (Visual Air-to-Ground)---CCIP, DTOS, AGM-65 VIS, IAM-VIS:}
In A-G VIS modes, valid SOI candidates are the HUD and both MFD. DMS Down toggles through the same 3-step pattern as in NAV. However, in A-G VIS, DMS Down becomes \textbf{operationally critical} rather than merely convenient.

A-G VIS delivery is fundamentally HUD-centric: the pilot acquires and designates the target visually using the HUD pipper (CCIP) or target designator box (AGM-65 VIS, IAM-VIS). These visual cues are controlled by CURSOR and TMS inputs, which are routed to whichever display is currently SOI. If SOI migrates to an MFD---such as TGP for sensor refinement, WPN for weapon status, or FCR/HSD for situational awareness---those same CURSOR and TMS commands will act on the MFD instead of the HUD, and visual designation on the HUD ceases to respond.

Therefore, \textbf{DMS Down and DMS Up} work in tandem in A-G VIS: DMS Down allows the pilot to temporarily move SOI to an MFD for sensor work or information review, while DMS Up immediately restores HUD SOI to resume visual designation. This up-down alternation is fundamental to efficient A-G VIS delivery and cannot be omitted without degrading command flow or situational awareness.

% CHAPTER 4: SECTION 4.3.1.2
\subsubsection{Modes Where HUD is NOT a Valid SOI Candidate}
\label{sec:C4-S3-S1-S2}

In air-to-air employment (A-A, DGFT, and MSL OVRD) modes, the avionics architecture restricts SOI to the MFD only. The HUD cannot be designated as SOI 
in these modes (see Sections~\ref{sec:C4-S1-S1} and \ref{sec:C4-S1-S3}). Consequently, DMS Down is limited to toggling SOI between the two MFD (L/R MFD $\leftrightarrow$ L/R MFD). This 2-way toggle allows the pilot to select which MFD sensor display receives hands-on command priority.

In A-A contexts, this is \textbf{operationally essential}: the pilot uses DMS Down to shift SOI focus between onde MFD and the ohter so he can access and have direct control over whichever format is being actually displayed: FCR for track management and missile employment, HSD for tactical picture and threat assessment or between the FCR and TGP for situational awareness or supplemental tracking. Efficient air-to-air engagement depends critically on rapid SOI management via DMS Down.

% CHAPTER 4: SECTION 4.3.2
\subsection{DMS Down Usage Table}
\label{sec:C4-S3-S2}

\begin{hotastable}{DMS Down Usage Across NAV, A-A, and A-G Master Modes}
  NAV & Down & Short & Toggle SOI cycle through displays & 
    DMS Down toggles SOI through HUD $\rightarrow$ L/R MFD $\rightarrow$ L/R MFD $\rightarrow$ HUD. With HUD as SOI, hands-on commands (CURSOR/ENABLE, TMS) manage HUD navigation symbology. Pressing DMS Down transfers SOI to the next display; the pilot can rotate through all three displays sequentially. &
    --- \\
  
  A-A & Down & Short & Toggle SOI between MFD only & 
    DMS Down toggles SOI only between the two MFD. The HUD cannot be SOI in A-A and remains a passive display. This is the primary HOTAS method for selecting which MFD sensor page receives hands-on command priority for track management, situational awareness, and weapons employment. & 
    2.1.1.2.3, 2.1.6.3 & 
    \trnref{18 BARCAP}, \trnref{17B IFF Intercept} \\
  
  A-G & Down & Short & Toggle SOI between HUD and MFD sensor pages & 
    DMS Down toggles SOI through HUD $\rightarrow$ L/R MFD $\rightarrow$ L/R MFD $\rightarrow$ HUD. In A-G PRE, DMS Down is a convenience tool for shifting hands-on focus between HUD and MFD sensor pages. \textbf{In A-G VIS (CCIP, DTOS, AGM-65 VIS, IAM-VIS), DMS Down is operationally critical:} the HUD is the primary visual designation interface. DMS Down allows the pilot to alternate between HUD visual cueing (TMS/CURSOR steering, pipper control, TD box positioning) and MFD sensor work (TGP search/refine, WPN status, FCR A-G ranging). Loss of HUD as SOI in A-G VIS prevents proper visual designation and must be recovered with DMS Up. & 
    2.1.1.2.3, 2.1.6.3 & 
    \trnref{10 GP Bombs}, \trnref{11 LGB}, \trnref{13 Maverick}, \trnref{14 Maverick Adv}, \trnref{15 IAM} \\
\end{hotastable}

% CHAPTER 4: SECTION 4.3.3
\subsection{DMS Down Exception States}
\label{sec:C4-S3-S3}

In certain special states and submodes, DMS Down may be temporarily ineffective, as a direct reflection of DMS Up (see Section~\ref{sec:C4-S2-S3}).

\begin{itemize}
  \item \textbf{Snowplow (SP) PRE State (not stabilised):} When the pilot enters Snowplow mode (a specialised A-G ground-stabilisation mode for slewing to arbitrary ground positions) and the SP position has not yet been stabilised with TMS Up, the SOI is effectively ``nowhere.'' Both the A-G radar and TGP MFD displays show \texttt{NOT SOI}, and neither display is designated as SOI. As a result, \textbf{DMS Down has no effect} in this state: the toggle mechanism has nowhere to advance SOI to. Once the SP position is stabilised with TMS Up (pressing TMS Up on the HUD), SOI returns to its previous designated display, and DMS Down resumes normal toggling behaviour.

  \item \textbf{MARK/OFLY Submode:} In the MARK/OFLY submode (a specialised target-acquisition context documented in \dashref{2.1.1.2.3}), the SOI cannot be designated or changed at all. Consequently, \textbf{DMS Down has no effect} in MARK/OFLY: you cannot toggle SOI when SOI designation itself is locked. This submode is rare in normal operations but is important to recognise if you encounter it during unusual procedures or system states.
\end{itemize}

%--------------------------------------------------------------------------
% CHAPTER 4: SECTION 4.4
%--------------------------------------------------------------------------
\section{DMS Left/Right}
\label{sec:C4-S4}

%--------------------------------------------------------------------------
% CHAPTER 5: CMS
%--------------------------------------------------------------------------
\chapter{CMS -- Countermeasures Management Switch}
\label{chap:C5}

%--------------------------------------------------------------------------
% CHAPTER 5: SECTION 5.1
%--------------------------------------------------------------------------
\section{Concept and Interaction with CMDS / ECM / RWR}\label{sec:C5-S1}

%--------------------------------------------------------------------------
% CHAPTER 5: SECTION 5.1.1
%--------------------------------------------------------------------------
\subsection{Concept}
\label{sec:C5-S1-S1}

The Countermeasures Management Switch (CMS) is a four-direction hat switch mounted on the control stick that serves as the pilot's primary control interface to the F-16's integrated electronic warfare (EW) defensive systems: the ALE-47 CMDS (automatic chaff/flare dispenser), ECM systems (external pods or internal avionics), and avionics-based threat defeat systems. The CMS supervises the aircraft's defensive response by controlling defensive program selection, managing ECM operational modes, and granting or withholding consent authority to all defensive subsystems.

Its role is to grant the pilot rapid tactical control over the aircraft's defensive posture. This control is operationally critical because defensive decisions frequently occur during high-G maneuvering when hand position cannot be redirected to distant cockpit panels. A pilot executing a 6-G defensive turn cannot simultaneously reach the CMDS MODE knob on the left console or the ECM control panel without abandoning aircraft control. By placing the CMS within thumb reach during full-stick maneuver, the design ensures that no tactical scenario regardless of G-load or workload forces the pilot to choose between aircraft control and defensive system authority.

RWR, although not directly linked to CMS, is more than a display device: it is the decision engine for both CMDS and ECM. The RWR continuously evaluates detected threat radars, classifies them (SEARCH, TRACK, LAUNCH), assigns threat priority, and communicates this information to both the ALE-47 CMDS (in AUTO or SEMI mode) and the ECM system (for band selection or jamming initiation).

For in-depth explanations about CMDS, ECM and RWR operation, see \dashrefs{2.7.1, 2.7.2, and 2.7.3}, respectively. This section focuses exclusively on CMS usage and control interface. As presented in the preceding chapters, condensed diagrams of ECM and other throttle and flight stick functionalities can be found on section \dashrefs{2.1.5}. Below is an image of the F-16 Flight Stick, with the CMS switch location.

\begin{figure}[H]
\centering
\includegraphics[width=0.65\textwidth]{F-16_Side_Stick_Controller-1.jpg}
\caption{F-16 Throttle and Flight Stick. Image by Falconpedia (\url{falcon4.wikidot.com}), via Wikimedia Commons (\url{https://commons.wikimedia.org/wiki/File:F-16_Side_Stick_Controller.jpg}), licensed under the Creative Commons Attribution-Share Alike 3.0 Unported (CC BY-SA 3.0) license.}
\label{fig:f16_hotas_cms_location}
\end{figure}

%--------------------------------------------------------------------------
% CHAPTER 5: SECTION 5.1.2
%--------------------------------------------------------------------------
\subsection{Interaction with CMDS / ECM}
\label{sec:C5-S1-S2}

Operationally, the CMS manages two distinct defensive layers: \textbf{CMDS} and \textbf{ECM} in both configurations: internal avionics and external ECM pod. The way this management is performed and all the corresponding button pressings will be detailed in the next Section~\ref{sec:C4-S2}.

Differences in F-16 Blocks or variants, especially regarding ECM, will be discussed in Section~\ref{sec:C5-S3}.

\begin{itemize}
  \item \textbf{ECM (External Pod):} Controls the external ECM pod's operational state through pilot-directed transmission modes and consent authority.
  \item \textbf{ECM (Integrated IDIAS):} Controls the integrated ECM system through automatic threat-reactive modes.
  \item \textbf{CMDS in Manual Mode:} Allows the pilot to execute pre-selected dispenser programs on demand, independent of automatic systems.
  \item \textbf{CMDS in Automatic/Semi-Automatic Modes:} Authorizes the ALE-47 CMDS to respond autonomously to RWR-detected threats when operating in AUTO or SEMI mode.
\end{itemize}

%--------------------------------------------------------------------------
% CHAPTER: 5 SECTION 5.2
%--------------------------------------------------------------------------
\section{CMS Switch Actuation}
\label{sec:C5-S2}

The Countermeasures Management Switch (CMS) is a four-direction hat switch located at the flight stick that provides pilots with rapid, direct control over the F-16's defensive systems during demanding tactical situations. This section tabulates all CMS button combinations, organized by operational layer: Counter Measures (CMDS manual/automatic/semi modes) and Defensive Avionics integration (jamming) with both external ECM pods (ALQ-131/ALQ-184) and internal avionics (IDIAS). CMS actuation is independent of the Master Mode currently selected.

Each table entry specifies:

\begin{itemize}
  \item \textbf{State}: The operational context (master mode, CMDS mode, sensor state).
  \item \textbf{Direction}: The physical direction for pressing the CMS hat (Up, Down, Left, Right).
  \item \textbf{Action}: The press type (Short, Long, Long Hold).
  \item \textbf{Function}: What the CMS command activates or controls.
  \item \textbf{Effect / Nuance}: The resulting system behavior, including tactics and constraints.
  \item \textbf{Dash34}: Reference section in the Dash-34 manual.
  \item \textbf{Training}: Recommended BMS training missions for hands-on practice.
\end{itemize}

% 5.2.1 CMS Actuation with CMDS
\subsection{CMS Actuation with CMDS}
\label{sec:C5-S2-S1}

The ALE-47 CMDS (Automatic Chaff and Flare Dispensing System) provides three operational modes: Manual (MAN), Automatic (AUTO), and Semi-Automatic (SEMI). Each mode grants the pilot different levels of control and autonomy over chaff and flare dispensing. The CMS is the primary interface for program execution and consent authority across all three modes.

% 5.2.1.1 Manual Mode
\subsubsection{Manual Mode}
\label{sec:C5-S2-S1-S1}

The CMDS Manual (MAN) mode grants the pilot direct, program-by-program control over countermeasure expenditure. Each CMS direction selects or executes a specific program. This mode is recommended when threat types are well-known or when chaff/flare inventory must be conserved, or by pilot choice.

\begin{hotastable}{CMS Actuation with CMDS Manual Mode}
CMDS MAN & Up & Short & Execute Program 1--4 &
Runs the program selected via the CMDS panel PRGM knob once per press. No threat sensing; purely pilot-commanded. Overrides any AUTO dispensing, if running.
& \dashref{2.7.2.2} & \trnref{18 (BARCAP)}, \trnref{28 (SEAD-EW)} \\
CMDS MAN & Left & Short & Execute Program 6 &
Flare-only program. Often pre-configured for close-range air-to-air engagements or MANPAD defense. No dependency on PRGM knob.
& \dashref{2.7.2.2} & \trnref{19 (GUN AIM)} \\
\end{hotastable}

% 5.2.1.2 Automatic Mode
\subsubsection{Automatic Mode}
\label{sec:C5-S2-S1-S2}

The CMDS Automatic (AUTO) mode enables the ALE-47 CMDS to dispense chaff/flare programs continuously in response to RWR-detected threats, without requiring pilot consent for each event. Pilot consent is given once by pressing CMS Down; dispensing continues until CMS Right is pressed or expendables are exhausted.

\begin{hotastable}{CMS Actuation with CMDS Automatic Mode}
CMDS AUTO & Up & Short & Execute Program 1--4 &
Manual override: runs the selected program once, interrupting any ongoing AUTO dispensing. After the manual program completes, AUTO resumes if threat persists. Useful for pilot override in high-threat scenarios.
& \dashref{2.7.2.2} & \trnref{18 (BARCAP)}, \trnref{28 (SEAD-EW)} \\
CMDS AUTO & Left & Short & Execute Program 6 &
Manual flare-only program, overrides AUTO. After execution, AUTO resumes.
& \dashref{2.7.2.2} & \trnref{19 (GUN AIM)} \\
CMDS AUTO & Down & Short & Give Consent; Enable AUTO Dispensing &
CMS Down grants consent for AUTO CMDS. RWR-detected threats trigger automatic program dispensing (selected via PRGM knob). Dispensing continues until threat clears or CMS Right is pressed. Consent state persists even if pilot switches to MAN mode; re-engaging AUTO will resume auto-dispense.
& \dashref{2.7.2.1} & \trnref{18 (BARCAP)}, \trnref{28 (SEAD-EW)} \\
CMDS AUTO & Right & Short & Cancel Consent; Disable AUTO &
CMS Right removes CMDS consent and places the ALE-47 in Standby. Automatic dispensing halts immediately. Pilot must re-issue CMS Down to resume AUTO operation or use the system manually.
& \dashref{2.7.2.1} & \trnref{18 (BARCAP)}, \trnref{28 (SEAD-EW)} \\
\end{hotastable}

% 5.2.1.3 Semi-Automatic Mode
\subsubsection{Semi-Automatic Mode}
\label{sec:C5-S2-S1-S3}

Semi-Automatic (SEMI) mode allows the ALE-47 to prompt the pilot for consent on a per-threat basis. When the RWR detects a threat requiring countermeasures, the CMDS displays ``DISPENSE RDY'' on the control unit and sounds a ``COUNTER'' voice message, requesting pilot consent via CMS Down.

\begin{hotastable}{CMS Actuation with CMDS Semi-Automatic Mode}
CMDS SEMI & Up & Short & Execute Program 1--4 &
Manual override: runs the selected program once, independent of RWR threat state. After execution, CMDS returns to monitoring for threats and issuing ``COUNTER'' prompts.
& \dashref{2.7.2.2} & \trnref{18 (BARCAP)}, \trnref{28 (SEAD-EW)} \\
CMDS SEMI & Left & Short & Execute Program 6 &
Manual flare-only program, independent of SEMI threat detection. After execution, CMDS resumes SEMI monitoring.
& \dashref{2.7.2.2} & \trnref{19 (GUN AIM)} \\
CMDS SEMI & Down & Short & Give Consent; Dispense One Program &
When CMDS issues ``COUNTER'' (threat detected), pilot presses CMS Down to execute one instance of the selected program. If threat persists or a new threat appears, CMDS will issue ``COUNTER'' again. Consent state is tracked; switching to AUTO while consent active will trigger immediate AUTO dispensing on next threat.
& \dashref{2.7.2.1} & \trnref{18 (BARCAP)}, \trnref{28 (SEAD-EW)} \\
CMDS SEMI & Right & Short & Cancel Consent; Return to Standby &
CMS Right removes SEMI consent. CMDS halts monitoring and returns to Standby. ``COUNTER'' messages cease.
& \dashref{2.7.2.1} & \trnref{18 (BARCAP)}, \trnref{28 (SEAD-EW)} \\
\end{hotastable}

% 5.2.2 CMS Actuation with ECM
\subsection{CMS Actuation with ECM}
\label{sec:C5-S2-S2}

External ECM pods (ALQ-131, ALQ-184) and internal IDIAS (Improved Defensive Internal Avionic System) provide pilot-controlled jamming across frequency bands. The CMS Down position controls transmit authority for external pods; CMS Left cycles modes for internal IDIAS. Both systems interact with the RF switch and respect landing gear constraints.

% 5.2.2.1 External ECM Pod
\subsubsection{External ECM Pod (ALQ-131 / ALQ-184)}
\label{sec:C5-S2-S2-S1}

External ECM pods provide pilot-controlled jamming across five frequency-band programs. The CMS Down position grants ``ECM consent,'' enabling the pod to transmit at the XMIT switch setting (modes 1, 2, or 3). The ECM Enable light on the miscellaneous panel indicates consent state.

\begin{hotastable}{CMS Actuation with External ECM Pod}
ECM Pod & Down & Short & Enable ECM Transmit; Grant Consent &
CMS Down illuminates the ECM Enable light and permits the external ECM pod to transmit in the mode set by the XMIT switch on the ECM control panel (XMIT 1: AUTO Avionics Priority; XMIT 2: AUTO ECM Priority; XMIT 3: Continuous Jam). Pod continues transmitting as long as ECM is not cancelled by the pilot or RF switch is moved away from NORM.
& \dashref{2.7.4.2.5} & \trnref{28 (SEAD-EW)} \\
ECM Pod & Right & Short & Disable ECM Transmit; Remove Consent &
CMS Right extinguishes the ECM Enable light and places the ECM pod in Standby, halting transmission immediately. Pod will not transmit until CMS Down is re-issued.
& \dashref{2.7.4.2.5} & \trnref{28 (SEAD-EW)} \\
\end{hotastable}

% 5.2.2.2 Internal ECM (IDIAS)
\subsubsection{ECM (IDIAS)}
\label{sec:C5-S2-S2-S2}

Internal avionics ECM (IDIAS: Improved Defensive Internal Avionic System) automatically selects frequency bands to jam based on RWR threat priority. CMS Left cycles through operational modes (Standby, Avionics Priority, ECM Priority). CMS Down is not used for IDIAS; mode control is via CMS Left and the XMTR button on the ECM panel.

\begin{hotastable}{CMS Actuation with Internal ECM (IDIAS)}
IDIAS ECM & Left & Short & Cycle ECM Operational Mode &
Each short press of CMS Left advances the ECM mode: STBY $\rightarrow$ AVNC (Avionics Priority) $\rightarrow$ ECM (ECM Priority) $\rightarrow$ AVNC $\rightarrow$ ECM (cycles).
& \dashref{2.7.4.1.2} & \trnref{28 (SEAD-EW)} \\
IDIAS ECM & Right & Short & Set ECM to Standby &
CMS Right forces IDIAS ECM into STBY mode, halting all jamming operations. Requires CMS Left cycling to return to AVNC or ECM mode.
& \dashref{2.7.4.1.2} & \trnref{28 (SEAD-EW)} \\
\end{hotastable}

% 5.2.3 CMS Consent and Constraints
\subsection{CMS Consent and Constraints}
\label{sec:C5-S2-S3}

This subsection clarifies the relationship between CMDS consent (CMS Down) and ECM transmit authority (CMS Down for external pod). Understanding these interactions is critical for effective defensive posture management during high-workload combat operations.

\begin{hotastable}{CMS Interaction with CMDS and ECM (Consent and Constraints)}
CMDS AUTO/SEMI + ECM Pod & Down & Short & Joint Consent (CMDS + ECM) &
Single CMS Down command grants consent to \textit{both} the ALE-47 CMDS (AUTO/SEMI) and the external ECM pod. There's no distinction between the two; both systems respond to the same CMS Down press. This unified control maximizes pilot situational awareness and frees workload during combat maneuvering.
& \dashref{2.7.2.1}, \dashref{2.7.4.2.5} & \trnref{18 (BARCAP)}, \trnref{28 (SEAD-EW)} \\
\end{hotastable}

% 5.2.4 Important Operational Notes
\subsection{Important Operational Notes}
\label{sec:C5-S2-S4}

The CMS provides rapid, tactile access to CMDS program selection and ECM transmit authority without requiring the pilot to manipulate distant panels during high-G maneuvering. Mastery of CMS actuation across all CMDS modes (MAN, SEMI, AUTO) and ECM configurations (external pod, IDIAS) is essential for effective defensive operations. Pilots must understand the consent state model, RF switch interactions, and inventory management to avoid unintended dispensing or system saturation.

% 5.2.4.1
\subsubsection{State Tracking}
\label{sec:C5-S2-S4-S1}

In AUTO and SEMI modes, the CMDS tracks the consent state even if the pilot temporarily switches to MAN mode. If the pilot gives CMS Down consent in AUTO, then switches the CMDS MODE knob to MAN, the consent state is retained. Upon re-engaging AUTO without issuing CMS Down again, the CMDS will immediately begin dispensing if a threat is detected. This behavior can be exploited for rapid mode switching during combat but may also lead to unintended dispensing if not carefully managed.

% 5.2.4.2
\subsubsection{Bingo Quantity Behavior}
\label{sec:C5-S2-S4-S2}

If expendables (chaff or flare) fall to or below the bingo quantity, the CMDS will still request consent (CMS Down) and continue dispensing. The ``LOW'' and ``OUT'' voice messages alert the pilot to low or exhausted inventory, but dispensing does not automatically stop. Pilot must monitor EWS upfront pages and manually manage inventory via CMDS MAN or by pressing CMS Right to inhibit AUTO.

% 5.2.4.3
\subsubsection{RF Switch Override}
\label{sec:C5-S2-S4-S3}

The RF switch on the throttle is a master control for ECM transmission. Moving the RF switch away from NORM (e.g., to QUIET or SILENT) overrides any previous CMS Down command and places both the external ECM pod and internal IDIAS in Standby. Returning RF to NORM does \textit{not} automatically restore transmission; the pilot must re-issue CMS Down.

% 5.2.4.4
\subsubsection{Consent vs. CMDS Consent}
\label{sec:C5-S2-S4-S4}

A single CMS Down press grants consent to both the ALE-47 CMDS and the external ECM pod. The pilot does not issue separate commands; the CMS Down action is unified. However, internal IDIAS uses CMS Left for mode cycling, not CMS Down. This distinction is critical for aircraft configured with IDIAS.

% 5.2.4.5
\subsubsection{Operations Safety}
\label{sec:C5-S2-S4-S5}

On the ground, ECM pods are held in Standby for safety reasons. Pilots must not hold CMS Down while on the ground in the vicinity of personnel, as the ECM pod may radiate and pose a hazard. Ground personnel must be clear before the pilot engages ECM for pre-flight high-level BIT (Built-In Test). Once airborne, ECM consent (CMS Down) can be issued and maintained as tactically required.

%--------------------------------------------------------------------------
% SECTION 5.3: CMS BLOCK / VARIANT NOTES
%--------------------------------------------------------------------------
\section{CMS Block and Variant Notes}
\label{sec:C5-S3}

Section~\ref{sec:C5-S2} defines CMS actuation procedures for CMDS and ECM systems. 
CMS interaction with \textbf{CMDS is uniform across all F-16 blocks and variants} (see Section~\ref{sec:C5-S2-S1}). However, \textbf{ECM configuration varies significantly by block and operator}, resulting in different CMS procedures: external ECM pods (ALQ-131/ALQ-184) use CMS Down as the transmit control, while internal IDIAS systems use CMS Left for mode cycling. These operational differences extend to panel controls (XMIT knob on external pods vs XMTR switch on IDIAS) and fundamentally change the pilot's CMS gesture sequence.

The present Section identifies which F-16 variants use which ECM configuration and maps them to the correct procedure section in Section~\ref{sec:C5-S2-S2}. Before flight, pilots must verify their aircraft's ECM configuration to ensure they apply the correct CMS procedures and avoid dangerous habit transfer between external ECM and IDIAS variants.

% Section 5.3.1 
\subsection{ECM Configurations present in BMS}
\label{sec:C5-S3-S1}

Falcon BMS 4.38.1 implements two distinct ECM system architectures, each with different CMS actuation methods. The following paragraphs describe how the CMS interacts with each configuration and bring table depicting the blocks/variants equipped with each system architecture; detailed actuation procedures are provided in Section~\ref{sec:C5-S2}.

% Section 5.3.1.1
\subsubsection{ECM Pods (ALQ-131 / ALQ-184)}
\label{sec:C5-S3-S1-S1}

These variants are equipped with an ECM Panel that employs manual jamming band selection. \textbf{CMS Down} provides ECM transmit consent.\\\\
The XMIT knob (3-position switch: 1, 2, 3) selects the jamming mode: XMIT 1 (Avionics Priority, AFT antenna only), XMIT 2 (ECM Priority, both FWD+AFT antennas), or XMIT 3 (Active Jam, continuous transmission independent of RWR threats). For detailed procedures, see Section~\ref{sec:C5-S2-S2} and \dashrefs{2.7.4.1.1 and 2.7.4.2.5}.

\begin{table}[h]
\centering
\caption{External ECM Pod Blocks/Variants}
\label{tab:C5-S3-external-ecm-pods}
\small
\renewcommand{\arraystretch}{1.25}
\begin{tabular}{|>{\raggedright\arraybackslash}p{3.0cm}|>{\raggedright\arraybackslash}p{5.8cm}|}
\hline
\rowcolor{headerblue}
\textcolor{white}{\textbf{Operator}} & \textcolor{white}{\textbf{Block/Variant}} \\
\hline
USAF & Blocks 40/42/50/52 (CM designation) \\
\hline
NATO & Block 15 operators (Belgium, Denmark, Netherlands, Norway) \\
\hline
International & Egypt, Korea KF-16C Block 32 \\
\hline
\end{tabular}
\end{table}

% Section 5.3.1.2
\subsubsection{ECM (IDIAS)}
\label{sec:C5-S3-S1-S2}

These variants use the IDIAS ECM Panel. \textbf{CMS Left} to cycles all operational modes.\\\\
The XMTR switch (2-position toggle: STBY, OPER) enables the ECM system; when in OPER, the mode (AVNC, ECM or STBY) is selected via CMS Left and determines ECM behavior. For detailed procedures, see Section~\ref{sec:C5-S2-S2} and \dashrefs{2.7.4.1.2 and 2.7.4.2.6}.
\begin{table}[h]
\centering
\caption{Internal ECM Blocks/Variants}
\label{tab:C5-S3-internal-ecm}
\small
\renewcommand{\arraystretch}{1.25}
\begin{tabular}{|>{\raggedright\arraybackslash}p{3.0cm}|>{\raggedright\arraybackslash}p{5.8cm}|}
\hline
\rowcolor{headerblue}
\textcolor{white}{\textbf{Operator}} & \textcolor{white}{\textbf{Block/Variant}} \\
\hline
Israel & F-16I Sufa Block 52, Barak I Block 30, Barak II Block 40 \\
\hline
International & Greek HAF (Blocks 50 PXII, 52 PXIII, 52+ PXIV Advanced), Korea KF-16C Block 52, Singapore F-16D Block 52 \\
\hline
\end{tabular}
\end{table}

% Section 5.3.1.3
\subsubsection{Clarification}
\label{sec:C5-S3-S1-S3}

The variants listed above represent variants available or not in Falcon BMS 4.38.1 but certailny don't reflect complete real-world inventories.
\newpage
%--------------------------------------------------------------------------
% CHAPTER 6: TRAINING REFERENCES (PLACEHOLDER)
%--------------------------------------------------------------------------
\chapter{Training references and practical flows}
\label{chap:C6}

\section{How to use this guide with BMS training missions}
\label{sec:C6-S1}

[Content to be developed in next phase]

\section{Recommended progression}
\label{sec:C6-S2}

[Content to be developed in next phase]

\section{Example flows for typical missions}
\label{sec:C6-S3}

[Content to be developed in next phase]
\newpage
%--------------------------------------------------------------------------
% CHAPTER 7: VISUAL REFERENCE (PLACEHOLDER)
%--------------------------------------------------------------------------
\chapter{HOTAS visual reference}
\label{chap:C7}

\section{F-16 HOTAS overview}
\label{sec:C7-S1}

[Content to be developed in next phase]

\section{TMS diagrams}
\label{sec:C7-S2}

[Content to be developed in next phase]

\section{DMS diagrams}
\label{sec:C7-S3}

[Content to be developed in next phase]

\section{CMS diagrams}
\label{sec:C7-S4}

[Content to be developed in next phase]

%--------------------------------------------------------------------------
% APPENDICES
%--------------------------------------------------------------------------
\appendix

\section{Block / variant overview}

\subsection{F-16CM Block 50/52}

[Content to be developed in next phase]

\subsection{F-16C/D Block 40/42}

[Content to be developed in next phase]

\subsection{F-16AM/BM MLU}

[Content to be developed in next phase]

\subsection{F-16I Sufa and Israeli variants}

[Content to be developed in next phase]

\subsection{Other export variants}

[Content to be developed in next phase]

\section{Tables index}

\subsection{TMS tables}

[To be populated]

\subsection{DMS tables}

[To be populated]

\subsection{CMS tables}

[To be populated]

\end{document}
