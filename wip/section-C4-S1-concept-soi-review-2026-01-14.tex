% ============================================================================
% FALCON BMS TMS/DMS/CMS HOTAS GUIDE
% WIP FILE TEMPLATE V1.0 — FINAL (Briefing v0.2.0.1 + TOC Fix)
% ============================================================================

% IMPORTANTE: Este é um TEMPLATE padronizado para arquivos WIP (Work-In-Progress)
% que serão integrados ao guide.tex.

% Nomenclatura: Siga wip-naming-v1.3
% Padrão: section-C{N}-S{M}[-S{K}]-{titulo}-{status}-{data}.tex
% Exemplo: section-C5-S2-cms-actuation-hotas-tables-final-2026-01-10.tex

% Status: review | review | final | approved | deprecated
% Locação: WIP/ (ativo) | ARCHIVE/ (aprovado/descartado)

\documentclass[11pt,a4paper]{article}

% --------------------------------------------------------------------------
% BASIC ENCODING AND LANGUAGE
% --------------------------------------------------------------------------

\usepackage[utf8]{inputenc}
\usepackage[T1]{fontenc}
\usepackage[english]{babel}

% --------------------------------------------------------------------------
% FONTS AND MICROTYPOGRAPHY
% --------------------------------------------------------------------------

\usepackage{lmodern}
\usepackage{microtype}
\usepackage{soul}

% --------------------------------------------------------------------------
% PAGE GEOMETRY AND LAYOUT
% --------------------------------------------------------------------------

\usepackage{geometry}
\geometry{a4paper, left=2.0cm, right=2.0cm, top=2.5cm, bottom=2.5cm}
\usepackage{setspace}
\onehalfspacing

% --------------------------------------------------------------------------
% COLORS AND LINKS
% --------------------------------------------------------------------------

\usepackage[table]{xcolor}
\definecolor{linkblue}{HTML}{004488}
\definecolor{linkred}{HTML}{882222}
\definecolor{headerblue}{HTML}{003366}
\definecolor{rowgray}{HTML}{F5F5F5}
\definecolor{subheadgray}{HTML}{E0E0E0}
\usepackage[pdfencoding=auto, psdextra, colorlinks=true, linkcolor=linkblue, citecolor=linkred, urlcolor=linkblue, breaklinks=true]{hyperref}
\usepackage{bookmark}

% --------------------------------------------------------------------------
% HEADERS AND FOOTERS
% --------------------------------------------------------------------------

\usepackage{fancyhdr}
\setlength{\headheight}{15pt}
\pagestyle{fancy}
\fancyhf{}
\fancyhead[L]{\leftmark}
\fancyhead[R]{\rightmark}
\fancyfoot[C]{\thepage}
\renewcommand{\headrulewidth}{0.4pt}
\renewcommand{\footrulewidth}{0pt}

% --------------------------------------------------------------------------
% TABLES AND MACROS
% CORRECTD: \end commands repositioned --------------------------------------------------------------------------

\usepackage{booktabs}
\usepackage{array}
\usepackage{longtable}
\usepackage{tabularx}

% Custom Columns
\newcolumntype{L}[1]{>{\raggedright\arraybackslash}p{#1}}
\newcolumntype{C}[1]{>{\centering\arraybackslash}p{#1}}
\newcolumntype{R}[1]{>{\raggedleft\arraybackslash}p{#1}}

% HOTAS table environment (Briefing v0.2.0.1, Section 6)
\newenvironment{hotastable}[1]{%
  \small
  \renewcommand{\arraystretch}{1.25}
  \begin{longtable}{L{1.6cm} L{1.0cm} L{1.0cm} L{3.4cm} L{5.8cm} L{1.4cm} L{1.4cm}}
  \caption{#1}\\
  \rowcolor{headerblue}
  \textbf{\color{white}State} &
  \textbf{\color{white}Dir} &
  \textbf{\color{white}Act} &
  \textbf{\color{white}Function} &
  \textbf{\color{white}Effect / Nuance} &
  \textbf{\color{white}Dash34} &
  \textbf{\color{white}Train} \\
  \endfirsthead
  %
  \rowcolor{headerblue}
  \textbf{\color{white}State} &
  \textbf{\color{white}Dir} &
  \textbf{\color{white}Act} &
  \textbf{\color{white}Function} &
  \textbf{\color{white}Effect / Nuance} &
  \textbf{\color{white}Dash34} &
  \textbf{\color{white}Train} \\
  \endhead
  %
  \multicolumn{7}{r}{\small\emph{Continued on next page}}\\
  \endfoot
  %
  \endlastfoot
}{%
  \end{longtable}
}

% --------------------------------------------------------------------------
% SIMPLE REFERENCE MACROS FOR BMS DOCS
% --------------------------------------------------------------------------

\providecommand{\dashref}[1]{Dash-34~\S~#1}
\providecommand{\dashone}[1]{Dash-1~\S~#1}
\providecommand{\trnref}[1]{TRN~#1}
\providecommand{\trnman}{BMS Training Manual 4.38.1}
\providecommand{\bmsver}{Falcon BMS~4.38.1}
\providecommand{\dashrefs}[1]{\textit{TO 1F-16CMAM-34-1-1}, Dash-34, sections \texttt{#1}}

% --------------------------------------------------------------------------
% VERSION CONTROL MACROS
% --------------------------------------------------------------------------

\newcommand{\docversion}{C4-S1}
\newcommand{\docbuild}{20260112}
\newcommand{\fulldocversion}{\docversion+\docbuild}

% --------------------------------------------------------------------------
% GRAPHICS
% --------------------------------------------------------------------------

\usepackage{graphicx}
\graphicspath{{fig/}}

% --------------------------------------------------------------------------
% TITLE
% --------------------------------------------------------------------------

\title{TMS, DMS and CMS Usage Guide for \bmsver}
\author{Carlos ``Metal'' Nader}
\date{Version \fulldocversion{0.2.4.0} | January 2026}

% --------------------------------------------------------------------------
% DOCUMENT BEGIN
% --------------------------------------------------------------------------

\begin{document}

\maketitle

\newpage

\tableofcontents

\newpage

% ============================================================================

% WIP FILE METADATA (NOT RENDERED IN PDF)

% ============================================================================

% File Name: section-C4-S1-concept-and-soi-dev-2026-01-12.tex

% WIP Naming Convention: v1.4

% Target Chapter: C4 (DMS — Display Management Switch)

% Target Section: S1 (Concept and Sensor of Interest)

% Target Subsection: S1.1, S1.2, S1.3

%

% WIP Status: dev

% Created: 2026-01-12

% Last Modified: 2026-01-12

% Integration Status: TARGET v0.3.0.0

%

% Narrative Completion: 100% - needs human review

% Table Fill Status: 0% (tables belong to Section 4.3)

%

% Notes:

% - Approved content from Carlos ``Metal'' Nader (2026-01-12)

% - Sections 4.1.1, 4.1.2 complete per briefing-v0.2.0.1 Section 3.1

% - No tables in this section; content is purely narrative

% - Cross-reference: Chapter 3 (TMS), Chapter 5 (CMS)

% ============================================================================

% ============================================================================

% APPROVED CONTENT — DMS Introductory Narrative (Section 4.1)

% ============================================================================

\section{DMS — Display Management Switch}

\label{sec:C4}

\subsection{Concept and Sensor of Interest (SOI)}

\label{sec:C4-S1}

The Display Management Switch (DMS) is a four-direction spring-loaded hat located on the flight stick. Its primary role is to manage which display or sensor receives hands-on control inputs, known as the Sensor of Interest (SOI), and to cycle through the Multifunction Display Set (MFDS) format pages.

Unlike the Target Management Switch (TMS), which performs tactical functions such as target designation and data management (see Chapter 3), the DMS is a transversal SOI architecture manager. It does not designate targets or change radar modes directly; instead, it selects \textit{which display or sensor} the pilot is currently controlling with other HOTAS inputs (such as CURSOR/ENABLE or TMS).

As presented in Chapter 3, condensed diagrams for throttle and flight stick switches functionalities can be found on \dashref{2.1.5}. Below is an image of the F-16 Flight Stick, with the DMS switch location.

\begin{figure}[h]
\centering
\includegraphics[width=0.65\textwidth]{F-16_Side_Stick_Controller-1.jpg}
\caption{F-16 Throttle and Flight Stick. Image by Falconpedia (\url{falcon4.wikidot.com}), via Wikimedia Commons (\url{https://commons.wikimedia.org/wiki/File:F-16_Side_Stick_Controller.jpg}), licensed under the Creative Commons Attribution-Share Alike 3.0 Unported (CC BY-SA 3.0) license.}
\label{fig:f16_hotas_dms_location}
\end{figure}

\paragraph*{DMS Across F-16 Blocks and Variants:}

The functionality of the DMS --- SOI selection, MFD format cycling and all associated behaviors --- is identical across all F-16 blocks and variants available in Falcon BMS. Differences in aircraft avionics do not alter DMS switch usage. For this reason, all DMS procedures in this chapter apply universally to the entire F-16 family.

\subsubsection{SOI Definition and Scope Across Displays}

\label{sec:C4-S1-S1}

The Sensor of Interest (SOI) is the display or sensor that currently receives HOTAS cursor slew commands and, where applicable, TMS actions. At any moment, only one display can be the SOI. Valid SOI displays include:

\begin{itemize}
    \item \textbf{Fire Control Radar (FCR)};
    \item \textbf{Targeting Pod (TGP)};
    \item \textbf{Horizontal Situation Display (HSD)};
    \item \textbf{HARM Attack Display (HAD)};
    \item \textbf{Weapon page (WPN)};
    \item \textbf{Heads-Up Display (HUD)}.
\end{itemize}

Displays that are \textit{not} valid SOI include: SMS (Stores Management Set), DTE (Data Transfer Equipment), TEST, and blank/inactive MFDS formats. These pages provide information or control but do not accept sensor-like slew or targeting inputs.

The SOI is indicated visually:
\begin{itemize}
    \item On the \textbf{HUD}: an asterisk (\texttt{*}) appears in the upper left corner when HUD is SOI.
    \item On an \textbf{MFD}: a border outline appears around the edges of the display when it is SOI. When an MFD format is \textit{not} SOI, the text ``NOT SOI'' may appear on the format (depending on the mode).
\end{itemize}

\paragraph{Valid SOI Displays by Master Mode:}
The availability of displays as valid SOI varies by master mode. Table~\ref{tab:C4-S1-SOI-by-mode} shows which displays can serve as SOI in the primary operational contexts. Note that in air-to-air modes (A-A, DGFT, MSL OVRD), the HUD is \textbf{never} available as SOI; A-A modes restrict the pilot to FCR, HSD, or TGP as the SOI. This constraint ensures that radar and tactical displays remain the primary source of truth in air-to-air engagements.

\small
\renewcommand{\arraystretch}{1.2}

\begin{longtable}{L{2.5cm} L{6.5cm} L{6.0cm}}
\caption{Valid SOI Displays by Master Mode\label{tab:C4-S1-SOI-by-mode}}\\

\rowcolor{headerblue}
\textbf{\color{white}Master Mode} & \textbf{\color{white}Valid SOI Displays} & \textbf{\color{white}Constraints \& Notes} \\
\midrule
\endfirsthead

\rowcolor{headerblue}
\textbf{\color{white}Master Mode} & \textbf{\color{white}Valid SOI Displays} & \textbf{\color{white}Constraints \& Notes} \\
\midrule
\endhead

\midrule
\multicolumn{3}{r}{\emph{Continues on next page}}\\
\endfoot

\bottomrule
\endlastfoot

NAV (Navigation) & HUD, FCR, TGP, HSD, WPN, HAD & All displays available. HUD is primary choice for situational awareness and NAV-specific tasks. \\
\midrule
A-A (Air-to-Air) & FCR, HSD, TGP & \textbf{HUD cannot be SOI.} SOI limited to radar and tactical displays. \\
\midrule
A-G (PRE) & HUD, FCR, TGP, WPN, HAD, HSD & All displays available. \\
\midrule
A-G (VIS) & HUD, FCR, TGP, WPN & Restricted to visual-capable displays. HUD used for target acquisition (e.g., AGM-65 VIS EO, DTOS, CCIP). \\
\midrule
DGFT (Dogfight) & FCR, HSD, TGP & \textbf{HUD cannot be SOI}. \\
\midrule
MSL OVRD (Missile Override) & FCR, HSD, TGP & \textbf{HUD cannot be SOI}. \\
\end{longtable}

As shown in Table~\ref{tab:C4-S1-SOI-by-mode}, the availability of SOI displays is strategically constrained by master mode to align pilot attention with the operational context. Navigation and air-to-ground modes offer maximum display flexibility, allowing the pilot to transition between HUD, radar, targeting pod, and weapon pages as needed.

Conversely, air-to-air and associated override modes eliminate HUD as a selectable SOI. This design reflects the fundamental principle that air-to-air engagements must be driven by primary sensor information (FCR radar returns) rather than HUD-derived or symbology-based cues. However, HMCS provides an independent off-boresight capability (see Section~\ref{sec:C4-S1-S3} for details).

For visual air-to-ground delivery (VIS modes), practical SOI control focuses on visual-capable sensors: the HUD for target designation, the TGP for optical tracking when available, and the WPN page for weapon-specific control. Although the FCR may continue to provide ranging data in the background, HUD- and sensor-video-driven symbology drive the attack in VIS delivery. See Section~\ref{sec:C4-S2-S1} for specific weapon examples (CCIP, DTOS, AGM-65 VIS EO, IAM VIS).

\subsubsection{Role of the DMS in SOI Selection}

\label{sec:C4-S1-S2}

The DMS manages SOI selection through two orthogonal axes of control:

\begin{itemize}
    \item \textbf{Vertical (Up / Down):} Selects \textit{which display} is SOI.
        \begin{itemize}
            \item \textbf{DMS Up:} Transfers SOI to the HUD (when permitted by master mode).
            \item \textbf{DMS Down:} Cycles SOI between MFDs, or from HUD to an MFD.
        \end{itemize}
    \item \textbf{Horizontal (Left / Right):} Steps through MFDS format pages on the left or right MFD, independently of which display is the SOI.
        \begin{itemize}
            \item \textbf{DMS Left:} Cycles the left MFD format (primary $\to$ secondary $\to$ tertiary).
            \item \textbf{DMS Right:} Cycles the right MFD format (primary $\to$ secondary $\to$ tertiary).
        \end{itemize}
\end{itemize}

\subsubsection{HUD as SOI in A-A and HMCS Capabilities}

\label{sec:C4-S1-S3}

The restriction that HUD cannot be designated as SOI in A-A master mode applies exclusively to the \textbf{SOI routing architecture} — the mechanism by which HOTAS inputs (cursor slew, TMS commands) are delivered to a specific display. This architectural constraint \textbf{does not eliminate the functional capability} of the HUD or related displays to acquire, track, or cue targets in A-A operations.

The \textbf{Helmet Mounted Cueing System (HMCS)} exemplifies this distinction. Although HMCS is an extension of the HUD display system and shares the HUD's architectural restriction in A-A mode (neither can be designated as SOI via DMS), HMCS retains \textbf{independent off-boresight targeting capability}. For example, the pilot can slave an AIM-9 seeker to the HMCS visor line-of-sight without regard to which display (FCR, HSD, or TGP) is currently the SOI, and can employ HMCS-derived target acquisition cues that function independently of SOI status.

This principle reflects the fundamental architectural design: the SOI mechanism manages \textbf{HOTAS input routing}, while \textbf{display functional capabilities operate orthogonally}. Displays designated as SOI in A-A (FCR, HSD, TGP) receive these HOTAS inputs; displays not designated as SOI (HUD, HMCS) provide cueing and targeting functions through independent mechanisms (e.g., helmet line-of-sight, derived sensor information). Both types of functionality are essential to A-A operations, despite the SOI designation limitation.

For further technical details on HMCS capabilities and behavior in A-A contexts, refer to \dashref{2.5} (Helmet Mounted Cueing System).

% ============================================================================

% END OF SECTION 4.1

% ============================================================================

\end{document}
