% ============================================================================
% FALCON BMS TMS/DMS/CMS HOTAS GUIDE
% WIP FILE — SECTION-LEVEL
% ============================================================================

% Nomenclatura (WIP-NAMING-v1.4):
% section-C5-S3-blocks-variants-review-2026-01-10.tex
% ├─ section: type identifier
% ├─ C5: target chapter 5 (CMS)
% ├─ S3: target section 3 (CMS Block and Variant Notes)
% ├─ blocks-variants: descriptive title (slugified)
% ├─ final: status (dev | review | final | approved | deprecated)
% └─ 2026-01-10: creation/last-edit date (YYYY-MM-DD)

\documentclass[11pt,a4paper]{article}

% --------------------------------------------------------------------------
% BASIC ENCODING AND LANGUAGE
% --------------------------------------------------------------------------

\usepackage[utf8]{inputenc}
\usepackage[T1]{fontenc}
\usepackage[english]{babel}

% --------------------------------------------------------------------------
% FONTS AND MICROTYPOGRAPHY
% --------------------------------------------------------------------------

\usepackage{lmodern}
\usepackage{microtype}

% --------------------------------------------------------------------------
% PAGE GEOMETRY AND LAYOUT
% --------------------------------------------------------------------------

\usepackage{geometry}
\geometry{a4paper, left=2.0cm, right=2.0cm, top=2.5cm, bottom=2.5cm}
\usepackage{setspace}
\onehalfspacing

% --------------------------------------------------------------------------
% COLORS AND LINKS
% --------------------------------------------------------------------------

\usepackage[table]{xcolor}
\definecolor{linkblue}{HTML}{004488}
\definecolor{linkred}{HTML}{882222}
\definecolor{headerblue}{HTML}{003366}
\definecolor{rowgray}{HTML}{F5F5F5}
\definecolor{subheadgray}{HTML}{E0E0E0}
\usepackage[pdfencoding=auto, psdextra, colorlinks=true, linkcolor=linkblue, citecolor=linkred, urlcolor=linkblue, breaklinks=true]{hyperref}
\usepackage{bookmark}

% --------------------------------------------------------------------------
% HEADERS AND FOOTERS
% --------------------------------------------------------------------------

\usepackage{fancyhdr}
\setlength{\headheight}{15pt}
\pagestyle{fancy}
\fancyhf{}
\fancyhead[L]{\leftmark}
\fancyhead[R]{\rightmark}
\fancyfoot[C]{\thepage}
\renewcommand{\headrulewidth}{0.4pt}
\renewcommand{\footrulewidth}{0pt}

% --------------------------------------------------------------------------
% TABLES AND MACROS
% --------------------------------------------------------------------------

\usepackage{booktabs}
\usepackage{array}
\usepackage{longtable}
\usepackage{tabularx}

% Custom Columns
\newcolumntype{L}[1]{>{\raggedright\arraybackslash}p{#1}}
\newcolumntype{C}[1]{>{\centering\arraybackslash}p{#1}}
\newcolumntype{R}[1]{>{\raggedleft\arraybackslash}p{#1}}

% HOTAS table environment (per BRIEFING v0.2.0.1, Section 6)
% CORRECTED: [1] instead of [1][] to force caption requirement
% CORRECTED: \end commands order
\newenvironment{hotastable}[1]{%
  \small
  \renewcommand{\arraystretch}{1.25}
  \begin{longtable}{L{1.6cm} L{1.0cm} L{1.0cm} L{3.4cm} L{5.8cm} L{1.4cm} L{1.4cm}}
  \caption{#1}\\
  \rowcolor{headerblue}
  \textbf{\color{white}State} &
  \textbf{\color{white}Dir} &
  \textbf{\color{white}Act} &
  \textbf{\color{white}Function} &
  \textbf{\color{white}Effect / Nuance} &
  \textbf{\color{white}Dash34} &
  \textbf{\color{white}Train} \\
  \endhead                    % ← AQUI PRIMEIRO (ORDEM B — ERRADA)
  
  \rowcolor{headerblue}
  \textbf{\color{white}State} &
  \textbf{\color{white}Dir} &
  \textbf{\color{white}Act} &
  \textbf{\color{white}Function} &
  \textbf{\color{white}Effect / Nuance} &
  \textbf{\color{white}Dash34} &
  \textbf{\color{white}Train} \\
  \endfirsthead               % ← DEPOIS (ORDEM B — ERRADA)
  
  \multicolumn{7}{r}{\small\emph{Continued on next page}}\\
  \endfoot
  
  \endlastfoot
}{%
  \end{longtable}
}

% --------------------------------------------------------------------------
% SIMPLE REFERENCE MACROS FOR BMS DOCS
% --------------------------------------------------------------------------
% CORRECTED: Changed from \newcommand to \providecommand for integration compatibility
% CORRECTED: Formatted with \S symbol per Briefing v0.2.0.1, Section 11.3

\providecommand{\dashref}[1]{Dash-34~\S~#1}
\providecommand{\dashone}[1]{Dash-1~\S~#1}
\providecommand{\trnref}[1]{TRN~#1}
\providecommand{\trnman}{BMS Training Manual 4.38.1}
\providecommand{\bmsver}{Falcon BMS~4.38.1}
\providecommand{\dashrefs}[1]{\textit{TO 1F-16CMAM-34-1-1}, Dash-34, sections \texttt{#1}}

% --------------------------------------------------------------------------
% VERSION CONTROL MACROS
% --------------------------------------------------------------------------

\newcommand{\docversion}{WIP}
\newcommand{\docbuild}{section-C5-S3-final}
\newcommand{\fulldocversion}{\docversion~\docbuild}

% --------------------------------------------------------------------------
% GRAPHICS
% --------------------------------------------------------------------------

\usepackage{graphicx}
\graphicspath{{fig/}}

% --------------------------------------------------------------------------
% TITLE (For Standalone Compilation)
% --------------------------------------------------------------------------

\title{F-16 Countermeasures Management Switch (CMS)\\[0.2cm]Section 5.3: CMS Block and Variant Notes\\[0.2cm]{\small \bmsver{}}}
\author{Carlos ``Metal'' Nader}
\date{WIP Status: review | Date: 2026-01-10}

% --------------------------------------------------------------------------
% DOCUMENT BEGIN
% --------------------------------------------------------------------------

\begin{document}

\maketitle

\pagenumbering{roman}

\newpage

\tableofcontents

\newpage

\pagenumbering{arabic}

% ============================================================================
% WIP FILE METADATA (NOT RENDERED IN PDF)
% ============================================================================

% File Name: section-C5-S3-blocks-variants-final-2026-01-10.tex
% WIP Naming Convention: v1.4 (CORRECTED from v1.3)
% Target Chapter: C5 (Countermeasures Management System — CMS)
% Target Section: S3 (CMS Block and Variant Notes)
% WIP Status: final (ready for integration into guide.tex)
% Created: 2026-01-07
% Last Modified: 2026-01-10
%
% Narrative Completion: 100%
% Table Fill Status: 90% (cross-reference table complete; no HOTAS tables in this section)
%
% Compatibility Status:
% ✅ C1: Changed \documentclass{report} → {article}
% ✅ C2: Changed hotastable [1][] → [1]
% ✅ C3: Changed \newcommand → \providecommand (macros)
% ✅ TOC: Added \newpage before AND after \tableofcontents (article class fix)
% ✅ Naming: Updated from v1.3 to v1.4 nomenclature (section-C5-S3-blocks-variants-final-2026-01-10)
%
% Integration notes:
% - Ready for integration into guide.tex
% - Delete preamble (\documentclass through \pagenumbering{arabic}) during integration
% - Keep content from \section{CMS Block and Variant Notes} forward
% - All three corrections + TOC fix + naming update applied
%
% Cross-references:
% - Briefing: v0.2.0.1, Section 6 (Column Filling Guidelines)
% - Dependencies: Section 5.2 (CMS Switch Actuation)
% - Related sections: 5.1 (Concept), 5.4+ (variant-specific procedures)

% ============================================================================
% ============================================================================
% SECTION 5.3: CMS BLOCK AND VARIANT NOTES
% ============================================================================

\section{CMS Block and Variant Notes}
\label{sec:C5-S3-cms-variants}

The operational behavior of the CMS varies depending on the F-16 variant, ECM pod configuration, and geographic region of deployment. This section clarifies the critical differences between external ECM pods (ALQ-131 / ALQ-184) and internal ECM systems (IDIAS: Improved Defensive Internal Avionic System), and provides a matrix of F-16 blocks and operators to guide pilots and maintainers in understanding which procedures apply to their aircraft.

Understanding these distinctions is \textit{essential for flight safety}. A CMS procedure correct for a Block 52 with external ECM pod will produce unexpected or dangerous results on a Block 52+ with IDIAS, and conversely. Pilots transitioning between aircraft variants must carefully review the applicable procedures before conducting combat operations.

% ============================================================================
% SUBSECTION 5.3.1: EXTERNAL ECM POD VARIANTS (ALQ-131 / ALQ-184)
% ============================================================================

\subsection{External ECM Pod Variants (ALQ-131 / ALQ-184)}
\label{sec:C5-S3-S1-external-ecm-pods}

The ALQ-131 and ALQ-184 are active jamming pods mounted on external hardpoints (typically on the fuselage or wing stations). Both pods provide multi-band frequency jamming and operate under identical CMS control logic: pilot grants transmit consent via CMS Aft, and selects jamming mode via the XMIT knob on the ECM control panel (modes 1, 2, or 3).

\subsubsection{Operational Characteristics}
\label{sec:C5-S3-S1-S1-external-characteristics}

\begin{itemize}

\item \textbf{Transmit Authority}: CMS Aft (short or long hold) enables ECM transmission. ECM Enable light on miscellaneous panel illuminates when consent is active.

\item \textbf{Mode Selection}: XMIT knob on ECM control panel selects mode (1 = AUTO Avionics Priority, 2 = AUTO ECM Priority, 3 = Continuous Jam).

\item \textbf{Frequency Bands}: Both ALQ-131 and ALQ-184 cover five frequency bands, automatically selected during continuous transmission.

\item \textbf{Interaction with RF Switch}: RF switch on throttle overrides CMS Aft. Moving RF to QUIET or SILENT disables pod transmission even if CMS Aft is held. Upon return to NORM, CMS Aft must be re-issued.

\item \textbf{Landing Gear Constraint}: When landing gear is extended (down), ECM pod is held in Standby regardless of CMS Aft state. Retracting gear and re-issuing CMS Aft restores transmission capability.

\item \textbf{Ground Safety}: On the ground, ECM pods must remain in Standby. Do not hold CMS Aft in the vicinity of personnel, as pod radiation poses a hazard.

\end{itemize}

\subsubsection{F-16 Blocks and Operators with External ECM}
\label{sec:C5-S3-S1-S2-external-operators}

The following F-16 variants are equipped with external ECM pods and follow the CMS procedures defined in Section 5.2 (CMS Actuation with ECM):

\begin{enumerate}

\item \textbf{USAF Air Combat Command (ACC) and Air Education and Training Command (AETC)}:

\begin{itemize}
\item Blocks 40, 42, 50, 52 (all variants, including MLU variants)
\item Standard configuration: ALQ-131 or ALQ-184 on Station 1 (left fuselage) or Station 7 (right fuselage)
\item Dash-34 reference: \dashref{2.7.4.2.1}
\end{itemize}

\item \textbf{NATO Allied Air Forces}:

\begin{itemize}
\item \textbf{Belgium}: F-16 Block 15, 25, 30/32 variants
\item \textbf{Denmark}: F-16 Block 20, 25, 30/32 variants
\item \textbf{Netherlands}: F-16 Block 15, 20, 25, 30/32 variants
\item \textbf{Norway}: F-16 Block 15, 20, 25, 30/32 variants
\item All NATO variants use ALQ-131 or ALQ-184 with identical CMS procedures
\end{itemize}

\item \textbf{International Partners}:

\begin{itemize}
\item \textbf{Egypt}: F-16 Blocks 32, 40, 52 with ALQ-131 or ALQ-184
\item \textbf{Pakistan}: F-16 Block 40, 42, 52 with ALQ-131 or ALQ-184
\item \textbf{Chile}: F-16 Block 32, 40, 50 with ALQ-131 or ALQ-184
\item \textbf{Turkey}: F-16 Block 40, 50, 52 with ALQ-131 or ALQ-184
\item \textbf{Japan}: F-16 (F-2 licensed production) with ALQ-131 equivalent
\item All international variants with external pods follow Dash-34 procedures
\end{itemize}

\end{enumerate}

% ============================================================================
% SUBSECTION 5.3.2: INTEGRATED ECM VARIANTS (IDIAS)
% ============================================================================

\subsection{Integrated ECM Variants (IDIAS)}
\label{sec:C5-S3-S2-integrated-idias}

The Improved Defensive Internal Avionic System (IDIAS) is an internal ECM system integrated into the F-16's avionics suite. Unlike external pods, IDIAS operates under fundamentally different CMS control logic: CMS Left (not CMS Aft) cycles through operational modes (Standby, Avionics Priority, ECM Priority), and the XMTR switch on the ECM panel is binary (Standby / Operate), not a three-position mode selector.

\subsubsection{Operational Characteristics}
\label{sec:C5-S3-S2-S1-idias-characteristics}

\begin{itemize}

\item \textbf{Transmit Authority}: CMS Left (repeated short presses) cycles through modes. XMTR switch (binary STBY/OPER) gates all modes. Unlike external pods, CMS Aft does \textit{not} control IDIAS transmission.

\item \textbf{Mode Cycling}: Each CMS Left press advances STBY $\rightarrow$ AVNC (Avionics Priority) $\rightarrow$ ECM (ECM Priority) $\rightarrow$ AVNC $\rightarrow$ ECM (repeating cycle).

\item \textbf{Frequency Band Selection}: IDIAS automatically selects frequency bands based on RWR threat priority. Pilot has no manual band selection control (automatic selection is feature, not limitation).

\item \textbf{Avionics Protection}: In AVNC mode, primary avionics (FCR/TFR/HARM) receive protection; only AFT antenna transmits. In ECM mode, both FWD and AFT antennas transmit; primary avionics may be degraded.

\item \textbf{Interaction with RF Switch}: RF switch on throttle affects IDIAS identically to external pods. Moving RF to QUIET or SILENT disables IDIAS transmission even if mode is ECM. Return to NORM requires mode re-selection via CMS Left.

\item \textbf{Landing Gear Constraint}: When landing gear is extended (down), IDIAS is held in Standby. Retracting gear and re-selecting mode via CMS Left restores transmission capability.

\item \textbf{Warm-up Period}: After power-on, IDIAS requires 5--6 minutes warm-up. During warm-up, STBY lamp flashes. Mode cycling is available but modes will not activate until warm-up completes and XMTR is set to OPER.

\item \textbf{Ground Safety}: IDIAS follows same ground safety practices as external pods. Do not transmit in vicinity of personnel.

\end{itemize}

\subsubsection{F-16 Blocks and Operators with IDIAS}
\label{sec:C5-S3-S2-S2-idias-operators}

The following F-16 variants are equipped with IDIAS and use the CMS Left procedures defined in Section 5.2 (CMS Actuation with ECM --- Internal ECM IDIAS):

\begin{enumerate}

\item \textbf{Israel Defense Force / Air Force (IDFAF)}:

\begin{itemize}
\item F-16I Barak I, Barak II, Sufa (legacy variants)
\item F-16C/D Blocks 30, 40 with IDIAS retrofit
\item IDFAF procured IDIAS as primary defensive system for enhanced survivability in high-threat Middle East environment
\item Dash-34 reference: \dashref{2.7.4.1.1}
\end{itemize}

\item \textbf{Hellenic Air Force (HAF) --- Greece}:

\begin{itemize}
\item F-16C Blocks 50, 52 (designated PXII, PXIII, PXIV in Greek service)
\item IDIAS equipped for NATO operations and interoperability with allied air forces
\item Dash-34 reference: \dashref{2.7.4.1.1}
\end{itemize}

\item \textbf{Republic of Korea Air Force (ROKAF)}:

\begin{itemize}
\item KF-16C Block 52 (upgrade from Block 32 with IDIAS integration)
\item ROKAF adopted IDIAS for air defense mission against North Korean air threats
\item Dash-34 reference: \dashref{2.7.4.1.1}
\end{itemize}

\item \textbf{Republic of Singapore Air Force (RSAF)}:

\begin{itemize}
\item F-16D Block 52 (primarily two-seat variant for training and lead flights)
\item IDIAS selected for regional air defense and interoperability with allied forces
\item Dash-34 reference: \dashref{2.7.4.1.1}
\end{itemize}

\item \textbf{F-16 Block 52+ with IDIAS Retrofit}:

\begin{itemize}
\item Any F-16 Block 52+ equipped with Improved Defensive Internal Avionic System (IDIAS)
\item Retrofit programs may apply to additional operators. Check technical order and aircraft-specific configuration.
\item Dash-34 reference: \dashref{2.7.4.1.1}
\end{itemize}

\end{enumerate}

% ============================================================================
% SUBSECTION 5.3.3: CRITICAL OPERATIONAL DIFFERENCES
% ============================================================================

\subsection{Critical Operational Differences}
\label{sec:C5-S3-S3-critical-differences}

This subsection highlights the most critical distinctions between external ECM and IDIAS to prevent procedural confusion and flight safety incidents.

\subsubsection{CMS Aft vs. CMS Left}
\label{sec:C5-S3-S3-S1-cms-aft-left}

\begin{itemize}

\item \textbf{External ECM Pod}:

\begin{itemize}
\item CMS Aft (short or long hold) = Grant transmit consent
\item Holding CMS Aft maintains transmission
\item Releasing CMS Aft does \textit{not} disable transmission (transmission persists until CMS Right is pressed or threat clears)
\item CMS Left is \textit{not used} for external ECM control
\end{itemize}

\item \textbf{IDIAS}:

\begin{itemize}
\item CMS Left (repeated short presses) = Cycle through modes (STBY $\rightarrow$ AVNC $\rightarrow$ ECM)
\item CMS Aft does \textit{not} control IDIAS transmission (using CMS Aft on IDIAS has \textit{no effect} on ECM)
\item Releasing CMS Left after a press advances mode and persists in that mode
\end{itemize}

\textit{DANGER}: Using CMS Aft on an IDIAS aircraft will NOT enable ECM transmission. A pilot transitioning from external ECM to IDIAS must break the habit of pressing CMS Aft and instead use CMS Left.

\end{itemize}

\subsubsection{XMIT Knob vs. XMTR Switch}
\label{sec:C5-S3-S3-S2-xmit-switch}

\begin{itemize}

\item \textbf{External ECM Pod}:

\begin{itemize}
\item XMIT knob on ECM control panel = Three-position selector (1, 2, 3)
\item Position 1: AUTO Avionics Priority (FCR/TFR protected)
\item Position 2: AUTO ECM Priority (ECM protected)
\item Position 3: Continuous Jam (all bands, continuous transmission)
\item Pilot manually selects mode based on tactical situation
\end{itemize}

\item \textbf{IDIAS}:

\begin{itemize}
\item XMTR switch on ECM panel = Binary selector (STBY / OPER)
\item No three-position mode selection available
\item Operational mode (STBY / AVNC / ECM) selected via CMS Left
\item XMTR OPER gates all modes; XMTR STBY holds IDIAS in Standby
\end{itemize}

\textit{Note}: IDIAS does \textit{not} have a ``Continuous Jam'' mode equivalent. Frequency band selection is automatic based on RWR threats.

\end{itemize}

% ============================================================================
% SUBSECTION 5.3.4: SUMMARY TABLE
% ============================================================================

\subsection{Variant Summary Cross-Reference}
\label{sec:C5-S3-S4-summary-table}

Table \ref{tab:C5-S3-variant-summary} provides a quick reference for determining which CMS procedures apply to a given F-16 variant:

\begin{table}[h]
\centering
\small
\renewcommand{\arraystretch}{1.3}
\begin{tabular}{|L{2.2cm}|L{2.0cm}|L{2.2cm}|L{2.2cm}|L{2.0cm}|L{1.8cm}|}
\hline
\rowcolor{headerblue}
\textbf{\color{white}F-16 Variant} &
\textbf{\color{white}ECM Type} &
\textbf{\color{white}CMS Transmit} &
\textbf{\color{white}CMS Mode Select} &
\textbf{\color{white}Mode Selector} &
\textbf{\color{white}Sect. Ref.} \\
\hline
Block 40/42/50/52 (USAF) & External Pod & CMS Aft & XMIT knob & 3-pos (1,2,3) & 5.2 \\
\hline
Block 15--32 (NATO) & External Pod & CMS Aft & XMIT knob & 3-pos (1,2,3) & 5.2 \\
\hline
Block 32/40/52 (Intl) & External Pod & CMS Aft & XMIT knob & 3-pos (1,2,3) & 5.2 \\
\hline
F-16I Barak (Israel) & IDIAS & CMS Left & CMS Left & Cycle STBY/AVNC/ECM & 5.2 \\
\hline
Block 50/52 PXII--IV (Greece) & IDIAS & CMS Left & CMS Left & Cycle STBY/AVNC/ECM & 5.2 \\
\hline
KF-16C Block 52 (Korea) & IDIAS & CMS Left & CMS Left & Cycle STBY/AVNC/ECM & 5.2 \\
\hline
F-16D Block 52 (Singapore) & IDIAS & CMS Left & CMS Left & Cycle STBY/AVNC/ECM & 5.2 \\
\hline
\end{tabular}
\caption{F-16 Variant ECM Configuration Cross-Reference}
\label{tab:C5-S3-variant-summary}
\end{table}

% ============================================================================
% SUBSECTION 5.3.5: SAFETY NOTES
% ============================================================================

\subsection{Operational Notes and Safety Reminders}
\label{sec:C5-S3-S5-safety-notes}

\subsubsection*{Procedure Compatibility}

Procedures documented in Section 5.2 (CMS Actuation) apply to both external ECM and IDIAS variants, but the specific CMS button presses differ significantly. A pilot must \textit{always verify} the aircraft's ECM configuration before conducting offensive or defensive operations:

\begin{itemize}
\item \textbf{Check the AVIONICS panel} for IDIAS label or external ECM pod indication
\item \textbf{Review the ECM control panel}: Does it have a three-position XMIT knob (external pod) or binary XMTR switch (IDIAS)?
\item \textbf{Consult the aircraft-specific technical order} (TO 1F-16CMAM-34-1-1 or equivalent)
\item \textbf{Ask the crew chief or maintenance}: They will confirm the configuration immediately
\end{itemize}

\subsubsection*{Cross-Training Hazards}

Pilots transitioning between aircraft with different ECM systems must be extremely careful not to apply old procedures to new aircraft:

\begin{itemize}
\item Pressing CMS Aft on IDIAS aircraft does \textit{nothing}; ECM remains in Standby
\item Pressing CMS Left on external ECM aircraft has \textit{no effect}; ECM requires CMS Aft
\item Selecting wrong XMIT mode during high-workload combat may result in unintended avionics degradation or inadequate jamming coverage
\item \textbf{Solution}: Conduct systems-specific training on the new aircraft before operational employment
\end{itemize}

\subsubsection*{Mission Planning Integration}

During mission planning, intelligence and mission planning staff should highlight ECM configuration as part of the threat brief:

\begin{itemize}
\item \textbf{Threat SAM types} may influence choice of ECM mode (Avionics Priority vs. ECM Priority)
\item \textbf{Terrain and engagement geometry} may favor external ECM pod positioning over internal IDIAS
\item \textbf{Fuel constraints} may favor aircraft without external ECM (reduced drag, extended range)
\item \textbf{Interoperability} with allied air forces must account for ECM system differences
\end{itemize}

% ============================================================================
% END OF SECTION
% ============================================================================

\end{document}