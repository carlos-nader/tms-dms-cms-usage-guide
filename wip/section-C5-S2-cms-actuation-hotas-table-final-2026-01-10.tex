% ============================================================================
% FALCON BMS TMS/DMS/CMS HOTAS GUIDE
% WIP FILE — SECTION-LEVEL
% ============================================================================

% Nomenclatura (WIP-NAMING-v1.4):
% section-C5-S2-cms-actuation-review-2026-01-10.tex
% ├─ section: type identifier
% ├─ C5: target chapter 5 (CMS)
% ├─ S2: target section 2 (CMS Switch Actuation)
% ├─ cms-actuation: descriptive title (slugified)
% ├─ FINAL: status (dev | review | final | approved | deprecated)
% └─ 2026-01-10: creation/last-edit date (YYYY-MM-DD)

\documentclass[11pt,a4paper]{article}

% --------------------------------------------------------------------------
% BASIC ENCODING AND LANGUAGE
% --------------------------------------------------------------------------

\usepackage[utf8]{inputenc}
\usepackage[T1]{fontenc}
\usepackage[english]{babel}

% --------------------------------------------------------------------------
% FONTS AND MICROTYPOGRAPHY
% --------------------------------------------------------------------------

\usepackage{lmodern}
\usepackage{microtype}

% --------------------------------------------------------------------------
% PAGE GEOMETRY AND LAYOUT
% --------------------------------------------------------------------------

\usepackage{geometry}
\geometry{a4paper, left=2.0cm, right=2.0cm, top=2.5cm, bottom=2.5cm}
\usepackage{setspace}
\onehalfspacing

% --------------------------------------------------------------------------
% COLORS AND LINKS
% --------------------------------------------------------------------------

\usepackage[table]{xcolor}
\definecolor{linkblue}{HTML}{004488}
\definecolor{linkred}{HTML}{882222}
\definecolor{headerblue}{HTML}{003366}
\definecolor{rowgray}{HTML}{F5F5F5}
\definecolor{subheadgray}{HTML}{E0E0E0}
\usepackage[pdfencoding=auto, psdextra, colorlinks=true, linkcolor=linkblue, citecolor=linkred, urlcolor=linkblue, breaklinks=true]{hyperref}
\usepackage{bookmark}

% --------------------------------------------------------------------------
% HEADERS AND FOOTERS
% --------------------------------------------------------------------------

\usepackage{fancyhdr}
\setlength{\headheight}{15pt}
\pagestyle{fancy}
\fancyhf{}
\fancyhead[L]{\leftmark}
\fancyhead[R]{\rightmark}
\fancyfoot[C]{\thepage}
\renewcommand{\headrulewidth}{0.4pt}
\renewcommand{\footrulewidth}{0pt}

% --------------------------------------------------------------------------
% TABLES AND MACROS
% --------------------------------------------------------------------------

\usepackage{booktabs}
\usepackage{array}
\usepackage{longtable}
\usepackage{tabularx}
\usepackage{caption}

% Custom Columns
\newcolumntype{L}[1]{>{\raggedright\arraybackslash}p{#1}}
\newcolumntype{C}[1]{>{\centering\arraybackslash}p{#1}}
\newcolumntype{R}[1]{>{\raggedleft\arraybackslash}p{#1}}

% HOTAS table environment (per BRIEFING v0.2.0.1, Section 6)
\newenvironment{hotastable}[1]{%
  \small
  \renewcommand{\arraystretch}{1.25}
  \begin{longtable}{L{1.6cm} L{1.0cm} L{1.0cm} L{3.4cm} L{5.8cm} L{1.4cm} L{1.4cm}}
  \caption{#1} \label{tab:#1} \\
  \rowcolor{headerblue}
  \textbf{\color{white}State} &
  \textbf{\color{white}Dir} &
  \textbf{\color{white}Act} &
  \textbf{\color{white}Function} &
  \textbf{\color{white}Effect / Nuance} &
  \textbf{\color{white}Dash34} &
  \textbf{\color{white}Train} \\
  \endfirsthead
  %
  \rowcolor{headerblue}
  \textbf{\color{white}State} &
  \textbf{\color{white}Dir} &
  \textbf{\color{white}Act} &
  \textbf{\color{white}Function} &
  \textbf{\color{white}Effect / Nuance} &
  \textbf{\color{white}Dash34} &
  \textbf{\color{white}Train} \\
  \endhead
  %
  \multicolumn{7}{r}{\small\emph{Continued on next page...}} \\
  \endfoot
  %
  \endlastfoot
}{%
  \end{longtable}%
}

% --------------------------------------------------------------------------
% SIMPLE REFERENCE MACROS FOR BMS DOCS
% --------------------------------------------------------------------------

\providecommand{\dashref}[1]{Dash-34~\S~#1}
\providecommand{\dashone}[1]{Dash-1~\S~#1}
\providecommand{\trnref}[1]{TRN~#1}
\providecommand{\trnman}{BMS Training Manual 4.38.1}
\providecommand{\bmsver}{Falcon BMS~4.38.1}
\providecommand{\dashrefs}[1]{\textit{TO 1F-16CMAM-34-1-1}, Dash-34, sections \texttt{#1}}

% --------------------------------------------------------------------------
% VERSION CONTROL MACROS
% --------------------------------------------------------------------------

\newcommand{\docversion}{WIP}
\newcommand{\docbuild}{section-C5-S2-final}
\newcommand{\fulldocversion}{\docversion~\docbuild}

% --------------------------------------------------------------------------
% GRAPHICS
% --------------------------------------------------------------------------

\usepackage{graphicx}
\graphicspath{{fig/}}

% --------------------------------------------------------------------------
% TITLE (For Standalone Compilation)
% --------------------------------------------------------------------------

\title{F-16 Countermeasures Management Switch (CMS)\\[0.2cm]Section 5.2: CMS Switch Actuation\\[0.2cm]{\small \bmsver{}}}
\author{Carlos ``Metal'' Nader}
\date{WIP Status: review | Date: 2026-01-10}

% --------------------------------------------------------------------------
% DOCUMENT BEGIN
% --------------------------------------------------------------------------

\begin{document}

\maketitle

\newpage

\tableofcontents

\newpage

% ============================================================================
% WIP FILE METADATA (NOT RENDERED IN PDF)
% ============================================================================

% File Name: section-C5-S2-cms-actuation-final-2026-01-10.tex
% WIP Naming Convention: v1.4
% Target Chapter: C5 (Countermeasures Management System — CMS)
% Target Section: S2 (CMS Switch Actuation)
% WIP Status: final (approved for integration)
% Created: 2026-01-08
% Last Modified: 2026-01-10
%
% Narrative Completion: 100%
% Table Fill Status: 100%
%
% Integration notes:
% - Ready for integration into guide.tex
% - Delete preamble (\documentclass through \newpage) during integration
% - Keep content from \section{} forward
% - All three corrections applied (C1, C2, C3)
% - TOC page separation fixed (\newpage before/after \tableofcontents)
%
% Cross-references:
% - Briefing: v0.2.0.1, Section 6 (Column Filling Guidelines)
% - Training missions: per training-mission-abbrev-table-v1.0.md
% - Dependencies: None (standalone section)

% ============================================================================
% ============================================================================
% SECTION 5.2: CMS SWITCH ACTUATION
% ============================================================================

\section{CMS Switch Actuation}
\label{sec:C5-S2-cms-actuation}

The Countermeasures Management Switch (CMS) is a four-direction hat switch located at the flight stick that provides pilots with rapid, direct control over the F-16's defensive systems during demanding tactical situations. This section tabulates all CMS button combinations, organized by operational layer: Counter Measures (CMDS manual/automatic/semi modes) and Defensive Avionics integration (jamming) with both external ECM pods (ALQ-131/ALQ-184) and internal avionics (IDIAS). CMS actuation is independent of the Master Mode currently selected.

Each table entry specifies:

\begin{itemize}
\item \textbf{State}: The operational context (master mode, CMDS mode, sensor state).
\item \textbf{Direction}: The physical direction for pressing the CMS hat (Up, Down, Left, Right).
\item \textbf{Action}: The press type (Short, Long, Long Hold).
\item \textbf{Function}: What the CMS command activates or controls.
\item \textbf{Effect / Nuance}: The resulting system behavior, including tactics and constraints.
\item \textbf{Dash34}: Reference section in the Dash-34 manual.
\item \textbf{Training}: Recommended BMS training missions for hands-on practice.
\end{itemize}

% ============================================================================
% SUBSECTION 5.2.1: CMS ACTUATION WITH CMDS
% ============================================================================

\subsection{CMS Actuation with CMDS}
\label{sec:C5-S2-S1-cms-cmds}

The ALE-47 CMDS (Automatic Chaff and Flare Dispensing System) provides three operational modes: Manual (MAN), Automatic (AUTO), and Semi-Automatic (SEMI). Each mode grants the pilot different levels of control and autonomy over chaff and flare dispensing. The CMS is the primary interface for program execution and consent authority across all three modes.

% ============================================================================
% SUBSUBSECTION 5.2.1.1: CMDS MANUAL MODE
% ============================================================================

\subsubsection{Manual Mode}
\label{sec:C5-S2-S1-S1-cms-cmds-man}

The CMDS Manual (MAN) mode grants the pilot direct, program-by-program control over countermeasure expenditure. Each CMS direction selects or executes a specific program. This mode is recommended when threat types are well-known or when chaff/flare inventory must be conserved, or by pilot choice.

\begin{hotastable}{CMS Actuation with CMDS Manual Mode}

CMDS MAN & Up & Short & Execute Program 1--4 &
Runs the program selected via the CMDS panel PRGM knob once per press. No threat sensing; purely pilot-commanded. Overrides any AUTO dispensing, if running.
& \dashref{2.7.2.2} & \trnref{18 (BARCAP)}, \trnref{28 (SEAD-EW)} \\

CMDS MAN & Left & Short & Execute Program 6 &
Flare-only program. Often pre-configured for close-range air-to-air engagements or MANPAD defense. No dependency on PRGM knob.
& \dashref{2.7.2.2} & \trnref{19 (GUN AIM)} \\

\end{hotastable}

% ============================================================================
% SUBSUBSECTION 5.2.1.2: CMDS AUTOMATIC MODE
% ============================================================================

\subsubsection{Automatic Mode}
\label{sec:C5-S2-S1-S2-cms-cmds-auto}

The CMDS Automatic (AUTO) mode enables the ALE-47 CMDS to dispense chaff/flare programs continuously in response to RWR-detected threats, without requiring pilot consent for each event. Pilot consent is given once by pressing CMS Aft; dispensing continues until CMS Right is pressed or expendables are exhausted.

\begin{hotastable}{CMS Actuation with CMDS Automatic Mode}

CMDS AUTO & Up & Short & Execute Program 1--4 &
Manual override: runs the selected program once, interrupting any ongoing AUTO dispensing. After the manual program completes, AUTO resumes if threat persists. Useful for pilot override in high-threat scenarios.
& \dashref{2.7.2.2} & \trnref{18 (BARCAP)}, \trnref{28 (SEAD-EW)} \\

CMDS AUTO & Left & Short & Execute Program 6 &
Manual flare-only program, overrides AUTO. After execution, AUTO resumes.
& \dashref{2.7.2.2} & \trnref{19 (GUN AIM)} \\

CMDS AUTO & Aft & Short & Give Consent; Enable AUTO Dispensing &
CMS Aft grants consent for AUTO CMDS. RWR-detected threats trigger automatic program dispensing (selected via PRGM knob). Dispensing continues until threat clears or CMS Right is pressed. Consent state persists even if pilot switches to MAN mode; re-engaging AUTO will resume auto-dispense.
& \dashref{2.7.2.1} & \trnref{18 (BARCAP)}, \trnref{28 (SEAD-EW)} \\

CMDS AUTO & Right & Short & Cancel Consent; Disable AUTO &
CMS Right removes CMDS consent and places the ALE-47 in Standby. Automatic dispensing halts immediately. Pilot must re-issue CMS Aft to resume AUTO operation or use the system manually.
& \dashref{2.7.2.1} & \trnref{18 (BARCAP)}, \trnref{28 (SEAD-EW)} \\

\end{hotastable}

% ============================================================================
% SUBSUBSECTION 5.2.1.3: CMDS SEMI-AUTOMATIC MODE
% ============================================================================

\subsubsection{Semi-Automatic Mode}
\label{sec:C5-S2-S1-S3-cms-cmds-semi}

Semi-Automatic (SEMI) mode allows the ALE-47 to prompt the pilot for consent on a per-threat basis. When the RWR detects a threat requiring countermeasures, the CMDS displays ``DISPENSE RDY'' on the control unit and sounds a ``COUNTER'' voice message, requesting pilot consent via CMS Aft.

\begin{hotastable}{CMS Actuation with CMDS Semi-Automatic Mode}

CMDS SEMI & Up & Short & Execute Program 1--4 &
Manual override: runs the selected program once, independent of RWR threat state. After execution, CMDS returns to monitoring for threats and issuing ``COUNTER'' prompts.
& \dashref{2.7.2.2} & \trnref{18 (BARCAP)}, \trnref{28 (SEAD-EW)} \\

CMDS SEMI & Left & Short & Execute Program 6 &
Manual flare-only program, independent of SEMI threat detection. After execution, CMDS resumes SEMI monitoring.
& \dashref{2.7.2.2} & \trnref{19 (GUN AIM)} \\

CMDS SEMI & Aft & Short & Give Consent; Dispense One Program &
When CMDS issues ``COUNTER'' (threat detected), pilot presses CMS Aft to execute one instance of the selected program. If threat persists or a new threat appears, CMDS will issue ``COUNTER'' again. Consent state is tracked; switching to AUTO while consent active will trigger immediate AUTO dispensing on next threat.
& \dashref{2.7.2.1} & \trnref{18 (BARCAP)}, \trnref{28 (SEAD-EW)} \\

CMDS SEMI & Right & Short & Cancel Consent; Return to Standby &
CMS Right removes SEMI consent. CMDS halts monitoring and returns to Standby. ``COUNTER'' messages cease.
& \dashref{2.7.2.1} & \trnref{18 (BARCAP)}, \trnref{28 (SEAD-EW)} \\

\end{hotastable}

% ============================================================================
% SUBSECTION 5.2.2: CMS ACTUATION WITH ECM
% ============================================================================

\subsection{CMS Actuation with ECM}
\label{sec:C5-S2-S2-cms-ecm}

External ECM pods (ALQ-131, ALQ-184) and internal IDIAS (Improved Defensive Internal Avionic System) provide pilot-controlled jamming across frequency bands. The CMS Aft position controls transmit authority for external pods; CMS Left cycles modes for internal IDIAS. Both systems interact with the RF switch and respect landing gear constraints.

% ============================================================================
% SUBSUBSECTION 5.2.2.1: EXTERNAL ECM POD
% ============================================================================

\subsubsection{External ECM Pod (ALQ-131 / ALQ-184)}
\label{sec:C5-S2-S2-S1-cms-ecm-external}

External ECM pods provide pilot-controlled jamming across five frequency-band programs. The CMS Aft position grants ``ECM consent,'' enabling the pod to transmit at the XMIT switch setting (modes 1, 2, or 3). The ECM Enable light on the miscellaneous panel indicates consent state.

\begin{hotastable}{CMS Actuation with External ECM Pod}

ECM Pod & Aft & Short & Enable ECM Transmit; Grant Consent &
CMS Aft illuminates the ECM Enable light and permits the external ECM pod to transmit in the mode set by the XMIT switch on the ECM control panel (XMIT 1: AUTO Avionics Priority; XMIT 2: AUTO ECM Priority; XMIT 3: Continuous Jam). Pod continues transmitting as long as ECM is not cancelled by the pilot or RF switch is moved away from NORM.
& \dashref{2.7.4.2.5} & \trnref{28 (SEAD-EW)} \\

ECM Pod & Right & Short & Disable ECM Transmit; Remove Consent &
CMS Right extinguishes the ECM Enable light and places the ECM pod in Standby, halting transmission immediately. Pod will not transmit until CMS Aft is re-issued.
& \dashref{2.7.4.2.5} & \trnref{28 (SEAD-EW)} \\

\end{hotastable}

% ============================================================================
% SUBSUBSECTION 5.2.2.2: INTERNAL ECM (IDIAS)
% ============================================================================

\subsubsection{Internal ECM (IDIAS)}
\label{sec:C5-S2-S2-S2-cms-ecm-idias}

Internal avionics ECM (IDIAS: Improved Defensive Internal Avionic System) automatically selects frequency bands to jam based on RWR threat priority. CMS Left cycles through operational modes (Standby, Avionics Priority, ECM Priority). CMS Aft is not used for IDIAS; mode control is via CMS Left and the XMTR button on the ECM panel.

\begin{hotastable}{CMS Actuation with Internal ECM (IDIAS)}

IDIAS ECM & Left & Short & Cycle ECM Operational Mode &
Each short press of CMS Left advances the ECM mode: STBY $\rightarrow$ AVNC (Avionics Priority) $\rightarrow$ ECM (ECM Priority) $\rightarrow$ AVNC $\rightarrow$ ECM (cycles).
& \dashref{2.7.4.1.2} & \trnref{28 (SEAD-EW)} \\

IDIAS ECM & Right & Short & Set ECM to Standby &
CMS Right forces IDIAS ECM into STBY mode, halting all jamming operations. Requires CMS Left cycling to return to AVNC or ECM mode.
& \dashref{2.7.4.1.2} & \trnref{28 (SEAD-EW)} \\

\end{hotastable}

% ============================================================================
% SUBSECTION 5.2.3: CMS CONSENT AND CONSTRAINTS
% ============================================================================

\subsection{CMS Consent and Constraints}
\label{sec:C5-S2-S3-cms-constraints}

This subsection clarifies the relationship between CMDS consent (CMS Aft) and ECM transmit authority (CMS Aft for external pod). Understanding these interactions is critical for effective defensive posture management during high-workload combat operations.

\begin{hotastable}{CMS Interaction with CMDS and ECM (Consent and Constraints)}

CMDS AUTO/SEMI + ECM Pod & Aft & Short & Joint Consent (CMDS + ECM) &
Single CMS Aft command grants consent to \textit{both} the ALE-47 CMDS (AUTO/SEMI) and the external ECM pod. There's no distinction between the two; both systems respond to the same CMS Aft press. This unified control maximizes pilot situational awareness and frees workload during combat maneuvering.
& \dashref{2.7.2.1}, \dashref{2.7.4.2.5} & \trnref{18 (BARCAP)}, \trnref{28 (SEAD-EW)} \\

\end{hotastable}

% ============================================================================
% SUBSECTION 5.2.4: IMPORTANT OPERATIONAL NOTES
% ============================================================================

\subsection{Important Operational Notes}
\label{sec:C5-S2-S4-cms-notes}

The CMS provides rapid, tactile access to CMDS program selection and ECM transmit authority without requiring the pilot to manipulate distant panels during high-G maneuvering. Mastery of CMS actuation across all CMDS modes (MAN, SEMI, AUTO) and ECM configurations (external pod, IDIAS) is essential for effective defensive operations. Pilots must understand the consent state model, RF switch interactions, and inventory management to avoid unintended dispensing or system saturation.

\subsubsection*{Consent State Tracking}

In AUTO and SEMI modes, the CMDS tracks the consent state even if the pilot temporarily switches to MAN mode. If the pilot gives CMS Aft consent in AUTO, then switches the CMDS MODE knob to MAN, the consent state is retained. Upon re-engaging AUTO without issuing CMS Aft again, the CMDS will immediately begin dispensing if a threat is detected. This behavior can be exploited for rapid mode switching during combat but may also lead to unintended dispensing if not carefully managed.

\subsubsection*{Bingo Quantity Behavior}

If expendables (chaff or flare) fall to or below the bingo quantity, the CMDS will still request consent (CMS Aft) and continue dispensing. The ``LOW'' and ``OUT'' voice messages alert the pilot to low or exhausted inventory, but dispensing does not automatically stop. Pilot must monitor EWS upfront pages and manually manage inventory via CMDS MAN or by pressing CMS Right to inhibit AUTO.

\subsubsection*{RF Switch Override}

The RF switch on the throttle is a master control for ECM transmission. Moving the RF switch away from NORM (e.g., to QUIET or SILENT) overrides any previous CMS Aft command and places both the external ECM pod and internal IDIAS in Standby. Returning RF to NORM does \textit{not} automatically restore transmission; the pilot must re-issue CMS Aft.

\subsubsection*{ECM Consent vs. CMDS Consent}

A single CMS Aft press grants consent to both the ALE-47 CMDS and the external ECM pod. The pilot does not issue separate commands; the CMS Aft action is unified. However, internal IDIAS uses CMS Left for mode cycling, not CMS Aft. This distinction is critical for aircraft configured with IDIAS.

\subsubsection*{Ground Operations Safety}

On the ground, ECM pods are held in Standby for safety reasons. Pilots must not hold CMS Aft while on the ground in the vicinity of personnel, as the ECM pod may radiate and pose a hazard. Ground personnel must be clear before the pilot engages ECM for pre-flight high-level BIT (Built-In Test). Once airborne, ECM consent (CMS Aft) can be issued and maintained as tactically required.

% ============================================================================
% END OF SECTION
% ============================================================================

\end{document}