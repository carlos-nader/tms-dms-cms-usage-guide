%%%%%%%%%%%%%%%%%%%%%%%%%%%%%%%%%%%%%%%%%%%%%%%%%%%%%%%%%%%%%%%%%%%%%%%%%%%%%%%
% FALCON BMS TMSDMSCMS HOTAS GUIDE WIP FILE TEMPLATE V1.0
% File Name: section-C4-S1-to-S5-dms-chapter-structure-dev-2026-01-12.tex
% WIP Naming Convention: v1.4
% Target Chapter: C4 (DMS -- Display Management Switch)
% Target Section: S1 to S5
% Target Subsection: All (S1.1, S1.2, S2.1, S2.2, S2.3, etc.)
% WIP Status: dev (Development - Structural skeleton ready)
% Created: 2026-01-12
% Last Modified: 2026-01-12
% Integration Status: PENDING APPROVAL - Awaiting author review of canonical compliance
% Narrative Completion: 5% (structure only)
% Table Fill Status: 0% (placeholders only)
% Notes:
%   - Structure refined from original plan using comparative analysis with CMS Chapter 5
%   - Incorporates DCLT (Declutter) as dedicated subsection (missing in original plan)
%   - Approximately 60% of CMS subsections (reflects DMS simplicity vs CMS complexity)
%   - All Dash-34 references to be verified: currently using placeholder 2.1.6 for MFD
%   - Training mission references TBD - recommend missions with heavy DMS usage
%   - Hotastables follow canonical template: 7 columns (State, Dir, Act, Fn, Eff, Nuance, Dash34/Train)
%   - Divergences: None anticipated from briefing v0.2.0.1
%%%%%%%%%%%%%%%%%%%%%%%%%%%%%%%%%%%%%%%%%%%%%%%%%%%%%%%%%%%%%%%%%%%%%%%%%%%%%%%

\documentclass[11pt,a4paper]{article}

% --------------------------------------------------------------------------
% BASIC ENCODING AND LANGUAGE
% --------------------------------------------------------------------------
\usepackage[utf8]{inputenc}
\usepackage[english]{babel}

% --------------------------------------------------------------------------
% FONTS AND MICROTYPOGRAPHY
% --------------------------------------------------------------------------
\usepackage[protrusion=true,expansion=true]{microtype}

% --------------------------------------------------------------------------
% MATH AND SYMBOLS
% --------------------------------------------------------------------------
\usepackage{amsmath}
\usepackage{amssymb}

% --------------------------------------------------------------------------
% PAGE GEOMETRY AND LAYOUT
% --------------------------------------------------------------------------
\usepackage[a4paper,left=2.0cm,right=2.0cm,top=2.5cm,bottom=2.5cm]{geometry}

% --------------------------------------------------------------------------
% COLORS AND LINKS
% --------------------------------------------------------------------------
\usepackage{xcolor}
\definecolor{linkblue}{HTML}{004488}
\definecolor{linkred}{HTML}{882222}
\definecolor{headerblue}{HTML}{003366}
\definecolor{rowgray}{HTML}{F5F5F5}
\definecolor{subheadgray}{HTML}{E0E0E0}
\usepackage[pdfencoding=auto,psdextra,colorlinks=true,linkcolor=linkblue,citecolor=linkred,urlcolor=linkblue,breaklinks=true]{hyperref}

% --------------------------------------------------------------------------
% HEADERS AND FOOTERS
% --------------------------------------------------------------------------
\usepackage{fancyhdr}
\lhead{}
\rhead{}
\chead{}
\setlength{\headrulewidth}{0.4pt}

% --------------------------------------------------------------------------
% TABLES AND MACROS
% --------------------------------------------------------------------------
\usepackage{longtable}
\usepackage{array}
\usepackage{colortbl}

% Custom column widths per Briefing v0.2.0.1 Section 6
\newcolumntype{L}[1]{>{\raggedright\let\newline\\\arraybackslash\hspace{0pt}}m{#1}}
\newcolumntype{C}[1]{>{\centering\let\newline\\\arraybackslash\hspace{0pt}}m{#1}}
\newcolumntype{R}[1]{>{\raggedleft\let\newline\\\arraybackslash\hspace{0pt}}m{#1}}

% HOTAS table environment per Briefing v0.2.0.1, Section 6
\newenvironment{hotastable}{%
  \begin{longtable}{L{1.6cm} L{1.0cm} L{1.0cm} L{3.4cm} L{5.8cm} L{1.4cm} L{1.4cm}}
  \rowcolor{headerblue}
  \textcolor{white}{\textbf{State}} & \textcolor{white}{\textbf{Dir}} & \textcolor{white}{\textbf{Act}} & 
  \textcolor{white}{\textbf{Function}} & \textcolor{white}{\textbf{Effect}} & \textcolor{white}{\textbf{Nuance}} & 
  \textcolor{white}{\textbf{Dash34/Train}} \\
  \hline
  \endhead
  \multicolumn{7}{r}{\textit{Continued on next page}} \\
  \endfoot
}{%
  \end{longtable}
}

% --------------------------------------------------------------------------
% SIMPLE REFERENCE MACROS FOR BMS DOCS
% --------------------------------------------------------------------------
\newcommand{\Dash34}{TO 1F-16CMAM-34-1-1}
\newcommand{\Dash1}{TO 1F-16CMAM-1}
\newcommand{\TRN}{BMS Training Manual 4.38.1}
\newcommand{\BMSFULL}{Falcon BMS 4.38.1}

% --------------------------------------------------------------------------
% VERSION CONTROL MACROS
% --------------------------------------------------------------------------
\newcommand{\wipversion}{0.1.0}
\newcommand{\wipdate}{12 January 2026}

% --------------------------------------------------------------------------
% GRAPHICS
% --------------------------------------------------------------------------
% Placeholder for future graphics setup (chapter 4 images TBD)

% --------------------------------------------------------------------------
% TITLE
% --------------------------------------------------------------------------
\title{DMS -- Display Management Switch\\
        Falcon BMS 4.38.1 HOTAS Guide -- Chapter 4\\
        \small{Canonical WIP Structure v\wipversion{}}}
\author{Carlos ``Metal'' Nader}
\date{Chapter 4 Structure v\wipversion{} -- \wipdate{}}

% --------------------------------------------------------------------------
% DOCUMENT BEGIN
% --------------------------------------------------------------------------
\begin{document}

\pagestyle{fancy}
\maketitle
\tableofcontents
\newpage

%===============================================================================
% WIP FILE METADATA (NOT RENDERED IN PDF)
%===============================================================================

% File Name: section-C4-S1-to-S5-dms-chapter-structure-dev-2026-01-12.tex
% WIP Naming Convention: v1.4
% Target Chapter: C4 (DMS -- Display Management Switch)
% Target Section: S1, S2, S3, S4, S5
% Target Subsection: S1.1, S1.2, S2.1, S2.2, S2.3, S3.1, S3.2, S3.3, S3.4, S4.1, S4.2, S4.3, S4.4, S5.1
% WIP Status: dev
% Created: 2026-01-12
% Last Modified: 2026-01-12
% Integration Status: PENDING APPROVAL
% Narrative Completion: 5%
% Table Fill Status: 0%
% Notes:
%   - Structure refined from original plan
%   - DCLT as dedicated subsection (was missing)
%   - Dash-34 refs: placeholder 2.1.6 (MFD) -- VERIFY
%   - Training mission refs: TBD
%   - Hotastable templates provided; rows to be completed

%===============================================================================
% CHAPTER 4: DMS -- DISPLAY MANAGEMENT SWITCH
%===============================================================================

\section{DMS -- Display Management Switch}
\label{secC4}

{\small \textit{Status: 5\% complete -- Structural skeleton ready} \par}

\subsection{Overview of This Chapter}

This chapter provides comprehensive coverage of the Display Management Switch (DMS), the pilot's primary interface for managing avionics displays and selecting the active Sensor of Interest (SOI) in the F-16. Unlike the Target Management Switch (TMS) covered in Chapter 3, the DMS is \textbf{deterministic and linear} in its behavior: each gesture always produces the same predictable result, regardless of Master Mode or system state.

The chapter is organized as follows:
\begin{enumerate}
    \item \textbf{Section 4.1}: Concept and role of DMS in the F-16 avionics architecture
    \item \textbf{Section 4.2}: DMS switch actuation -- the primary section detailing all DMS controls
    \item \textbf{Section 4.3}: DMS behavior in specific tactical contexts (by Master Mode and Sensor)
    \item \textbf{Section 4.4}: Operational constraints and important notes
    \item \textbf{Section 4.5}: Block and variant differences (minimal for DMS)
\end{enumerate}

%===============================================================================
% SECTION 4.1: CONCEPT AND SENSOR OF INTEREST (SOI) SELECTION
%===============================================================================

\section{Concept and Sensor of Interest (SOI) Selection}
\label{secC4-S1}

{\small \textit{Status: 0\% complete -- Narrative/Content pending} \par}

\subsection{4.1.1 DMS Role and Ergonomics}
\label{secC4-S1-S1}

\subsubsection*{Status}
{\small Narrative: 0\% | Content Outline: Ready \par}

\subsubsection*{Content Outline (TODO)}

This subsection should establish:

\begin{itemize}
    \item \textbf{Why DMS is simple}: Explain that unlike TMS (context-dependent, timing-critical), DMS is completely deterministic. Same input always produces same output.
    \item \textbf{Ergonomic placement}: DMS located on flight stick within thumb reach. Allows pilots to manage displays without redirecting hand position.
    \item \textbf{Operational criticality}: DMS enables rapid SOI cycling during high-workload situations (unlike distant MFD touchscreen controls).
    \item \textbf{Contrast with TMS}: While TMS is complex and context-sensitive, DMS provides simple linear control for display management.
    \item \textbf{Historical stability}: DMS behavior unchanged since at least BMS 4.36, making this guide broadly applicable.
\end{itemize}

\subsubsection*{Narrative to Write}

\noindent
\textit{TODO: Approximately 300--400 words. Topics: physical location of DMS on stick, thumb-activated operation, why deterministic design matters operationally, pilot workload reduction, simplicity vs TMS complexity.}

%-------

\subsection{4.1.2 Sensor of Interest (SOI) Hierarchy}
\label{secC4-S1-S2}

\subsubsection*{Status}
{\small Narrative: 0\% | Content Outline: Ready \par}

\subsubsection*{Content Outline (TODO)}

This subsection should explain:

\begin{itemize}
    \item \textbf{What is SOI}: Define as the actively controlled sensor whose data is displayed and manipulated via stick controls (particularly TMS).
    \item \textbf{SOI Hierarchy}: Explain cycling order via DMS UP/DOWN:
    \begin{enumerate}
        \item Targeting Pod (TGP)
        \item Fire Control Radar (FCR)
        \item Radar Warning Receiver (RWR)
        \item Navigation (NAV)
        \item Cycle repeats
    \end{enumerate}
    \item \textbf{SOI Persistence}: Selected SOI persists across Master Mode changes. If FCR selected in A-A, then switch to A-G, FCR remains SOI until cycled to different sensor.
    \item \textbf{Context Changes with SOI}: DMS UP/DOWN meaning changes based on SOI:
    \begin{itemize}
        \item FCR SOI: cycle between radar modes (STT, PRH, ACM, etc.)
        \item TGP SOI: cycle between targeting modes (POINT, AREA, PICTURE, etc.)
        \item RWR SOI: cycle between threat display formats
        \item NAV SOI: cycle between navigation display formats
    \end{itemize}
    \item \textbf{Visual Feedback}: Current SOI always visible on MFD, typically indicated by highlighted border or label.
\end{itemize}

\subsubsection*{Narrative to Write}

\noindent
\textit{TODO: Approximately 400--500 words. Topics: SOI concept, cycling hierarchy, persistence across modes, context-dependent meanings, visual indicators, tactical implications.}

%===============================================================================
% SECTION 4.2: DMS SWITCH ACTUATION -- SOI AND FORMAT STEPPING (MAIN)
%===============================================================================

\section{DMS Switch Actuation -- SOI Selection and Format Stepping}
\label{secC4-S2}

{\small \textit{Status: 10\% complete -- Hotastable templates ready, narratives pending} \par}

This is the \textbf{primary section} of the DMS chapter. It details all DMS button combinations, organized by functional intent (SOI selection, Format stepping, Declutter mode) rather than by Master Mode or Sensor context. Context-specific variations are addressed in Section 4.3.

%-------

\subsection{4.2.1 DMS UP/DOWN -- Sensor of Interest (SOI) Selection}
\label{secC4-S2-S1}

\subsubsection*{Status}
{\small Narrative: 0\% | Hotastable: Template ready \par}

\subsubsection*{Content Outline (TODO)}

\begin{itemize}
    \item \textbf{Brief introduction} (approx. 100 words): Explain that DMS UP/DOWN cycles through SOI hierarchy. Result always: UP = previous SOI, DOWN = next SOI.
    \item \textbf{Hotastable with 2 rows}:
    \begin{itemize}
        \item Row 1: Any Master Mode, Any SOI | UP (short press) | Selects Previous SOI
        \item Row 2: Any Master Mode, Any SOI | DOWN (short press) | Selects Next SOI
    \end{itemize}
    \item \textbf{Cross-references}: Point to Section 4.1.2 (SOI Hierarchy) and Section 4.3 (context-specific behaviors).
    \item \textbf{Training references}: Link to BMS Training Missions requiring heavy SOI cycling (recommend: mixed-sensor scenarios).
\end{itemize}

\subsubsection*{Narrative to Write}

\noindent
\textit{TODO: Approximately 200--300 words. Topics: SOI cycling operation, tactical flow, rapid sensor switching, workload reduction.}

\vspace{0.3cm}

\subsubsection*{Hotastable Template}

\begin{hotastable}
State & Dir & Act & Function & Effect & Nuance & Dash34/Train \\
\hline
\multicolumn{7}{|l|}{\textit{TODO: Complete table rows. Use template rows below as starting point.}} \\
\hline
Any SOI & UP & Shrt & Select Previous SOI & DMS UP advances backward through SOI hierarchy: TGP to NAV to RWR to FCR to TGP. SOI change immediate. MFD updates to new SOI's last-used display format. & Selection persists across Master Mode changes. SOI context retained per sensor (e.g., if FCR was in STT mode, returning to FCR preserves STT). & \Dash34{} 2.1.6 / TBD \\
\hline
Any SOI & DOWN & Shrt & Select Next SOI & DMS DOWN advances forward through SOI hierarchy: TGP to FCR to RWR to NAV to TGP. SOI change immediate. MFD updates. Works regardless of OSB visibility or declutter state. & Same persistence behavior as UP. Cycling deterministic; no timing sensitivity. & \Dash34{} 2.1.6 / TBD \\
\hline
\end{hotastable}

%-------

\subsection{4.2.2 DMS LEFT/RIGHT -- MFD Format Stepping}
\label{secC4-S2-S2}

\subsubsection*{Status}
{\small Narrative: 0\% | Hotastable: Template ready \par}

\subsubsection*{Content Outline (TODO)}

\begin{itemize}
    \item \textbf{Brief introduction} (approx. 100 words): DMS LEFT/RIGHT steps through available MFD display formats for currently selected SOI. Each sensor (FCR, TGP, RWR, NAV) maintains own format list.
    \item \textbf{Hotastable with 2 rows}:
    \begin{itemize}
        \item Row 1: [Current SOI], [Any Format] | LEFT (short press) | Previous MFD Format
        \item Row 2: [Current SOI], [Any Format] | RIGHT (short press) | Next MFD Format
    \end{itemize}
    \item \textbf{Format Memory}: Each MFD format maintains independent state (zoom level, range setting, symbol visibility, etc.). Returning to previously-used format retains all state settings.
    \item \textbf{Cross-references}: Section 4.3 for format options specific to each sensor/Master Mode.
    \item \textbf{Training references}: Missions emphasizing MFD format management.
\end{itemize}

\subsubsection*{Narrative to Write}

\noindent
\textit{TODO: Approximately 250--350 words. Topics: format cycling, state retention, tactical use, high-G maneuvering benefit.}

\vspace{0.3cm}

\subsubsection*{Hotastable Template}

\begin{hotastable}
State & Dir & Act & Function & Effect & Nuance & Dash34/Train \\
\hline
\multicolumn{7}{|l|}{\textit{TODO: Complete table rows. Use template rows below as starting point.}} \\
\hline
{[}Current SOI{]} & LEFT & Shrt & Previous MFD Format & DMS LEFT cycles backward through available MFD formats for current SOI. Display changes immediately to previous format. Format-specific state (zoom, range, symbol filters) retained. & Formats available depend on SOI and Master Mode. See Section 4.3 for complete format listings per context. & \Dash34{} 2.1.6 / TBD \\
\hline
{[}Current SOI{]} & RIGHT & Shrt & Next MFD Format & DMS RIGHT cycles forward through available MFD formats for current SOI. Display changes immediately. Format-specific state retained. Useful during high-workload or high-G maneuvering. & Same format cycling logic as LEFT. Allows rapid stepping without using MFD touchscreen (beneficial during combat). & \Dash34{} 2.1.6 / TBD \\
\hline
\end{hotastable}

%-------

\subsection{4.2.3 DCLT (Declutter) Mode -- Visual Management}
\label{secC4-S2-S3}

\subsubsection*{Status}
{\small Narrative: 0\% | Hotastable: Template ready \par}

\subsubsection*{Content Outline (TODO)}

\begin{itemize}
    \item \textbf{What is Declutter}: DCLT is visual simplification mode that removes labels and symbols from MFD while retaining critical flight data. Useful for situational awareness during congested displays or multi-target environments.
    \item \textbf{Two declutter modes}:
    \begin{enumerate}
        \item Toggle Declutter: brief press (less than 1 second) toggles clutter on/off
        \item Customize Declutter: long hold (1 second or more) accesses programmable declutter page
    \end{enumerate}
    \item \textbf{State persistence}: DCLT state is format-specific (not SOI-specific). If you set declutter in FCR STT format, leaving and returning to STT retains declutter state. Switching to FCR PRH may have independent declutter state.
    \item \textbf{Hotastable with 2 rows}:
    \begin{itemize}
        \item Row 1: [Any Format] | (any direction) | Brief (less than 1s) | Toggle Declutter
        \item Row 2: [Any Format] | (any direction) | Long (1s or more) | Access Declutter Configuration
    \end{itemize}
    \item \textbf{Cross-references}: Note this is the ONLY DMS function with timing criticality. Contrast with SOI/Format operations (instant, timing-independent).
\end{itemize}

\subsubsection*{Narrative to Write}

\noindent
\textit{TODO: Approximately 300--400 words. Topics: declutter functionality, toggle vs customize timing, state format-specific, tactical use, visual confirmation requirement.}

\vspace{0.3cm}

\subsubsection*{Hotastable Template}

\begin{hotastable}
State & Dir & Act & Function & Effect & Nuance & Dash34/Train \\
\hline
\multicolumn{7}{|l|}{\textit{TODO: Complete table rows. Use template rows below as starting point.}} \\
\hline
{[}Any Format{]} & {[}Any{]} & Brief & Toggle Declutter On/Off & DMS brief press alternates visual declutter state. When decluttered: OSB labels, threat symbols, non-essential annotations hidden. Critical flight data remains (range, heading, airspeed symbology). & Declutter state is format-specific, not SOI-specific. Useful in multi-target intercept scenarios. No voice warning or visual indicator of state; pilot must confirm visually. & \Dash34{} 2.1.6 / TBD \\
\hline
{[}Any Format{]} & {[}Any{]} & Long & Access Declutter Config & DMS long hold opens declutter customization page. Pilot can enable/disable decluttering of specific elements (radial lines, grid, labels, threat circles, etc.). Selection stored per format. & Customization available per individual pilot preference. Some pilots prefer minimal declutter (grid plus labels); others prefer maximum (only FPM and range). Accessible only via long hold; not available via OSB. & \Dash34{} 2.1.6 / TBD \\
\hline
\end{hotastable}

%===============================================================================
% SECTION 4.3: DMS IN TACTICAL CONTEXTS
%===============================================================================

\section{DMS in Tactical Contexts}
\label{secC4-S3}

{\small \textit{Status: 0\% complete -- Context-specific narratives pending} \par}

This section elaborates on DMS behavior in specific operational contexts, organized by Master Mode and Sensor. While Section 4.2 presented DMS control generically, this section provides concrete examples of what DMS UP/DOWN and LEFT/RIGHT mean when actually flying.

%-------

\subsection{4.3.1 DMS with FCR in Air-to-Air Mode}
\label{secC4-S3-S1}

\subsubsection*{Content Outline (TODO)}

\begin{itemize}
    \item \textbf{Introduction}: In A-A Master Mode with FCR as SOI, DMS UP/DOWN cycle between radar search/track modes (not between sensors). Each mode has associated MFD formats.
    \item \textbf{Radar modes in A-A}: List and briefly describe: STT (Single Target Track), PRH (Patrol Range Height), ACM (Air Combat Maneuvering), DGFT (Dogfight), etc.
    \item \textbf{Hotastable}: Rows for each radar mode cycling and associated format changes
    \item \textbf{Format examples}: Explain what MFD formats are available in STT vs PRH vs ACM modes
    \item \textbf{Tactical flow}: Describe typical DMS usage in intercept/BVR engagement scenario
\end{itemize}

\subsubsection*{Narrative to Write}

\noindent
\textit{TODO: Approximately 400--500 words. Topics: FCR A-A context, radar modes, format cycling, tactical engagement flow.}

%-------

\subsection{4.3.2 DMS with TGP in Air-to-Ground Mode}
\label{secC4-S3-S2}

\subsubsection*{Content Outline (TODO)}

\begin{itemize}
    \item \textbf{Introduction}: In A-G Master Mode with TGP as SOI, DMS UP/DOWN cycle between targeting pod modes (POINT, AREA, PICTURE, etc.), not between sensors.
    \item \textbf{TGP modes in A-G}: Define POINT (single-pixel target), AREA (geometric shape for complex structures), PICTURE (wide-area overview), STANDBY, etc.
    \item \textbf{Hotastable}: Rows for each TGP mode cycling and associated format changes
    \item \textbf{Format examples}: Explain MFD formats unique to TGP (e.g., PIP -- Picture-in-Picture for context)
    \item \textbf{Tactical flow}: Describe typical scenario (find target with PICTURE, transition to AREA for structure tracking, narrow to POINT for laser lasing)
    \item \textbf{DCLT interaction}: Note that DCLT particularly useful in TGP mode to simplify display during high-precision lasing
\end{itemize}

\subsubsection*{Narrative to Write}

\noindent
\textit{TODO: Approximately 400--500 words. Topics: TGP A-G context, targeting modes, format cycling, laser employment flow.}

%-------

\subsection{4.3.3 DMS with RWR (Always Available)}
\label{secC4-S3-S3}

\subsubsection*{Content Outline (TODO)}

\begin{itemize}
    \item \textbf{Introduction}: RWR (Radar Warning Receiver) available in all Master Modes as a SOI. DMS UP/DOWN in RWR mode cycle between threat display formats, not modes.
    \item \textbf{RWR display formats}: List format options (Lobe display, Table format, Azimuth/Elevation, etc.)
    \item \textbf{Hotastable}: Rows for RWR format cycling
    \item \textbf{Tactical importance}: RWR is crucial for situational awareness of threat radar. Quick format cycling allows pilot to assess threat geometry without using OSBs.
    \item \textbf{Context across Master Modes}: Note that RWR SOI selection and format cycling work identically regardless of Master Mode (A-A, A-G, NAV, DGFT)
\end{itemize}

\subsubsection*{Narrative to Write}

\noindent
\textit{TODO: Approximately 300--400 words. Topics: RWR as SOI, format cycling, threat assessment, cross-mode consistency.}

%-------

\subsection{4.3.4 DMS in Navigation Mode}
\label{secC4-S3-S4}

\subsubsection*{Content Outline (TODO)}

\begin{itemize}
    \item \textbf{Introduction}: Explain DMS behavior when in NAV Master Mode. Focus shifts from weapons/targeting to navigation displays (HSD, HSI, INS, waypoint management).
    \item \textbf{Navigation formats}: List available formats (Horizontal Situation Display, Heading/Situation Indicator, INS readout, Waypoint/Steerpoint pages, etc.)
    \item \textbf{Hotastable}: Rows for navigation format cycling
    \item \textbf{SOI cycling in NAV mode}: Explain if SOI selection behaves differently (typically navigation sensors like INS/GPS prioritized)
    \item \textbf{Tactical flow}: Describe typical NAV mode scenario (preflight planning, cruise, approach to target area)
\end{itemize}

\subsubsection*{Narrative to Write}

\noindent
\textit{TODO: Approximately 300--400 words. Topics: NAV mode context, available displays, format cycling, mission planning flow.}

%===============================================================================
% SECTION 4.4: DMS CONSTRAINTS AND OPERATIONAL NOTES
%===============================================================================

\section{DMS Constraints and Operational Notes}
\label{secC4-S4}

{\small \textit{Status: 0\% complete -- Bullet-point narratives pending} \par}

This section clarifies important operational constraints and behaviors that pilots must understand.

%-------

\subsection{4.4.1 SOI State Persistence}
\label{secC4-S4-S1}

\subsubsection*{Content Outline (TODO)}

\begin{itemize}
    \item SOI selected in one Master Mode persists when you switch Master Modes
    \item Example: A-A with FCR selected, switch to A-G, FCR remains selected (unless FCR unavailable in A-G)
    \item Consequence: Pilot must consciously re-select appropriate SOI when changing Master Modes
    \item Safeguard: Visual indication on MFD shows current SOI at all times
\end{itemize}

\subsubsection*{Narrative to Write}

\noindent
\textit{TODO: Approximately 150--200 words on SOI persistence across modes.}

%-------

\subsection{4.4.2 Format Memory}
\label{secC4-S4-S2}

\subsubsection*{Content Outline (TODO)}

\begin{itemize}
    \item Each MFD format (e.g., FCR STT, FCR PRH, TGP POINT) maintains independent state
    \item State includes: zoom level, range setting, symbol visibility, azimuth orientation, etc.
    \item When pilot cycles back to previously-used format, all state is restored
    \item Benefit: Allows rapid format cycling without losing context
    \item Caveat: Pilot must be aware that left/right stepping may restore unexpected zoom/range settings
\end{itemize}

\subsubsection*{Narrative to Write}

\noindent
\textit{TODO: Approximately 150--200 words on format memory behavior.}

%-------

\subsection{4.4.3 Declutter State Retention}
\label{secC4-S4-S3}

\subsubsection*{Content Outline (TODO)}

\begin{itemize}
    \item DCLT state (on/off and customization settings) stored per format, not per SOI
    \item Switching to different format may have different DCLT state
    \item Customized declutter settings persist across flights (user-configurable)
    \item Tactical implication: Pilot can set up preferred declutter states for each critical format
\end{itemize}

\subsubsection*{Narrative to Write}

\noindent
\textit{TODO: Approximately 150--200 words on DCLT state retention.}

%-------

\subsection{4.4.4 No Timing Criticality (Contrast with TMS)}
\label{secC4-S4-S4}

\subsubsection*{Content Outline (TODO)}

\begin{itemize}
    \item DMS has zero sub-second timing criticality (except DCLT brief vs long, approx 1 second)
    \item DMS UP held 0.1s equals DMS UP held 2.0s; result identical
    \item DMS LEFT brief tap equals DMS LEFT sustained hold; result identical
    \item Contrast: TMS with timing less than 0.6s versus greater than or equal to 0.6s fundamentally changes mode (SCAN vs LOS)
    \item Implication: Pilot can safely manipulate DMS without worrying about timing precision
    \item Safety benefit: DMS is forgiving and reliable; no risk of mode switching due to hold duration
\end{itemize}

\subsubsection*{Narrative to Write}

\noindent
\textit{TODO: Approximately 200--250 words explaining DMS determinism and lack of timing sensitivity.}

%===============================================================================
% SECTION 4.5: DMS BLOCK AND VARIANT NOTES
%===============================================================================

\section{DMS Block and Variant Notes}
\label{secC4-S5}

{\small \textit{Status: 0\% complete -- Content likely minimal} \par}

Unlike the CMS (Chapter 5), which has significant variation between External ECM Pod and Internal IDIAS configurations, the DMS is \textbf{remarkably consistent} across all F-16 blocks and variants. This section will be brief.

%-------

\subsection{4.5.1 DMS Behavior Across All Blocks}
\label{secC4-S5-S1}

\subsubsection*{Content Outline (TODO)}

\begin{itemize}
    \item \textbf{Universal consistency}: All F-16 blocks and variants (Block 40/42/50/52, export variants, etc.) implement DMS identically
    \item \textbf{Physical switch}: 4-direction hat switch on flight stick (standard across all variants)
    \item \textbf{UP/DOWN function}: SOI cycling is identical
    \item \textbf{LEFT/RIGHT function}: MFD format stepping is identical
    \item \textbf{DCLT}: Declutter functionality is identical
    \item \textbf{Implication}: Pilot training on DMS transfers directly between any F-16 variant
    \item \textbf{No table needed}: Unlike CMS, no variant comparison table required
\end{itemize}

\subsubsection*{Narrative to Write}

\noindent
\textit{TODO: Approximately 200--250 words on DMS consistency across blocks and variants.}

%===============================================================================
% END OF CHAPTER 4 STRUCTURE
%===============================================================================

\newpage

\section*{Appendix: WIP Development Notes}

\subsection*{Pending Tasks}

\begin{enumerate}
    \item Verify Dash-34 section references (placeholder: 2.1.6 for MFD -- is this correct in DASH-34-1?)
    \item Identify 5--8 BMS Training Missions with heavy DMS usage (mixed sensor scenarios)
    \item Complete list of FCR radar modes in A-A Master Mode (STT, PRH, ACM, DGFT, others?)
    \item Complete list of TGP operational modes in A-G Master Mode (POINT, AREA, PICTURE, STANDBY, others?)
    \item Complete list of RWR display format options
    \item Complete list of navigation displays available in NAV Master Mode
    \item Obtain/create F-16 stick photo showing DMS location (compare with CMS image from Chapter 5)
    \item Verify cross-references to Chapter 3 (TMS context) and Chapter 5 (CMS)
\end{enumerate}

\subsection*{Open Questions}

\begin{enumerate}
    \item Does SWAP have a meaningful definition in BMS context? Original plan mentioned it but never defined. Refined structure eliminates it. Acceptable?
    \item Should we include tactical scenario examples (similar to CMS Section 5.2.4)? Proposed location: end of Section 4.3 or beginning of 4.4.
    \item Are there DMS behaviors in DGFT (Dogfight) Master Mode warranting separate 4.3.5 subsection, or covered sufficiently in 4.3.1 (FCR A-A)?
    \item Should we document DMS edge cases or BMS-specific quirks (e.g., format wraparound, SOI unavailability in certain modes)?
\end{enumerate}

\subsection*{Integration Notes}

\begin{itemize}
    \item This WIP file provides canonical structure and content outline for Chapter 4 DMS
    \item All hotastables are templates; actual content rows must be researched from \Dash34{} and \TRN{}
    \item Narratives are outlined with TODO markers; authors should write 250--500 words per indicated target length
    \item Once complete, this structure will be split into multiple WIP files per section (following wip-naming-v1.4 convention)
    \item Target integration: v0.3.0.0 of the guide (once all narrative content is drafted, reviewed, and validated against \Dash34{} and BMS Training Manual)
\end{itemize}

\end{document}

%===============================================================================
% END OF DMS CHAPTER 4 WIP STRUCTURE
% File: section-C4-S1-to-S5-dms-chapter-structure-dev-2026-01-12.tex
% Status: dev (Structural skeleton, ready for content development)
%===============================================================================
