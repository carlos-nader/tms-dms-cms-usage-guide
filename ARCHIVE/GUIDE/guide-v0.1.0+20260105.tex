\documentclass[11pt,a4paper]{report}

% --------------------------------------------------------------------------
% BASIC ENCODING AND LANGUAGE
% --------------------------------------------------------------------------
\usepackage[utf8]{inputenc}
\usepackage[T1]{fontenc}
\usepackage[english]{babel}

% --------------------------------------------------------------------------
% FONTS AND MICROTYPOGRAPHY
% --------------------------------------------------------------------------
\usepackage{lmodern} % Latin Modern fonts
\usepackage{microtype} % Better kerning and justification

% --------------------------------------------------------------------------
% PAGE GEOMETRY AND LAYOUT
% --------------------------------------------------------------------------
\usepackage{geometry}
\geometry{
a4paper,
left=2.5cm,
right=2.5cm,
top=2.5cm,
bottom=2.5cm
}

\usepackage{setspace}
\onehalfspacing % Slightly looser for readability

% --------------------------------------------------------------------------
% COLORS AND LINKS
% --------------------------------------------------------------------------
\usepackage{xcolor}
\definecolor{linkblue}{HTML}{004488}
\definecolor{linkred}{HTML}{882222}

\usepackage[
pdfencoding=auto,
psdextra,
colorlinks=true,
linkcolor=linkblue,
citecolor=linkred,
urlcolor=linkblue,
breaklinks=true
]{hyperref}
\usepackage{bookmark} % Better PDF bookmarks

% --------------------------------------------------------------------------
% HEADERS AND FOOTERS
% --------------------------------------------------------------------------
\usepackage{fancyhdr}
\setlength{\headheight}{15pt} % FIX: fancyhdr warning about headheight
\pagestyle{fancy}
\fancyhf{} % Clear all
\fancyhead[L]{\leftmark} % Chapter title on left
\fancyhead[R]{\rightmark} % Section title on right
\fancyfoot[C]{\thepage} % Page number center
\renewcommand{\headrulewidth}{0.4pt}
\renewcommand{\footrulewidth}{0pt}

% --------------------------------------------------------------------------
% TABLES
% --------------------------------------------------------------------------
\usepackage{booktabs}
\usepackage{array}
\usepackage{longtable}
\usepackage{tabularx}

\newcolumntype{L}[1]{>{\raggedright\arraybackslash}p{#1}}
\newcolumntype{C}[1]{>{\centering\arraybackslash}p{#1}}
\newcolumntype{R}[1]{>{\raggedleft\arraybackslash}p{#1}}

% HOTAS table environment: standard columns for TMS/DMS/CMS
\newenvironment{hotastable}[1][]{
\begin{longtable}{L{2.8cm} L{1.6cm} L{2.0cm} L{3.0cm} L{4.0cm} L{2.0cm} L{2.4cm}}
\caption{#1}\\
\toprule
\textbf{Context} &
\textbf{Direction} &
\textbf{Action} &
\textbf{Function} &
\textbf{Effect / notes} &
\textbf{Dash-34 ref.} &
\textbf{Training ref.} \\
\midrule
\endfirsthead
\toprule
\textbf{Context} &
\textbf{Direction} &
\textbf{Action} &
\textbf{Function} &
\textbf{Effect / notes} &
\textbf{Dash-34 ref.} &
\textbf{Training ref.} \\
\midrule
\endhead
\midrule
\multicolumn{7}{r}{\small\emph{Continued on next page}}\\
\bottomrule
\endfoot
\bottomrule
\endlastfoot
}{
\end{longtable}
}

% --------------------------------------------------------------------------
% SIMPLE REFERENCE MACROS FOR BMS DOCS
% --------------------------------------------------------------------------
\newcommand{\dashref}[1]{Dash-34, section~#1}
\newcommand{\dashone}[1]{Dash-1, section~#1}
\newcommand{\trnref}[1]{BMS Training Mission~#1}
\newcommand{\trnman}{BMS Training Manual 4.38.1}
\newcommand{\bmsver}{Falcon BMS~4.38.1}

% --------------------------------------------------------------------------
% VERSION CONTROL MACROS
% --------------------------------------------------------------------------
% Naming convention: guide-v{VERSION}+{BUILD}.tex
% For 0.x versions: MAJOR.MINOR.PATCH+BUILD (no status suffix)
% For 1.0.0+: MAJOR.MINOR.PATCH-STATUS+BUILD (status: RC1, RC2, Stable)

\newcommand{\docversion}{0.1.0}
\newcommand{\docbuild}{20260105}
\newcommand{\docstartdate}{05 January 2026}
\newcommand{\docenddate}{DD MMM 2026}
\newcommand{\chapterscompletedof}{1/7}
\newcommand{\tablesfilledpct}{0\%}
\newcommand{\fulldocversion}{\docversion+\docbuild}

% --------------------------------------------------------------------------
% GRAPHICS (FOR FUTURE HOTAS FIGURES)
% --------------------------------------------------------------------------
\usepackage{graphicx}
\graphicspath{{fig/}} % directory where you will put your images

% --------------------------------------------------------------------------
% TITLE
% --------------------------------------------------------------------------
\title{TMS, DMS and CMS Usage Guide for \bmsver}
\author{Carlos "Metal" Nader}
\date{Version \fulldocversion \\
      Progress: Chapters \chapterscompletedof{} | Tables \tablesfilledpct \\
      January 2026}

% --------------------------------------------------------------------------
% DOCUMENT
% --------------------------------------------------------------------------
\begin{document}

\maketitle
\pagenumbering{roman}
\tableofcontents
\clearpage
\pagenumbering{arabic}

% --------------------------------------------------------------------------
% CHAPTER 1: INTRODUCTION
% --------------------------------------------------------------------------
\chapter{Introduction}

This document is a community-made reference guide for \bmsver, focused on practical use of three specific HOTAS controls: the Target Management Switch (TMS), the Display Management Switch (DMS), and the Countermeasures Management Switch (CMS). Although this guide was developed under Falcon BMS 4.38.1, the fundamental behavior of these switches has remained constant since at least Falcon BMS 4.36, making this guide applicable to almost any player of Falcon BMS. Although other controls exist on the F-16 throttle and stick---such as the Communication Switch, the Dogfight/MRM Override, and the RDR Cursor Enable control---they are mentioned only when essential to understanding the behavior and context of TMS, DMS, and CMS.\footnote{Falcon BMS core avionics and weapons behaviour are documented in \dashone{2} and \dashref{2}. \trnman{} describes how these systems are trained in practice.} Its goal is to reorganize information that is spread across the Dash-1, Dash-34 and the BMS Training Manual into mode-based tables and short explanations, so that virtual pilots can quickly understand what each switch press does in a given context.\footnote{See \dashref{2.1.5} (Hands-On Controls) and the foreword of \trnman{} for the role of TMS, DMS and CMS in BMS training.}

This is the \emph{TMS, DMS and CMS Usage Guide -- Version \fulldocversion}, prepared between \docstartdate{} and \docenddate{}, and created with extensive assistance from an AI language model (Perplexity AI) to help structure, cross-reference and format the material. The human author remains fully responsible for every choice of content, interpretation and final wording, and any mistakes or omissions are attributable to the author alone, not to the AI system.

This work is entirely unofficial. The author is not affiliated with Benchmark Sims, MicroProse, any real-world air force, or any aircraft or weapons manufacturer. All interpretations, simplifications, errors and omissions in this document are solely the responsibility of the author and must not be attributed to the Falcon BMS development team or to any real-world organization.\footnote{Compare the official disclaimer and copyright statements in the foreword of \trnman{}.} Nothing in this document should ever be used for real-world operations, training, or procedures.

Readers are explicitly permitted to copy, share, translate, and modify this document for non-commercial use within the Falcon BMS community, as long as proper credit is given to the original author and no derivative work claims any official status. You are encouraged to extend it with additional aircraft, weapons, examples, or national variants, or to correct any mistakes you may find, so that the community benefits from continuous improvement.

The structure of the document mirrors the way BMS itself presents the jet: it starts with general HOTAS concepts (TMS, DMS, CMS and related switches), then covers Air-to-Air radar and IFF usage, followed by Air-to-Ground sensors and SPI logic, and finally weapon-specific employment (Maverick, IAMs, LGBs, HARM, Harpoon, SPICE and others), with variant notes for different F-16 blocks and export models.\footnote{See the organization of \dashref{2} and the weapon employment sections of \trnman{}.} Each section includes cross-references to the relevant chapters of the official BMS manuals and, where applicable, to specific BMS training missions that let you practice the techniques described.\footnote{Training missions and their learning objectives are listed in the table of contents of \trnman{}.}

% --------------------------------------------------------------------------
% DEVELOPMENT TIMELINE SECTION
% --------------------------------------------------------------------------
\section{Development timeline and status}

This guide was developed in structured phases, beginning \docstartdate{}. Current development status and targets are shown in Table~1.1 below.

\begin{center}
\begin{tabular}{L{3.2cm} L{2.5cm} L{2.5cm}}
\toprule
\textbf{Metric} & \textbf{Current} & \textbf{Target} \\
\midrule
Start Date & \docstartdate & — \\
Current Version & \fulldocversion & — \\
Chapters Complete & \chapterscompletedof & 7/7 \\
Tables Filled & \tablesfilledpct & 100\% \\
\bottomrule
\end{tabular}
\end{center}

The development roadmap is structured in three phases: (1) \emph{Chapter scaffolding} (Versions 0.1.0–0.7.0), during which all chapters receive narrative content and table structures; (2) \emph{Table population} (Versions 0.7.1–0.7.5), during which all tables are filled with complete HOTAS behavior descriptions and diagrams are generated; and (3) \emph{Review and release} (Versions 1.0.0-RC1 through 1.0.0-Stable), during which content is reviewed for accuracy, consistency, and clarity. Each phase produces a versioned PDF artifact, and all versions are archived for traceability.

% --------------------------------------------------------------------------
% SECTION 1.1: SCOPE AND PURPOSE
% --------------------------------------------------------------------------
\section{Scope and purpose}

This guide focuses exclusively on three HOTAS switches: the Target Management Switch (TMS), the Display Management Switch (DMS), and the Countermeasures Management Switch (CMS). Although other controls exist on the F-16 throttle and stick---such as the Communication Switch, the Dogfight/MRM Override, the RDR Cursor Enable control, and others---they are mentioned only when essential to understanding the behavior and context of TMS, DMS, and CMS.

This is not a comprehensive HOTAS or avionics manual. Instead, it is a usage guide organized by context, with emphasis on practical tables that show what each switch input does in specific flight modes, sensor configurations, and weapon employment scenarios. The guide bridges information scattered across official documentation and training missions, making it immediately accessible to pilots who ask: \emph{``In this radar mode, what does TMS Up do?''} or \emph{``How do I cycle through MFD formats with the DMS?''}

The guide assumes knowledge of basic F-16 operation and familiarity with master modes (NAV, A-A, A-G, DGFT). It does not replace the Dash-34 or Training Manual; rather, it complements them by organizing TMS/DMS/CMS behavior into searchable tables with cross-references back to official sources and practical training missions where each behavior can be practiced.

% --------------------------------------------------------------------------
% SECTION 1.2: VERSION, AUTHORSHIP AND AI ASSISTANCE
% --------------------------------------------------------------------------
\section{Version, authorship and AI assistance}

\textbf{Document Version:} \fulldocversion{} (Progress: Chapters \chapterscompletedof{} | Tables \tablesfilledpct)

\textbf{Falcon BMS Version:} 4.38.1 (Update 1)

\textbf{Authorship:} This guide was created by a member of the Falcon BMS community with structured assistance from AI language models (Perplexity AI). The human author identified scope, validated content against official Falcon BMS documentation, made all organizational and editorial decisions, and bears full responsibility for the guide's accuracy and presentation. AI tools were used for research organization, cross-referencing, and text generation---not for defining technical correctness.

\textbf{Disclaimer:} This is an unofficial, community-made document not affiliated with, endorsed by, or affiliated with Benchmark Sims, MicroProse, any military organization, or any aircraft or weapons manufacturer. All technical content is paraphrased in original words from official BMS documentation. No copyrighted material is reproduced directly. The guide is provided ``as-is'' for educational and simulation training purposes only.

\textbf{Copyright \& Sharing:} This guide may be freely copied, printed, translated, and shared within the Falcon BMS community for non-commercial use. Derivative works and contributions are encouraged, provided that proper credit is given to the original author and no derivative version claims official status.

% --------------------------------------------------------------------------
% SECTION 1.3: SOURCES AND REFERENCES
% --------------------------------------------------------------------------
\section{Sources and references}

This guide is based on the following primary Falcon BMS documents consulted during research and development:

\begin{enumerate}
\item \textbf{TO BMS 1F-16CMAM-34-1-1} (Dash-34, Change 4.38) -- Avionics and Nonnuclear Weapons Delivery Flight Manual
  \begin{itemize}
    \item Sections 2.1.5 (HOTAS Hands-On Controls)
    \item Sections 2.3.1 (ANAPG-68V5 Fire Control Radar)
    \item Section 2.7 (Defensive Avionics -- CMS, ECM, CMDS)
    \item Weapon-specific chapters (HARM, Maverick, IAMs, LGBs, Harpoon, SPICE, etc.)
  \end{itemize}

\item \textbf{BMS Training Manual 4.38.1} (October 2025) -- Training missions and learning objectives
  \begin{itemize}
    \item Individual mission descriptions (TRN 11--28, with emphasis on weapons employment and avionics training)
    \item Mission learning objectives and practical procedures
  \end{itemize}

\item \textbf{TO BMS 1F-16CMAM-1} (Dash-1) -- F-16 Aircraft Systems, Normal and Abnormal Procedures
  \begin{itemize}
    \item Referenced for overall aircraft context and system interactions
  \end{itemize}

\item \textbf{BMS User Manual 4.38} -- BMS user interface and setup
  \begin{itemize}
    \item Referenced for MFD, ICP, and UFC control information
  \end{itemize}

\item \textbf{MCH 11-F16 Vol 5} (May 1996) -- F-16 Flight Manual Vol 5 (Surface Attack and Weapons Delivery)
  \begin{itemize}
    \item Referenced for operational context and ordnance procedures
  \end{itemize}

\item \textbf{Falcon BMS Cockpit Arrangement Diagrams} (Multiple blocks)
  \begin{itemize}
    \item F-16C Block 50/52, Block 40/42, Block 30/32, F-16A MLU variants
    \item Visual reference for HOTAS switch positions across variants
  \end{itemize}
\end{enumerate}

\noindent\emph{Note:} This reference list will be updated throughout the development of the guide as new sources are consulted. Always refer to the most current version of this document to see the complete list of references.

% --------------------------------------------------------------------------
% SECTION 1.4: DOCUMENT STRUCTURE AND HOW TO READ IT
% --------------------------------------------------------------------------
\section{Document structure and how to read it}

\subsection{Part A: Foundational Chapters (2--3)}

Chapter~2 establishes core HOTAS concepts: Sensor of Interest (SOI), short vs.\ long press timing, master modes, and an overview of TMS/DMS/CMS roles. Read this first if you are new to the F-16 or HOTAS in general.

\subsection{Part B: Switch-Specific Chapters (3--5)}

Chapters~3, 4, and 5 each focus on one switch and contain detailed tables using the \texttt{hotastable} environment.

\subsubsection{Table structure}

Each table follows a seven-column format:

\begin{center}
\begin{tabular}{L{2.2cm} L{1.6cm} L{1.6cm} L{2.2cm} L{3.2cm} L{1.9cm} L{1.8cm}}
\toprule
\textbf{Context} & \textbf{Direction} & \textbf{Action} & \textbf{Function} & \textbf{Effect / notes} & \textbf{Dash-34 ref.} & \textbf{Training ref.} \\
\midrule
Specifies master mode + sensor/weapon (e.g., ``A-A CRM -- FCR RWS'') & Physical switch direction: Up, Down, Left, Right & Type of press: Short, Long, Short repeated, or Long (hold) & Brief name for what the switch does & 1--3 sentence explanation of system behavior; includes nuances and conditions & Section reference(s) in Dash-34 for detailed documentation & BMS training mission(s) where behavior can be practiced \\
\bottomrule
\end{tabular}
\end{center}

\subsubsection{How to find information}

\begin{enumerate}
\item Identify the \textbf{master mode and sensor/weapon context} from the leftmost ``Context'' column (e.g., ``A-G -- TGP'' for Targeting Pod in Air-to-Ground mode).
\item Find the \textbf{direction} you intend to press (Up, Down, Left, Right).
\item Determine whether you will use a \textbf{short or long press} (consult the ``Action'' column if unsure).
\item Read the \textbf{Function} and \textbf{Effect / notes} columns to understand what happens.
\item Use \textbf{Dash-34 ref.} to read the official specification in detail.
\item Use \textbf{Training ref.} to load a practice mission and test the behavior under realistic conditions.
\end{enumerate}

\subsection{Part C: Training and Visual Reference (Chapters 6--7)}

\textbf{Chapter~6} links this guide to the 33 BMS training missions, offering a recommended progression and example tactical flows. Use this chapter to plan your training sequence.

\textbf{Chapter~7} provides schematic diagrams of the TMS, DMS, and CMS hats with arrows and short labels for each direction in common contexts. These are quick-reference visuals; always consult the tables for complete behavior descriptions.

\subsection{Part D: Appendices}

\textbf{Appendix~A} notes any differences in TMS/DMS/CMS behavior across F-16 blocks and variants (e.g., Block 50/52 vs.\ Block 40/42).

\textbf{Appendix~B} provides a comprehensive index of all major tables and their locations.

% --------------------------------------------------------------------------
% CHAPTER 2: HOTAS FUNDAMENTALS
% --------------------------------------------------------------------------
\chapter{HOTAS fundamentals}

\section{Sensor of Interest (SOI) and display logic}

\section{Short vs long presses and timing}

\section{Master modes and context-sensitive behaviour}

\section{Overview of TMS, DMS and CMS}

% --------------------------------------------------------------------------
% CHAPTER 3: CMS
% --------------------------------------------------------------------------
\chapter{CMS -- Countermeasures Management Switch}

\section{Concept and interaction with CMDS / ECM}

\section{CMS switch actions (all modes)}

\section{CMS -- Block / variant notes}

% --------------------------------------------------------------------------
% CHAPTER 4: TMS
% --------------------------------------------------------------------------
\chapter{TMS -- Target Management Switch}

\section{Concept and general behaviour}

\section{TMS in Air-to-Air}

\subsection{FCR CRM (RWS / ULS / VSR)}

\subsection{SAM / DT-SAM}

\subsection{TWS}

\subsection{STT}

\subsection{ACM (30x20, 10x60, BORE, SLEW)}

\subsection{IFF interrogations (SCAN / LOS)}

\section{TMS in Air-to-Ground -- sensors and SPI}

\subsection{FCR A-G (GM / GMT / SEA / AGR)}

\subsection{TGP A-G}

\subsection{HUD / HMCS (SPI, Snowplow, CZ, VIP/VRP cues)}

\subsection{Markpoints and steerpoint management}

\section{TMS in A-G weapon employment}

\subsection{Unguided bombs and rockets (CCIP / CCRP / DTOS)}

\subsection{EO weapons -- Maverick (VIS / PRE / BORE)}

\subsection{IAMs (JDAM / JSOW / WCMD / SPICE / others)}

\subsection{LGBs and laser employment}

\subsection{Anti-radiation (HARM POS / HAS / HAD)}

\subsection{Naval weapons (Harpoon, others)}

\section{TMS -- Block / variant notes}

% --------------------------------------------------------------------------
% CHAPTER 5: DMS
% --------------------------------------------------------------------------
\chapter{DMS -- Display Management Switch}

\section{Concept and SOI control}

\section{DMS in Air-to-Air}

\subsection{MFD format cycling and SWAP}

\subsection{HUD / HMCS SOI behaviour}

\section{DMS in Air-to-Ground}

\subsection{Sensor handoff and SOI choreography}

\subsection{Special cases (IAM, HARM, Harpoon)}

\section{DMS -- Block / variant notes}

% --------------------------------------------------------------------------
% CHAPTER 6: TRAINING REFERENCES AND FLOWS
% --------------------------------------------------------------------------
\chapter{Training references and practical flows}

\section{How to use this guide with BMS training missions}

\section{Recommended progression}

\section{Example flows for typical missions}

\section{Checklist: what to practice next}

% --------------------------------------------------------------------------
% CHAPTER 7: HOTAS VISUAL REFERENCE
% --------------------------------------------------------------------------
\chapter{HOTAS visual reference}

\section{F-16 HOTAS overview}

\section{TMS diagrams}

\section{DMS diagrams}

\section{CMS diagrams}

% --------------------------------------------------------------------------
% APPENDICES
% --------------------------------------------------------------------------
\appendix

\chapter{Block / variant overview}

\section{F-16CM Block 50/52}

\section{F-16C/D Block 40/42}

\section{F-16AM/BM MLU}

\section{F-16I Sufa and Israeli variants}

\section{Other export variants}

\chapter{Tables index}

\section{TMS tables}

\section{DMS tables}

\section{CMS tables}

\end{document}
