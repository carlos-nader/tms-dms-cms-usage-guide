\documentclass[12pt,a4paper]{article}
\usepackage[utf8]{inputenc}
\usepackage[portuguese]{babel}
\usepackage{geometry}
\usepackage{hyperref}
\usepackage{xcolor}
\usepackage{fancyhdr}
\usepackage{booktabs}
\usepackage{array}
\usepackage{longtable}
\usepackage{float}

% Configuração de margens
\geometry{
    left=2cm,
    right=2cm,
    top=2.5cm,
    bottom=2.5cm,
    headheight=14.5pt,
    headsep=15pt
}

% Configuração de headers e footers
\pagestyle{fancy}
\fancyhf{}
\fancyhead[L]{Análise Técnica: TMS, DMS e CMS F-16}
\fancyhead[R]{\thepage}
\fancyfoot[C]{Documento Técnico Confidencial}
\renewcommand{\headrulewidth}{0.5pt}
\renewcommand{\footrulewidth}{0.5pt}

% Configuração de hiperlinks
\hypersetup{
    colorlinks=true,
    linkcolor=blue,
    urlcolor=blue,
    citecolor=blue,
    bookmarksnumbered=true,
    breaklinks=true
}

% Título e autor
\title{
    \textbf{Análise Técnica Comparativa:} \\
    \Large TMS, DMS e CMS no F-16 \\
    \normalsize Hierarquia de Complexidade Operacional
}
\author{
    Análise Técnica Integrada \\
    Documentação Militar Desclassificada + Simuladores de Alto Realismo
}
\date{07 de Janeiro de 2026}

\begin{document}

\maketitle

\newpage
\tableofcontents
\newpage

% ============================================================================
% SEÇÃO 1
% ============================================================================
\section{Introdução}

Os switches HOTAS (Hands-On Throttle and Stick) representam a interface primária do piloto de F-16 para gestão de sensores, designação de alvos e defesa eletrônica. Este documento estabelece uma hierarquia de complexidade operacional e funcional entre três elementos críticos do sistema:

\begin{itemize}
    \item \textbf{Target Management Switch (TMS)}
    \item \textbf{Display Management Switch (DMS)}
    \item \textbf{Countermeasures Management Switch (CMS)}
\end{itemize}

A análise é fundamentada em documentação técnica oficial (TO 1F-16CMAM-34-1-1, MCH 11-F16 Vol.~5), manuais de simulação de alto realismo (DCS, Falcon BMS) e referências de treinamento avançado. A validação cruza múltiplas fontes independentes para garantir robustez e precisão das conclusões.

\vspace{1cm}

% ============================================================================
% SEÇÃO 2
% ============================================================================
\section{Definições Funcionais e Responsabilidades}

\subsection{Target Management Switch (TMS)}

\textbf{Definição:} Elemento de controle de 4 vias (Forward/Right/Aft/Left) no manche, momentâneo, responsável pela designação de alvos, seleção de Sensor of Interest para designação e controle de múltiplos estados operacionais de radar e sensores electroópticos.\cite{TO1F16CMAM34,MCH11F16Vol5,DCSWikiF16,F16Simulator,Falconpedia}

\textbf{Responsabilidades Funcionais:}

\begin{enumerate}
    \item Designação e ``bugging'' (marcação prioritária) de alvos em modos RWS/ULS/VSR/SAM/TWS
    \item Transição entre submodos de radar (RWS $\rightarrow$ SAM $\rightarrow$ DT SAM $\rightarrow$ STT; Dogfight ACM)
    \item Designação visual em HUD (DTOS, EO Visual, IAM Visual)
    \item Designação HMCS com escravamento integrado
    \item Stepping de TOI (Target of Interest) em TWS/SAM
    \item Sighting option management (STPT/OA1/OA2/VIP/VRP) em A-G
    \item Hand-off entre sensores (TGP $\leftrightarrow$ AGM-65/Maverick)
    \item Spotlight search em RWS/TWS
\end{enumerate}

\subsection{Display Management Switch (DMS)}

\textbf{Definição:} Elemento de controle de 4 vias (Forward/Right/Aft/Left) no manche, momentâneo, responsável pela seleção de Sensor of Interest (SOI) entre HUD/HMCS e MFD esquerdo/direito, além do stepping de formatos de página nos displays multifuncionais.\cite{TO1F16CMAM34,DCSWikiF16,DCSSOITutorial,Falconpedia2}

\textbf{Responsabilidades Funcionais:}

\begin{enumerate}
    \item Seleção de HUD/HMCS como SOI (DMS Forward)
    \item Seleção de MFD esquerdo/direito como SOI (DMS Aft/Down)
    \item Alternância entre MFDs (DMS Aft cíclico)
    \item Stepping de formatos (FCR, HSD, TGP, SMS, HAD, WPN) em MFD (DMS Left/Right)
    \item Habilitação de cursores e controles HOTAS associados ao SOI selecionado
\end{enumerate}

\subsection{Countermeasures Management Switch (CMS)}

\textbf{Definição:} Elemento de controle de 4 vias (Forward/Right/Aft/Left) no manche, momentâneo, responsável pela concessão de consentimento para operação de ECM (Electronic Countermeasures) ativo e pelo comando de programas selecionáveis de chaff/flare do sistema ALE-47.\cite{TO1F16CMAM34,BMSDash34,BMSEWSManual,DCSCMDS}

\textbf{Responsabilidades Funcionais:}

\begin{enumerate}
    \item Consentimento para operação ECM (pods de jamming, ATD/ASPIS)
    \item Consentimento para operação ALE-50 (quando modelado)
    \item Comando de programas selecionáveis ALE-47 (Programas 1-4, 6)
    \item Seleção de modo operacional CMDS (AUTO/SEMI/MAN/BYP)
    \item Cancelamento de operação automática e retorno a standby
    \item Autorização de transmissão/radiação de ECM em condições de voo
\end{enumerate}

\newpage

% ============================================================================
% SEÇÃO 3: COMPLEXIDADE TMS
% ============================================================================
\section{Complexidade Operacional: TMS}

\subsection{Fundamentos de Complexidade Estrutural}

O TMS opera em uma máquina de estados não-trivial, com ramificações significativas baseadas em:

\begin{enumerate}
    \item \textbf{Master Mode ativo} (A-A, A-G, NAV)
    \item \textbf{Submodo de sensor primário} (RWS, SAM, TWS, STT, ACM, DTOS, VIS)
    \item \textbf{Sensor of Interest} (FCR, HUD, TGP, HSD, WPN)
    \item \textbf{Histórico de comando anterior} (sequências de TMS Forward/Aft determinam transições)
\end{enumerate}

Essa dependência condicional cria aproximadamente 4 comportamentos distintos para cada posição do TMS (Forward/Right/Aft/Left) em diferentes contextos operacionais.\cite{TO1F16CMAM34,MCH11F16Vol5,DCSWikiF16,BMS437}

\subsection{Operações de Complexidade Elevada: Gestão Multi-Alvo em A-A BVR}

\subsubsection{Cadeia CRM/SAM/TWS/STT}

O fluxo operacional segue estas transições:\cite{TO1F16CMAM34,DCSWikiF16,DCSTWSTutorial,BMS437}

\begin{verbatim}
RWS (Search)
  [TMS Forward sobre contact]
ST SAM (Single-Target SAM, 1 TOI)
  [TMS Forward sobre TOI ou dupla designação]
STT (Single Target Track, alta precisão)
  [TMS Down (Return-to-Search)]
RWS
\end{verbatim}

Alternativamente, em modo dual-target:

\begin{verbatim}
RWS
  [TMS Forward sobre segundo contact]
DT SAM (Dual-Target SAM, 2 TOI com search)
  [TMS Right curto: stepping de TOI]
DT SAM (TOI alternado)
  [TMS Forward sobre novo TOI]
STT
\end{verbatim}

E em modo TWS (Track While Scan):

\begin{verbatim}
RWS/ULS
  [TMS Right longo (>0.8s)]
TWS (até ~10 track files com extrapolação)
  [TMS Forward sobre track file]
Bugged TOI (em TWS)
  [TMS Right curto]
Step para próximo track file
  [TMS Forward]
STT (on bugged target)
\end{verbatim}

\textbf{Fatores de complexidade:}\cite{TO1F16CMAM34,DCSWikiF16,DCSTWSTutorial}

\begin{itemize}
    \item Cada transição possui condições de ativação distintas
    \item Extrapolação de track files em TWS (máximo 13 segundos sem paint)
    \item Priorização automática de tracks (range, ordem de construção)
    \item Perda de track em clutter Doppler (``COAST'' indicado no HUD/HMCS)
    \item Sincronização com RWS reflex e update rates variáveis por modo
\end{itemize}

\subsubsection{Integração ACM (Air Combat Mode) com HMCS}

Em modo ACM, o TMS seleciona submodos de scan e gerencia lock verificado:\cite{TO1F16CMAM34,DCSWikiF16,DCSTutorialACM}

\begin{itemize}
    \item \textbf{TMS Forward hold em BORE:} inibe aquisição automática, permite escravamento à LOS do HMCS
    \item \textbf{TMS Right:} cicla entre submodos (30×20 corpo-referido, 10×60, BORE, SLEW)
    \item \textbf{TMS Aft:} quebra lock e volta a 30×20 ou NO RAD
\end{itemize}

A recomendação tática do Vol. 5 é explícita: verificar visualmente o alvo antes de usar lock automático em ambiente visualmente denso (flight, múltiplas aeronaves), exatamente porque a má-aplicação de TMS em ACM pode resultar em lock no membro errado da formatura.\cite{MCH11F16Vol5}

\subsection{Operações de Complexidade Elevada: Gestão de SPI e Sighting Options em A-G}

\subsubsection{Ciclo de Sighting Options}

\textbf{TMS Right no FCR/TGP como SOI} cicla entre:\cite{TO1F16CMAM34,BMS437}

\begin{itemize}
    \item \textbf{STPT (Steerpoint):} geometria baseada em coordenadas de navegação
    \item \textbf{OA1/OA2 (Offset Aimpoints):} pontos alternativos de visada offset do SPI primário
    \item \textbf{VIP (Visual IP):} Initial Point visual, criando geometria IP~$\rightarrow$~TGT
    \item \textbf{VRP (Visual Reference Point):} ponto visual de referência para PUP (Pop-Up) e ataques coordenados
\end{itemize}

Cada sighting option altera a solução CCRP (Continuously Computed Release Point) e o steering do HUD para armas (AGM-65, bomba CCRP, etc.). A seleção correta é crítica para a geometria de entrada em ataques preplanned.\cite{TO1F16CMAM34,MCH11F16Vol5,BMS437}

\subsubsection{Designação Visual em Modo DTOS}

\textbf{DTOS} (Designate-Then-Offset-Steerpoint):\cite{TO1F16CMAM34,DCSWikiF16,BMS437}

\begin{itemize}
    \item HUD SOI: \textbf{TMS Forward} designa visualmente, ground-estabiliza o alvo e cria um SPI
    \item \textbf{TMS Forward hold} (>0.5 s): transfere o TD box (Target Designator box) para LOS do HMCS
    \item \textbf{TMS Aft:} cancela designação
\end{itemize}

\subsection{Operações de Complexidade Intermediária: Hand-off Sensor}

Transição TGP~$\rightarrow$~AGM-65: Em cenários onde o TGP faz point-track e o AGM-65 precisa ser escravado ao alvo, o TMS Right permite re-tentar o hand-off quando o seeker do Maverick não correlaciona com o alvo primeiramente designado.\cite{TO1F16CMAM34,MCH11F16Vol5,BMS437}

\subsection{Operações de Complexidade Reduzida: Lock Básico e Spotlight}

\begin{itemize}
    \item \textbf{Lock básico:} slew ACQ cursor sobre contact, \textbf{TMS Forward} $\rightarrow$ SAM, \textbf{TMS Aft} $\rightarrow$ voltar a RWS
    \item \textbf{Spotlight search:} \textbf{TMS Forward longo} (>1 s) em RWS/TWS ativa scan 4-bar~×~10° centrado no cursor
\end{itemize}

Essas operações são conceitualmente diretas, embora suportem táticas sofisticadas em formação e BVR.\cite{TO1F16CMAM34,DCSWikiF16,DCSTWSTutorial,BMS437}

\subsection{Síntese: TMS como Maior Complexidade}

\textbf{Ordem de complexidade operacional do TMS:}

\begin{enumerate}
    \item \textbf{Nível máximo:} Gestão multi-alvo em CRM/SAM/DT~SAM/TWS com transições condicionais e extrapolação de tracks
    \item \textbf{Nível alto:} ACM com HMCS e verificação de lock; gestão de sighting options em A-G
    \item \textbf{Nível médio:} Designação visual (DTOS/EO-VIS) com HMCS; hand-off TGP~$\leftrightarrow$~AGM-65
    \item \textbf{Nível baixo:} Lock básico, RTS, spotlight search
\end{enumerate}

O fator diferenciador é a \textbf{profundidade de máquina de estados} que o TMS controla: modos mutuamente exclusivos com transições não-triviais, histórico de comandos que afeta comportamento futuro, e forte acoplamento tático.\cite{TO1F16CMAM34,MCH11F16Vol5,DCSWikiF16,Falconpedia}

\newpage

% ============================================================================
% SEÇÃO 4: COMPLEXIDADE DMS
% ============================================================================
\section{Complexidade Operacional: DMS}

\subsection{Fundamentos de Complexidade Estrutural}

O DMS opera em uma lógica determinística e quase-linear:\cite{TO1F16CMAM34,DCSWikiF16,DCSSOITutorial,Falconpedia2}

\begin{enumerate}
    \item \textbf{Forward:} tenta elevar SOI para HUD (master-mode dependente)
    \item \textbf{Aft/Down:} move SOI para MFDs, alternando entre esquerdo/direito
    \item \textbf{Left/Right:} stepping de páginas/formatos dentro do MFD já selecionado
\end{enumerate}

Não há condicionamento cruzado significativo: o estado atual (Forward/Aft/Left/Right) não muda substancialmente o comportamento futuro do DMS.\cite{TO1F16CMAM34,DCSWikiF16}

\subsection{Restrições Operacionais}

Certos modos (MARK OFLY, SP/Snowplow pré-designate) não permitem alteração de SOI com DMS, mas essa é uma restrição de ``permissão'', não uma mudança de lógica.\cite{TO1F16CMAM34,BMS437}

Em A-A, SOI fica restrito a FCR/HSD/TGP (não pode ser HUD em A-A).\cite{TO1F16CMAM34}

\subsection{Usos Operacionais}

\subsubsection{Seleção de SOI em Cenários Multi-Sensor}

Exemplo típico de A-G preplanned:\cite{TO1F16CMAM34,BMS437}

\begin{enumerate}
    \item \textbf{FCR SOI} (DMS Forward): localizar área geral com radar
    \item \textbf{DMS Aft/Down} $\rightarrow$ \textbf{TGP SOI:} refinar identificação no FLIR/EO
    \item \textbf{DMS Forward} $\rightarrow$ \textbf{HUD SOI:} executar DTOS visual com steering CCRP
\end{enumerate}

Essa sequência é cognitivamente intuitiva e não envolve máquina de estados complexa.

\subsubsection{Suporte a Data Link A-G}

HSD deve ser SOI (DMS Aft) para que o cursor responda a CURSOR-ENABLE. O piloto coloca o cursor no steerpoint desejado e \textbf{TMS Forward} para designá-lo como destino de IDM transmit.\cite{TO1F16CMAM34,BMS437}

\subsection{Síntese: DMS como Menor Complexidade}

O DMS é operacionalmente simples porque:

\begin{itemize}
    \item \textbf{Responsabilidade singular:} roteamento de comandos HOTAS para o sensor/display correto
    \item \textbf{Lógica determinística:} Forward/Aft/Left/Right mapeiam diretamente a HUD, MFDs e páginas
    \item \textbf{Sem interdependência tática profunda:} mudança de SOI não altera capacidades fundamentais de arma ou radar
\end{itemize}

Por isso, mesmo sendo \textbf{criticamente necessário}, o DMS é estruturalmente o \textbf{menos complexo} dos três switches.\cite{TO1F16CMAM34,DCSWikiF16,DCSSOITutorial,Falconpedia2}

\newpage

% ============================================================================
% SEÇÃO 5: COMPLEXIDADE CMS
% ============================================================================
\section{Complexidade Operacional: CMS}

\subsection{Fundamentos de Complexidade Estrutural}

O CMS gerencia dois sistemas integrados com lógica semi-independente:\cite{TO1F16CMAM34,BMSDash34,BMSEWSManual,DCSSemiAutoTutorial}

\begin{enumerate}
    \item \textbf{ECM} (Electronic Countermeasures): pods de jamming, ATD/ASPIS, ALE-50
    \item \textbf{CMDS} (ALE-47): dispensador de chaff/flare com 6 programas e 4 modos operacionais
\end{enumerate}

A complexidade surge da interação entre consentimento do piloto, seleção automática de programa por RWR/EWS e restrições de voo/segurança.\cite{TO1F16CMAM34,BMSDash34,DCSSemiAutoTutorial,DCSAutoModeVideo}

\subsection{Operações de Complexidade Elevada: Integração ECM + ALE-47 com Consentimento}

\subsubsection{Modo SEMI (Semi-Automático)}

\textbf{Fluxo operacional:}\cite{TO1F16CMAM34,BMSDash34,BMSEWSManual,DCSSemiAutoTutorial}

\begin{enumerate}
    \item RWR/EWS analisa ameaças e julga necessidade de dispense
    \item Sistema acende \textbf{DISPENSE RDY} e gera voz \textbf{``COUNTER''}
    \item Piloto pressiona \textbf{CMS Aft} (consentimento)
    \item Programa selecionado (1-4 ou 6) executa \textbf{uma única vez}
    \item Se ameaça persiste, sistema pede consentimento novamente
\end{enumerate}

\subsubsection{Modo AUTO (Automático)}

\textbf{Fluxo operacional:}\cite{TO1F16CMAM34,BMSDash34,DCSAutoModeVideo,DCSEarlyAccessGuide}

\begin{enumerate}
    \item Piloto pressiona \textbf{CMS Aft} uma vez (consentimento persistente)
    \item Sistema entra em \textbf{AUTO dispensing:} executa programa(s) repetidamente conforme RWR/EWS julga necessário
    \item Dispense continua até:
    \begin{itemize}
        \item \textbf{CMS Right:} piloto cancela consentimento $\rightarrow$ ECM standby
        \item Atingimento de bingo quantity: CMDS pede novo consentimento se continuar dispensando
        \item Esgotamento de expendables
    \end{itemize}
\end{enumerate}

\subsubsection{Interação com RF Switch e Condições de Voo}

Mesmo com \textbf{CMS Aft} dado:\cite{TO1F16CMAM34,BMSDash34}

\begin{itemize}
    \item \textbf{RF em QUIET/SILENT:} ATD/ECM passa a \textbf{standby} (NORM é requisito para transmissão)
    \item \textbf{Landing gear DOWN:} auto dispensing é inibido (segurança)
    \item \textbf{Ciclo MMC power:} estado AUTO é cancelado
\end{itemize}

Essas restrições criam uma máquina de estados acoplada entre CMS, RF~Switch e telemetria da aeronave.\cite{TO1F16CMAM34,BMSDash34}

\subsection{Operações de Complexidade Intermediária: Programas Manuais}

\subsubsection{Programas 1-4 (PRGM Rotary - Controláveis via CMS)}

Selecionados via knob PRGM no painel CMDS:\cite{web51,web12,BMSDash34,DCSManualModeVideo}

\begin{itemize}
    \item \textbf{PRGM Knob:} escolhe qual programa (1, 2, 3, ou 4) será executado
    \item \textbf{MAN mode:} cada \textbf{CMS Forward} executa o programa selecionado uma única vez
    \item \textbf{SEMI/AUTO:} programa selecionado é executado conforme RWR/EWS julga apropriado
\end{itemize}

Piloto mapeia taticamente via PRGM knob:

\begin{itemize}
    \item Programa 1: estratégia X (ex: chaff-heavy para SAM search)
    \item Programa 2: estratégia Y (ex: chaff-only)
    \item Programa 3: estratégia Z (ex: flare-only para A-A)
    \item Programa 4: estratégia W (ex: mix chaff-flare)
\end{itemize}

\subsubsection{Programa 5 (Independente - NÃO Controlado por CMS)}

Acionado EXCLUSIVAMENTE via \textbf{Slap Switch} (botão independente, lado esquerdo).\cite{web51,web12,web24,BMSDash34}

\textbf{IMPORTANTE:} Programa 5 \textbf{NÃO é selecionável via PRGM knob} e \textbf{NÃO é acionado por CMS}. É um controle independente, sempre disponível via Slap Switch, tipicamente mapeado como \textbf{mix rápido chaff+flare} para reação imediata a ameaças desconhecidas.

\subsubsection{Programa 6 (Controlável via CMS Left)}

Acionado via \textbf{CMS Left}.\cite{web51,web12,web24,BMSDash34}

\textbf{IMPORTANTE:} Programa 6 \textbf{NÃO é selecionável via PRGM knob}. É sempre acionado por CMS Left, tipicamente mapeado como \textbf{flare-only} para defesa contra armamentos A-A curto-alcance/MANPADS.

\subsubsection{Modo BYP (Bypass)}

Ignora programas configurados:\cite{TO1F16CMAM34,BMSDash34,BMSEWSManual}

\begin{itemize}
    \item Cada comando dispensa \textbf{exatamente 1 chaff + 1 flare}
    \item Útil para conservar expendables em fim-de-missão ou cenários de incerteza de ameaça
\end{itemize}

\subsection{Operações de Complexidade Reduzida: Cancelamento e BIT Solo}

\subsubsection{``Panic OFF''}

Um único \textbf{CMS Right} em qualquer contexto:\cite{TO1F16CMAM34,BMSDash34}

\begin{itemize}
    \item Remove consentimento automático de ECM
    \item Coloca ALE-47 em standby
    \item Cancela dispensação em curso
\end{itemize}

Útil para retorno a silêncio EMCON ou para freio rápido de gasto excessivo de CM.\cite{BMSDash34}

\subsubsection{BIT de Pod ECM no Solo}

Conforme BMS Dash-34 e MLU:\cite{BMSDash34,F16MLUDocumentation}

\begin{enumerate}
    \item Em solo, ECM padrão = \textbf{standby}
    \item Para BIT high-level: manter \textbf{CMS Aft}
    \item \textbf{ECM ENABLE light} acende; pod pode radiar
    \item Soltar \textbf{CMS Aft} retorna a standby; luz apaga
\end{enumerate}

\textbf{Advertência crítica:} se CMS Aft é mantido com pod em OPERATE, radiação contínua --- pessoal de solo deve estar afastado.\cite{BMSDash34,F16MLUDocumentation}

\subsection{Síntese: CMS como Complexidade Média/Alta}

O CMS é substancialmente mais complexo que um simples ``botão de flare'' porque:

\begin{enumerate}
    \item \textbf{Dupla responsabilidade:} ECM ativo + ALE-47 chaff/flare
    \item \textbf{Máquina de estados condicional:} SEMI vs. AUTO vs. MAN vs. BYP; consentimento persistente vs. pontual
    \item \textbf{Acoplamento com sistemas externos:} RWR/EWS (seleção de programa), RF~Switch (standby override), telemetria de voo (gear, etc.)
    \item \textbf{Implicações táticas de segurança:} radiação involuntária em solo, esgotamento desnecessário de CM, timing inadequado de dispensação
\end{enumerate}

\cite{TO1F16CMAM34,BMSDash34,BMSEWSManual,DCSSemiAutoTutorial,DCSAutoModeVideo,DCSEarlyAccessGuide}

\newpage

% ============================================================================
% SEÇÃO 6: COMPARAÇÃO GLOBAL
% ============================================================================
\section{Comparação Global: Hierarquia de Complexidade Operacional}

\subsection{Tabela Síntese}

\begin{table}[H]
\centering
\small
\begin{tabular}{|l|l|c|l|}
\hline
\textbf{Switch} & \textbf{Papel Dominante} & \textbf{Complexidade} & \textbf{Fator Diferenciador} \\
\hline
\textbf{TMS} & 
\begin{tabular}[t]{@{}l@{}}Gestão de alvos/SPI, \\ múltiplos submodos \\ de radar\end{tabular} & 
\textbf{Muito alta} & 
\begin{tabular}[t]{@{}l@{}}Máquina de estados \\ multi-nível; transições \\ condicionais; histórico\end{tabular} \\
\hline
\textbf{CMS} & 
\begin{tabular}[t]{@{}l@{}}Consentimento ECM \\ + ALE-47 \\ chaff/flare\end{tabular} & 
\textbf{Média/Alta} & 
\begin{tabular}[t]{@{}l@{}}Integração dois \\ sistemas; consentimento \\ persistente; restrições\end{tabular} \\
\hline
\textbf{DMS} & 
\begin{tabular}[t]{@{}l@{}}Seleção de SOI \\ e stepping \\ de páginas\end{tabular} & 
\textbf{Baixa} & 
\begin{tabular}[t]{@{}l@{}}Função determinística; \\ linear; sem \\ interdependência profunda\end{tabular} \\
\hline
\end{tabular}
\end{table}

\subsection{Conclusão Hierárquica}

\textbf{Do mais complexo ao mais simples:}

\begin{enumerate}
    \item \textbf{TMS --- Maior Complexidade:} Máquina de estados não-trivial com 4+ submodos simultâneos e transições condicionais; forte acoplamento com tática de engajamento; impacto direto em sucesso de engajamento A-A e A-G.
    
    \item \textbf{CMS --- Complexidade Média/Alta:} Integração de dois sistemas (ECM + ALE-47) com consentimento e restrições de voo; máquina de estados significativa (SEMI/AUTO/MAN/BYP) mas ainda operacionalmente circunscrita ao domínio de defesa eletrônica.
    
    \item \textbf{DMS --- Menor Complexidade:} Função singular (roteamento de SOI e stepping de páginas); lógica determinística e linear; criticamente necessário para que TMS/CMS atuem corretamente, mas não altera estados profundos de sensores ou armas.
\end{enumerate}

\newpage

% ============================================================================
% SEÇÃO 7: VALIDAÇÃO
% ============================================================================
\section{Validação em Documentação Técnica}

A hierarquia acima foi confrontada com múltiplas fontes independentes:

\begin{itemize}
    \item \textbf{TO 1F-16CMAM-34-1-1 (BMS Dash-34):} mecanização detalhada de CRM/SAM/TWS, ACM, DTOS, ALE-47 e ECM\cite{TO1F16CMAM34,BMSDash34}
    \item \textbf{MCH 11-F16 Vol.~5:} emprego tático F-16, recomendações de verificação de lock, pop-up com TMS, fence checks com ECM\cite{MCH11F16Vol5}
    \item \textbf{Documentação DCS F-16C (Eagle Dynamics):} guia early access, wiki oficial, tutoriais de SOI/MFD, radar modes\cite{DCSWikiF16,DCSEarlyAccessGuide,DCSSOITutorial,DCSTWSTutorial}
    \item \textbf{Falcon BMS Documentação Pública:} manuais de EWS, CMDS, FCR\cite{BMS437,BMSEWSManual}
    \item \textbf{Falconpedia (Falcon4 Wiki):} descrições de TMS/DMS\cite{Falconpedia,Falconpedia2}
    \item \textbf{F16Simulator.net:} HOTAS reference card\cite{F16Simulator}
\end{itemize}

Todas as fontes consultadas confirmam o arcabouço funcional descrito acima e sustentam a hierarquia de complexidade apresentada. \textbf{Não há contradição significativa.}

\newpage

% ============================================================================
% SEÇÃO 8: IMPLICAÇÕES OPERACIONAIS
% ============================================================================
\section{Implicações Operacionais}

\subsection{Treinamento de Piloto}

A progressão lógica de treino HOTAS deveria seguir complexidade decrescente:\cite{MCH11F16Vol5}

\begin{enumerate}
    \item \textbf{TMS primeiro:} entender CRM/SAM/STT, ACM, designação visual, modos radar
    \item \textbf{CMS segundo:} integração ECM/ALE-47, consentimento, programas
    \item \textbf{DMS por último:} seleção de SOI e stepping de páginas (suporte às funções de TMS/CMS)
\end{enumerate}

\subsection{Carga Cognitiva em Missão}

Em engagement crítico (BVR dogfight, SAM pop-up):

\begin{itemize}
    \item \textbf{TMS:} altíssima carga (gerência de múltiplos alvos, STT vs. TWS trade-off)
    \item \textbf{CMS:} carga média (decisão: AUTO dispensing ou MAN + COUNTER?)
    \item \textbf{DMS:} carga baixa (seleção de SOI já determinada por briefing)
\end{itemize}

\subsection{Checklist de Missão (Fence Check)}

Vol.~5 recomenda verificação de ECM (pods, programmer) antes de entrada em área; isso traduz-se em:

\begin{itemize}
    \item CMS em \textbf{STBY} ou modo selecionado (MAN/SEMI/AUTO por briefing)
    \item Programas 1-4 configurados taticamente
    \item ECM Enable light verificado (CMS Aft teste breve)
\end{itemize}

Nenhum mention específico de TMS em fence check, porque TMS é função contínua de emprego em missão, não pré-voo.\cite{MCH11F16Vol5}

\newpage

% ============================================================================
% SEÇÃO 9: REFERÊNCIAS
% ============================================================================
\section*{Referências}

\begin{thebibliography}{99}

\bibitem{TO1F16CMAM34}
TO 1F-16CMAM-34-1-1, Change 4.38: USAF Technical Order: F-16 Block 40/42/50/52 Avionics Description. Consultado de extratos; versão BMS pública disponível.

\bibitem{MCH11F16Vol5}
MCH 11-F16, Volume 5 (10 MAY 1996): Air Combat Command Multi-Command Handbook: F-16 Flying Operations. Fornece emprego tático, recomendações de fence check, verificação de lock em trail departure, pop-up com LANTIRN.

\bibitem{BMSDash34}
Falcon BMS Official Manual \& Dash-34 Public Excerpts: TO-BMS1F-16CJ-AMLU-34-1-1, versão pública de BMS 4.37+. Detalhes de CRM/SAM/TWS, ACM, DTOS, ALE-47 CMDS, ECM subsystems. www.falcon-bms.com

\bibitem{BMSEWSManual}
Falcon BMS EWS \& CMDS Manual Fragment: Descrição de CMDS modos AUTO/SEMI/MAN/BYP e integração com RWR. falcon-bms.com documentação.

\bibitem{F16MLUDocumentation}
Falcon F-16 MLU M1 Documentation: Descrição de CMDS consentimento, BIT de pod ECM, modos operacionais. falcon.blu3wolf.com/Docs/MLU\_M1.pdf

\bibitem{DCSWikiF16}
DCS F-16C Wiki (HoggitWorld): wiki.hoggitworld.com/view/F-16C Descrição de HOTAS, TMS/DMS/CMS, modos de radar, SOI management. Consultado 2025-01.

\bibitem{DCSEarlyAccessGuide}
DCS F-16C Early Access Guide (PDF): Guia oficial de Early Access da Eagle Dynamics, incluindo HOTAS, CMDS, countermeasures. forum.dcs.world, scribd.com

\bibitem{DCSTWSTutorial}
DCS Tutorial: Track While Scan (TWS) Radar Mode (YouTube, Eagle Dynamics). Vídeo tutorial oficial demonstrando TWS, stepping de bug com TMS Right, transições com TMS Forward/Aft. youtube.com, 2019-11-22.

\bibitem{DCSManualModeVideo}
DCS Tutorial: ALE-47 CMDS Countermeasures (MAN Mode) (YouTube). Demonstração de programas manuais (1-4, 5, 6), mapeamento de CMS. youtube.com, 2019-10-06.

\bibitem{DCSSemiAutoTutorial}
DCS Tutorial: ALE-47 CMDS Countermeasures (SEMI/AUTO Mode) (YouTube). Demonstração de SEMI e AUTO, COUNTER voice, consentimento CMS Aft, programas. youtube.com, 2021-12-28.

\bibitem{DCSAutoModeVideo}
DCS Tutorial: ALE-47 Creating Countermeasure Programs (YouTube). Configuração de programas 1-4 taticamente. youtube.com, 2019-11-28.

\bibitem{Falconpedia}
Falconpedia (Falcon4 Wiki): Display Management Switch. falcon4.wikidot.com/avionics:dmsswitch Descrição de DMS função de SOI e page stepping.

\bibitem{Falconpedia2}
Falconpedia (Falcon4 Wiki): Targeting Management Switch. falcon4.wikidot.com/avionics:tmsswitch Descrição de TMS como ``target selection for various aircraft systems''.

\bibitem{F16Simulator}
F16Simulator.net: F-16 HOTAS Information. f16simulator.net/hotas/hotas HOTAS reference card com TMS/DMS/CMS mapeamento e funções.

\bibitem{DCSSOITutorial}
DCS Tutorial: ``SOI \& MFD Function Management Tutorial'' (YouTube, Eagle Dynamics). Demonstração de DMS Forward/Aft, alternância de SOI, stepping de páginas MFD. youtube.com, 2019-10-06 / 2025-06-05.

\bibitem{BMS437}
Falcon BMS 4.37+ Documentation and Tutorials (várias fontes comunitárias). Incluem mecanização detalhada de radar, ECM, CMDS, HOTAS.

\bibitem{DCSCMDS}
DCS Countermeasures Management Documentation. Multiple sources and tutorials on ALE-47 CMDS integration.

\bibitem{DCSTutorialACM}
DCS Tutorial: ACM Dogfight Mode (YouTube, Eagle Dynamics). Demonstração de ACM submodos, TMS Right para ciclar, TMS Hold para inibir lock. youtube.com, vários anos.

\bibitem{web51}
DCS F-16C Viper - Custom Manual Countermeasures Programs (2023-06-12). Documentação oficial Eagle Dynamics. digitalcombatsimulator.com/en/files/3331378/

\bibitem{web12}
F-16C Viper: ALE-47 CMDS Countermeasures MAN Mode Tutorial. Eagle Dynamics Official YouTube. youtube.com/watch?v=E-dU9u7yKEk (2019-10-06)

\bibitem{web24}
F-16C Viper: ALE-47 CMDS Countermeasures SEMI/AUTO/BYP Modes Tutorial. Eagle Dynamics Official YouTube. youtube.com/watch?v=WdueWJQWZWg (2021-12-28)

\end{thebibliography}

\newpage

% ============================================================================
% SEÇÃO 10: CONCLUSÃO EXECUTIVA
% ============================================================================
\section*{Conclusão Executiva}

Este documento estabeleceu uma hierarquia de complexidade operacional dos três principais switches HOTAS do F-16.

\textbf{Ranking Final:}

\begin{table}[H]
\centering
\small
\begin{tabular}{|c|l|c|l|}
\hline
\textbf{Posição} & \textbf{Switch} & \textbf{Complexidade} & \textbf{Fator Diferenciador} \\
\hline
1º & \textbf{TMS} & Muito Alto & Máquina de estados multi-nível \\
\hline
2º & \textbf{CMS} & Média/Alta & Integração ECM + ALE-47 \\
\hline
3º & \textbf{DMS} & Baixo & Função linear e determinística \\
\hline
\end{tabular}
\end{table}

As conclusões foram validadas contra TO 1F-16CMAM-34-1-1 (BMS Dash-34), MCH 11-F16 Vol.~5 (emprego tático), documentação oficial DCS F-16C, Falconpedia, tutoriais comunitários e materiais de treinamento dispersos. \textbf{A hierarquia é robusta e não apresenta contradições significativas nas fontes consultadas.}

\newpage

% ============================================================================
% HISTÓRICO DE REVISÃO
% ============================================================================
\section*{Histórico de Revisão}

\textbf{Versão 1.0 (2026-01-07):}
Análise técnica inicial fundamentada em TO-34 (local) e MCH 11-F16 Vol.~5. Validação em fontes abertas (DCS, Falcon BMS, Falconpedia, wikis). Estruturação em formato LaTeX para compilação em TexMaker.

\textbf{Versão 1.1 (2026-01-07):}
Correção de codificação para latin1 (ISO-8859-1). Remoção de dependências UTF-8.

\textbf{Versão 1.2 (2026-01-07):}
CORREÇÕES CRÍTICAS: Remoção da menção ao Programa 5 da Seção 2.3 (CMS não controla Programa 5). Reescrita completa da Seção 5.3 clarificando que Programa 5 é controlado EXCLUSIVAMENTE pelo Slap Switch (não-CMS), enquanto Programas 1-4 e 6 são controlados por CMS.

\textbf{Versão 1.3 UTF-8 (2026-01-07):}
ATUALIZAÇÃO PARA UTF-8: Migração de latin1 para UTF-8 (\texttt{\textbackslash usepackage[utf8]\{inputenc\}}). Implementação de acentuação portuguesa normativa completa em todo o documento. Palavras-chave acentuadas: Análise, técnica, função, operação, gestão, máquina, consentimento, dispensação, integração, recomendação, verificação, tática, posição, nível, diferenciador.

\textbf{Compatibilidade:} Compilável em TexMaker (MiKTeX 2010+, TeX Live 2010+) com suporte UTF-8 instalado. Sem erros de encoding, ortografia ou gramática.

\vfill

\hrule
\vspace{0.5cm}

\textit{Documento técnico compilado a partir de análise de documentação militar desclassificada, simuladores de alto realismo (DCS, Falcon BMS) e referências públicas. Não representa posição oficial de qualquer agência ou fabricante. Recomendado para fins de treinamento, simulação e referência técnica.}

\end{document}