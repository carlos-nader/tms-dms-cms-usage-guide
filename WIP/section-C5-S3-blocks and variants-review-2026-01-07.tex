\documentclass[12pt,a4paper,oneside]{article}

\usepackage[T1]{fontenc}
\usepackage[english]{babel}
\usepackage[margin=2cm]{geometry}
\usepackage{hyperref}
% NOVO (para múltiplas seções):
\newcommand{\dashrefs}[1]{\textit{TO 1F-16CMAM-34-1-1}, Dash-34, sections \texttt{#1}}
\usepackage{xcolor}
\usepackage{amsmath}
\usepackage{amssymb}
\usepackage{booktabs}
\usepackage{graphicx}
\usepackage{fancyhdr}
\usepackage{setspace}
\usepackage{parskip}
\usepackage{soul} %

% Configuracao de hiperlinks
\hypersetup{
    colorlinks=true,
    linkcolor=blue,
    urlcolor=blue,
    citecolor=blue,
    pdftitle={F-16 HOTAS Analysis: CMS System - Under Revision},
    pdfauthor={Carlos "Metal" Nader},
    pdfkeywords={CMS, CMDS, ECM, F-16, HOTAS, IDIAS}
}

% Espacamento
\setstretch{1.15}
\setlength{\parindent}{0pt}
\setlength{\parskip}{12pt}

% Cabecalho e rodape
\pagestyle{fancy}
\fancyhead[L]{F-16 Countermeasures Management Switch (CMS)}
\fancyhead[R]{Section 5.1 - Under Revision}
\fancyfoot[C]{\thepage}
\fancyfoot[L]{TMS, DMS and CMS Usage Guide}
\fancyfoot[R]{\today}

% Formatacao de secoes
\usepackage{titlesec}
\titleformat{\section}
  {\normalfont\Large\bfseries\color{darkblue}}
  {\thesection}{1em}{}
\titleformat{\subsection}
  {\normalfont\large\bfseries\color{darkblue}}
  {\thesubsection}{1em}{}
\titleformat{\subsubsection}
  {\normalfont\normalsize\bfseries\color{darkblue}}
  {\thesubsubsection}{1em}{}
\titleformat{\paragraph}
  {\normalfont\normalsize\itshape\color{darkblue}}
  {}{0em}{}

\definecolor{darkblue}{RGB}{0,51,102}

\title{
  \textbf{Section 5.1: Concept and Interaction with CMDS/ECM/RWR}\\
}

\date{\today}

\renewcommand{\thesection}{\arabic{section}}
\renewcommand{\thesubsection}{\thesection.\arabic{subsection}}

\setcounter{section}{4}

\begin{document}

\newpage
\tableofcontents
\newpage

\section{CMS --- Countermeasures Management Switch}

\subsection{Concept and Interaction with CMDS / ECM / RWR}

\subsubsection{Concept}

\subsubsection{Interaction with CMDS / ECM}

\subsection{CMS Actuation (all modes)}

\newpage

\subsection{CMS Block and Variant Notes}

\begin{itemize}
\item \textbf {External ECM Pod Variants (ALQ-131/ALQ-184):}
\end{itemize} 

\begin{enumerate}
\item \textbf{USAF:} Blocks 40, 42, 50, 52 (all variants)
\item \textbf{NATO:} Belgium, Denmark, Netherlands, Norway (Block 15 and later)
\item \textbf{Others:} Republic of Korea Block 32, Egypt Blocks 32/40/52, F-16I Netz/Barak I legacy variants
\end{enumerate}
    
Operational procedures specify the use of CMS aft for ECM consent, XMIT knob for mode selection (1,2 and 3) and \hl{ALQ-131/ALQ-184 control buttons for program management} \marginpar{IA deve verificar veracidade}.

\begin{itemize}
\item \textbf{Integrated ECM Variants (IDIAS):}
\end{itemize}

\begin{enumerate}
    \item \textbf{Israel (IDFAF):} F-16I BARAK variants (Barak I, Barak II, Sufa), F-16C/D Blocks 30--40
    \item \textbf{Greece (HAF):} F-16C Blocks 50/52 (PXII, PXIII, PXIV)
    \item \textbf{Republic of Korea:} KF-16C Block 52 (upgrade from Block 32)
    \item \textbf{Singapore (RSAF):} F-16D Block 52
    \item \textbf{F-16 Block 52+ with IDIAS:} Any F-16 Block 52+ equipped with Improved Defensive Internal Avionics System
\end{enumerate}

Operational procedures specify the use of CMS LEFT for ECM consent (cycling STBY to AVNC to ECM to AVNC), XMTR switch (binary STBY/OPER) for power, and automatic band selection by RWR.

\begin{itemize}
\item \textbf{Critical Operation Difference}
\end{itemize}

Operating procedures that are correct for a Block 52 conventional aircraft with external ECM pod may produce unexpected or dangerous results on a Block 52+ IDIAS aircraft, and conversely. Specifically:

\begin{enumerate}
    \item Using CMS aft on an IDIAS aircraft will NOT enable ECM; only CMS LEFT will cycle into ECM operational states.
    \item The ECM XMT switch in a IDIAS aircraft is binary (STBY/OPER).
    \item IDIAS ECM automatically selects bands based on RWR threat.
\end{enumerate}

\end{document}